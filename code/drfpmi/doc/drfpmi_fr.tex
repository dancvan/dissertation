\documentclass[11pt]{article}
    \usepackage[breakable]{tcolorbox}
    \usepackage{parskip} % Stop auto-indenting (to mimic markdown behaviour)
    

    % Basic figure setup, for now with no caption control since it's done
    % automatically by Pandoc (which extracts ![](path) syntax from Markdown).
    \usepackage{graphicx}
    % Maintain compatibility with old templates. Remove in nbconvert 6.0
    \let\Oldincludegraphics\includegraphics
    % Ensure that by default, figures have no caption (until we provide a
    % proper Figure object with a Caption API and a way to capture that
    % in the conversion process - todo).
    \usepackage{caption}
    \DeclareCaptionFormat{nocaption}{}
    \captionsetup{format=nocaption,aboveskip=0pt,belowskip=0pt}

    \usepackage{float}
    \floatplacement{figure}{H} % forces figures to be placed at the correct location
    \usepackage{xcolor} % Allow colors to be defined
    \usepackage{enumerate} % Needed for markdown enumerations to work
    \usepackage{geometry} % Used to adjust the document margins
    \usepackage{amsmath} % Equations
    \usepackage{amssymb} % Equations
    \usepackage{textcomp} % defines textquotesingle
    % Hack from http://tex.stackexchange.com/a/47451/13684:
    \AtBeginDocument{%
        \def\PYZsq{\textquotesingle}% Upright quotes in Pygmentized code
    }
    \usepackage{upquote} % Upright quotes for verbatim code
    \usepackage{eurosym} % defines \euro

    \usepackage{iftex}
    \ifPDFTeX
        \usepackage[T1]{fontenc}
        \IfFileExists{alphabeta.sty}{
              \usepackage{alphabeta}
          }{
              \usepackage[mathletters]{ucs}
              \usepackage[utf8x]{inputenc}
          }
    \else
        \usepackage{fontspec}
        \usepackage{unicode-math}
    \fi

    \usepackage{fancyvrb} % verbatim replacement that allows latex
    \usepackage{grffile} % extends the file name processing of package graphics
                         % to support a larger range
    \makeatletter % fix for old versions of grffile with XeLaTeX
    \@ifpackagelater{grffile}{2019/11/01}
    {
      % Do nothing on new versions
    }
    {
      \def\Gread@@xetex#1{%
        \IfFileExists{"\Gin@base".bb}%
        {\Gread@eps{\Gin@base.bb}}%
        {\Gread@@xetex@aux#1}%
      }
    }
    \makeatother
    \usepackage[Export]{adjustbox} % Used to constrain images to a maximum size
    \adjustboxset{max size={0.9\linewidth}{0.9\paperheight}}

    % The hyperref package gives us a pdf with properly built
    % internal navigation ('pdf bookmarks' for the table of contents,
    % internal cross-reference links, web links for URLs, etc.)
    \usepackage{hyperref}
    % The default LaTeX title has an obnoxious amount of whitespace. By default,
    % titling removes some of it. It also provides customization options.
    \usepackage{titling}
    \usepackage{longtable} % longtable support required by pandoc >1.10
    \usepackage{booktabs}  % table support for pandoc > 1.12.2
    \usepackage{array}     % table support for pandoc >= 2.11.3
    \usepackage{calc}      % table minipage width calculation for pandoc >= 2.11.1
    \usepackage[inline]{enumitem} % IRkernel/repr support (it uses the enumerate* environment)
    \usepackage[normalem]{ulem} % ulem is needed to support strikethroughs (\sout)
                                % normalem makes italics be italics, not underlines
    \usepackage{soul}      % strikethrough (\st) support for pandoc >= 3.0.0
    \usepackage{mathrsfs}
    

    
    % Colors for the hyperref package
    \definecolor{urlcolor}{rgb}{0,.145,.698}
    \definecolor{linkcolor}{rgb}{.71,0.21,0.01}
    \definecolor{citecolor}{rgb}{.12,.54,.11}

    % ANSI colors
    \definecolor{ansi-black}{HTML}{3E424D}
    \definecolor{ansi-black-intense}{HTML}{282C36}
    \definecolor{ansi-red}{HTML}{E75C58}
    \definecolor{ansi-red-intense}{HTML}{B22B31}
    \definecolor{ansi-green}{HTML}{00A250}
    \definecolor{ansi-green-intense}{HTML}{007427}
    \definecolor{ansi-yellow}{HTML}{DDB62B}
    \definecolor{ansi-yellow-intense}{HTML}{B27D12}
    \definecolor{ansi-blue}{HTML}{208FFB}
    \definecolor{ansi-blue-intense}{HTML}{0065CA}
    \definecolor{ansi-magenta}{HTML}{D160C4}
    \definecolor{ansi-magenta-intense}{HTML}{A03196}
    \definecolor{ansi-cyan}{HTML}{60C6C8}
    \definecolor{ansi-cyan-intense}{HTML}{258F8F}
    \definecolor{ansi-white}{HTML}{C5C1B4}
    \definecolor{ansi-white-intense}{HTML}{A1A6B2}
    \definecolor{ansi-default-inverse-fg}{HTML}{FFFFFF}
    \definecolor{ansi-default-inverse-bg}{HTML}{000000}

    % common color for the border for error outputs.
    \definecolor{outerrorbackground}{HTML}{FFDFDF}

    % commands and environments needed by pandoc snippets
    % extracted from the output of `pandoc -s`
    \providecommand{\tightlist}{%
      \setlength{\itemsep}{0pt}\setlength{\parskip}{0pt}}
    \DefineVerbatimEnvironment{Highlighting}{Verbatim}{commandchars=\\\{\}}
    % Add ',fontsize=\small' for more characters per line
    \newenvironment{Shaded}{}{}
    \newcommand{\KeywordTok}[1]{\textcolor[rgb]{0.00,0.44,0.13}{\textbf{{#1}}}}
    \newcommand{\DataTypeTok}[1]{\textcolor[rgb]{0.56,0.13,0.00}{{#1}}}
    \newcommand{\DecValTok}[1]{\textcolor[rgb]{0.25,0.63,0.44}{{#1}}}
    \newcommand{\BaseNTok}[1]{\textcolor[rgb]{0.25,0.63,0.44}{{#1}}}
    \newcommand{\FloatTok}[1]{\textcolor[rgb]{0.25,0.63,0.44}{{#1}}}
    \newcommand{\CharTok}[1]{\textcolor[rgb]{0.25,0.44,0.63}{{#1}}}
    \newcommand{\StringTok}[1]{\textcolor[rgb]{0.25,0.44,0.63}{{#1}}}
    \newcommand{\CommentTok}[1]{\textcolor[rgb]{0.38,0.63,0.69}{\textit{{#1}}}}
    \newcommand{\OtherTok}[1]{\textcolor[rgb]{0.00,0.44,0.13}{{#1}}}
    \newcommand{\AlertTok}[1]{\textcolor[rgb]{1.00,0.00,0.00}{\textbf{{#1}}}}
    \newcommand{\FunctionTok}[1]{\textcolor[rgb]{0.02,0.16,0.49}{{#1}}}
    \newcommand{\RegionMarkerTok}[1]{{#1}}
    \newcommand{\ErrorTok}[1]{\textcolor[rgb]{1.00,0.00,0.00}{\textbf{{#1}}}}
    \newcommand{\NormalTok}[1]{{#1}}

    % Additional commands for more recent versions of Pandoc
    \newcommand{\ConstantTok}[1]{\textcolor[rgb]{0.53,0.00,0.00}{{#1}}}
    \newcommand{\SpecialCharTok}[1]{\textcolor[rgb]{0.25,0.44,0.63}{{#1}}}
    \newcommand{\VerbatimStringTok}[1]{\textcolor[rgb]{0.25,0.44,0.63}{{#1}}}
    \newcommand{\SpecialStringTok}[1]{\textcolor[rgb]{0.73,0.40,0.53}{{#1}}}
    \newcommand{\ImportTok}[1]{{#1}}
    \newcommand{\DocumentationTok}[1]{\textcolor[rgb]{0.73,0.13,0.13}{\textit{{#1}}}}
    \newcommand{\AnnotationTok}[1]{\textcolor[rgb]{0.38,0.63,0.69}{\textbf{\textit{{#1}}}}}
    \newcommand{\CommentVarTok}[1]{\textcolor[rgb]{0.38,0.63,0.69}{\textbf{\textit{{#1}}}}}
    \newcommand{\VariableTok}[1]{\textcolor[rgb]{0.10,0.09,0.49}{{#1}}}
    \newcommand{\ControlFlowTok}[1]{\textcolor[rgb]{0.00,0.44,0.13}{\textbf{{#1}}}}
    \newcommand{\OperatorTok}[1]{\textcolor[rgb]{0.40,0.40,0.40}{{#1}}}
    \newcommand{\BuiltInTok}[1]{{#1}}
    \newcommand{\ExtensionTok}[1]{{#1}}
    \newcommand{\PreprocessorTok}[1]{\textcolor[rgb]{0.74,0.48,0.00}{{#1}}}
    \newcommand{\AttributeTok}[1]{\textcolor[rgb]{0.49,0.56,0.16}{{#1}}}
    \newcommand{\InformationTok}[1]{\textcolor[rgb]{0.38,0.63,0.69}{\textbf{\textit{{#1}}}}}
    \newcommand{\WarningTok}[1]{\textcolor[rgb]{0.38,0.63,0.69}{\textbf{\textit{{#1}}}}}


    % Define a nice break command that doesn't care if a line doesn't already
    % exist.
    \def\br{\hspace*{\fill} \\* }
    % Math Jax compatibility definitions
    \def\gt{>}
    \def\lt{<}
    \let\Oldtex\TeX
    \let\Oldlatex\LaTeX
    \renewcommand{\TeX}{\textrm{\Oldtex}}
    \renewcommand{\LaTeX}{\textrm{\Oldlatex}}
    % Document parameters
    % Document title
    \title{drfpmi\_fr}
    
    
    
    
    
    
    
% Pygments definitions
\makeatletter
\def\PY@reset{\let\PY@it=\relax \let\PY@bf=\relax%
    \let\PY@ul=\relax \let\PY@tc=\relax%
    \let\PY@bc=\relax \let\PY@ff=\relax}
\def\PY@tok#1{\csname PY@tok@#1\endcsname}
\def\PY@toks#1+{\ifx\relax#1\empty\else%
    \PY@tok{#1}\expandafter\PY@toks\fi}
\def\PY@do#1{\PY@bc{\PY@tc{\PY@ul{%
    \PY@it{\PY@bf{\PY@ff{#1}}}}}}}
\def\PY#1#2{\PY@reset\PY@toks#1+\relax+\PY@do{#2}}

\@namedef{PY@tok@w}{\def\PY@tc##1{\textcolor[rgb]{0.73,0.73,0.73}{##1}}}
\@namedef{PY@tok@c}{\let\PY@it=\textit\def\PY@tc##1{\textcolor[rgb]{0.24,0.48,0.48}{##1}}}
\@namedef{PY@tok@cp}{\def\PY@tc##1{\textcolor[rgb]{0.61,0.40,0.00}{##1}}}
\@namedef{PY@tok@k}{\let\PY@bf=\textbf\def\PY@tc##1{\textcolor[rgb]{0.00,0.50,0.00}{##1}}}
\@namedef{PY@tok@kp}{\def\PY@tc##1{\textcolor[rgb]{0.00,0.50,0.00}{##1}}}
\@namedef{PY@tok@kt}{\def\PY@tc##1{\textcolor[rgb]{0.69,0.00,0.25}{##1}}}
\@namedef{PY@tok@o}{\def\PY@tc##1{\textcolor[rgb]{0.40,0.40,0.40}{##1}}}
\@namedef{PY@tok@ow}{\let\PY@bf=\textbf\def\PY@tc##1{\textcolor[rgb]{0.67,0.13,1.00}{##1}}}
\@namedef{PY@tok@nb}{\def\PY@tc##1{\textcolor[rgb]{0.00,0.50,0.00}{##1}}}
\@namedef{PY@tok@nf}{\def\PY@tc##1{\textcolor[rgb]{0.00,0.00,1.00}{##1}}}
\@namedef{PY@tok@nc}{\let\PY@bf=\textbf\def\PY@tc##1{\textcolor[rgb]{0.00,0.00,1.00}{##1}}}
\@namedef{PY@tok@nn}{\let\PY@bf=\textbf\def\PY@tc##1{\textcolor[rgb]{0.00,0.00,1.00}{##1}}}
\@namedef{PY@tok@ne}{\let\PY@bf=\textbf\def\PY@tc##1{\textcolor[rgb]{0.80,0.25,0.22}{##1}}}
\@namedef{PY@tok@nv}{\def\PY@tc##1{\textcolor[rgb]{0.10,0.09,0.49}{##1}}}
\@namedef{PY@tok@no}{\def\PY@tc##1{\textcolor[rgb]{0.53,0.00,0.00}{##1}}}
\@namedef{PY@tok@nl}{\def\PY@tc##1{\textcolor[rgb]{0.46,0.46,0.00}{##1}}}
\@namedef{PY@tok@ni}{\let\PY@bf=\textbf\def\PY@tc##1{\textcolor[rgb]{0.44,0.44,0.44}{##1}}}
\@namedef{PY@tok@na}{\def\PY@tc##1{\textcolor[rgb]{0.41,0.47,0.13}{##1}}}
\@namedef{PY@tok@nt}{\let\PY@bf=\textbf\def\PY@tc##1{\textcolor[rgb]{0.00,0.50,0.00}{##1}}}
\@namedef{PY@tok@nd}{\def\PY@tc##1{\textcolor[rgb]{0.67,0.13,1.00}{##1}}}
\@namedef{PY@tok@s}{\def\PY@tc##1{\textcolor[rgb]{0.73,0.13,0.13}{##1}}}
\@namedef{PY@tok@sd}{\let\PY@it=\textit\def\PY@tc##1{\textcolor[rgb]{0.73,0.13,0.13}{##1}}}
\@namedef{PY@tok@si}{\let\PY@bf=\textbf\def\PY@tc##1{\textcolor[rgb]{0.64,0.35,0.47}{##1}}}
\@namedef{PY@tok@se}{\let\PY@bf=\textbf\def\PY@tc##1{\textcolor[rgb]{0.67,0.36,0.12}{##1}}}
\@namedef{PY@tok@sr}{\def\PY@tc##1{\textcolor[rgb]{0.64,0.35,0.47}{##1}}}
\@namedef{PY@tok@ss}{\def\PY@tc##1{\textcolor[rgb]{0.10,0.09,0.49}{##1}}}
\@namedef{PY@tok@sx}{\def\PY@tc##1{\textcolor[rgb]{0.00,0.50,0.00}{##1}}}
\@namedef{PY@tok@m}{\def\PY@tc##1{\textcolor[rgb]{0.40,0.40,0.40}{##1}}}
\@namedef{PY@tok@gh}{\let\PY@bf=\textbf\def\PY@tc##1{\textcolor[rgb]{0.00,0.00,0.50}{##1}}}
\@namedef{PY@tok@gu}{\let\PY@bf=\textbf\def\PY@tc##1{\textcolor[rgb]{0.50,0.00,0.50}{##1}}}
\@namedef{PY@tok@gd}{\def\PY@tc##1{\textcolor[rgb]{0.63,0.00,0.00}{##1}}}
\@namedef{PY@tok@gi}{\def\PY@tc##1{\textcolor[rgb]{0.00,0.52,0.00}{##1}}}
\@namedef{PY@tok@gr}{\def\PY@tc##1{\textcolor[rgb]{0.89,0.00,0.00}{##1}}}
\@namedef{PY@tok@ge}{\let\PY@it=\textit}
\@namedef{PY@tok@gs}{\let\PY@bf=\textbf}
\@namedef{PY@tok@ges}{\let\PY@bf=\textbf\let\PY@it=\textit}
\@namedef{PY@tok@gp}{\let\PY@bf=\textbf\def\PY@tc##1{\textcolor[rgb]{0.00,0.00,0.50}{##1}}}
\@namedef{PY@tok@go}{\def\PY@tc##1{\textcolor[rgb]{0.44,0.44,0.44}{##1}}}
\@namedef{PY@tok@gt}{\def\PY@tc##1{\textcolor[rgb]{0.00,0.27,0.87}{##1}}}
\@namedef{PY@tok@err}{\def\PY@bc##1{{\setlength{\fboxsep}{\string -\fboxrule}\fcolorbox[rgb]{1.00,0.00,0.00}{1,1,1}{\strut ##1}}}}
\@namedef{PY@tok@kc}{\let\PY@bf=\textbf\def\PY@tc##1{\textcolor[rgb]{0.00,0.50,0.00}{##1}}}
\@namedef{PY@tok@kd}{\let\PY@bf=\textbf\def\PY@tc##1{\textcolor[rgb]{0.00,0.50,0.00}{##1}}}
\@namedef{PY@tok@kn}{\let\PY@bf=\textbf\def\PY@tc##1{\textcolor[rgb]{0.00,0.50,0.00}{##1}}}
\@namedef{PY@tok@kr}{\let\PY@bf=\textbf\def\PY@tc##1{\textcolor[rgb]{0.00,0.50,0.00}{##1}}}
\@namedef{PY@tok@bp}{\def\PY@tc##1{\textcolor[rgb]{0.00,0.50,0.00}{##1}}}
\@namedef{PY@tok@fm}{\def\PY@tc##1{\textcolor[rgb]{0.00,0.00,1.00}{##1}}}
\@namedef{PY@tok@vc}{\def\PY@tc##1{\textcolor[rgb]{0.10,0.09,0.49}{##1}}}
\@namedef{PY@tok@vg}{\def\PY@tc##1{\textcolor[rgb]{0.10,0.09,0.49}{##1}}}
\@namedef{PY@tok@vi}{\def\PY@tc##1{\textcolor[rgb]{0.10,0.09,0.49}{##1}}}
\@namedef{PY@tok@vm}{\def\PY@tc##1{\textcolor[rgb]{0.10,0.09,0.49}{##1}}}
\@namedef{PY@tok@sa}{\def\PY@tc##1{\textcolor[rgb]{0.73,0.13,0.13}{##1}}}
\@namedef{PY@tok@sb}{\def\PY@tc##1{\textcolor[rgb]{0.73,0.13,0.13}{##1}}}
\@namedef{PY@tok@sc}{\def\PY@tc##1{\textcolor[rgb]{0.73,0.13,0.13}{##1}}}
\@namedef{PY@tok@dl}{\def\PY@tc##1{\textcolor[rgb]{0.73,0.13,0.13}{##1}}}
\@namedef{PY@tok@s2}{\def\PY@tc##1{\textcolor[rgb]{0.73,0.13,0.13}{##1}}}
\@namedef{PY@tok@sh}{\def\PY@tc##1{\textcolor[rgb]{0.73,0.13,0.13}{##1}}}
\@namedef{PY@tok@s1}{\def\PY@tc##1{\textcolor[rgb]{0.73,0.13,0.13}{##1}}}
\@namedef{PY@tok@mb}{\def\PY@tc##1{\textcolor[rgb]{0.40,0.40,0.40}{##1}}}
\@namedef{PY@tok@mf}{\def\PY@tc##1{\textcolor[rgb]{0.40,0.40,0.40}{##1}}}
\@namedef{PY@tok@mh}{\def\PY@tc##1{\textcolor[rgb]{0.40,0.40,0.40}{##1}}}
\@namedef{PY@tok@mi}{\def\PY@tc##1{\textcolor[rgb]{0.40,0.40,0.40}{##1}}}
\@namedef{PY@tok@il}{\def\PY@tc##1{\textcolor[rgb]{0.40,0.40,0.40}{##1}}}
\@namedef{PY@tok@mo}{\def\PY@tc##1{\textcolor[rgb]{0.40,0.40,0.40}{##1}}}
\@namedef{PY@tok@ch}{\let\PY@it=\textit\def\PY@tc##1{\textcolor[rgb]{0.24,0.48,0.48}{##1}}}
\@namedef{PY@tok@cm}{\let\PY@it=\textit\def\PY@tc##1{\textcolor[rgb]{0.24,0.48,0.48}{##1}}}
\@namedef{PY@tok@cpf}{\let\PY@it=\textit\def\PY@tc##1{\textcolor[rgb]{0.24,0.48,0.48}{##1}}}
\@namedef{PY@tok@c1}{\let\PY@it=\textit\def\PY@tc##1{\textcolor[rgb]{0.24,0.48,0.48}{##1}}}
\@namedef{PY@tok@cs}{\let\PY@it=\textit\def\PY@tc##1{\textcolor[rgb]{0.24,0.48,0.48}{##1}}}

\def\PYZbs{\char`\\}
\def\PYZus{\char`\_}
\def\PYZob{\char`\{}
\def\PYZcb{\char`\}}
\def\PYZca{\char`\^}
\def\PYZam{\char`\&}
\def\PYZlt{\char`\<}
\def\PYZgt{\char`\>}
\def\PYZsh{\char`\#}
\def\PYZpc{\char`\%}
\def\PYZdl{\char`\$}
\def\PYZhy{\char`\-}
\def\PYZsq{\char`\'}
\def\PYZdq{\char`\"}
\def\PYZti{\char`\~}
% for compatibility with earlier versions
\def\PYZat{@}
\def\PYZlb{[}
\def\PYZrb{]}
\makeatother


    % For linebreaks inside Verbatim environment from package fancyvrb.
    \makeatletter
        \newbox\Wrappedcontinuationbox
        \newbox\Wrappedvisiblespacebox
        \newcommand*\Wrappedvisiblespace {\textcolor{red}{\textvisiblespace}}
        \newcommand*\Wrappedcontinuationsymbol {\textcolor{red}{\llap{\tiny$\m@th\hookrightarrow$}}}
        \newcommand*\Wrappedcontinuationindent {3ex }
        \newcommand*\Wrappedafterbreak {\kern\Wrappedcontinuationindent\copy\Wrappedcontinuationbox}
        % Take advantage of the already applied Pygments mark-up to insert
        % potential linebreaks for TeX processing.
        %        {, <, #, %, $, ' and ": go to next line.
        %        _, }, ^, &, >, - and ~: stay at end of broken line.
        % Use of \textquotesingle for straight quote.
        \newcommand*\Wrappedbreaksatspecials {%
            \def\PYGZus{\discretionary{\char`\_}{\Wrappedafterbreak}{\char`\_}}%
            \def\PYGZob{\discretionary{}{\Wrappedafterbreak\char`\{}{\char`\{}}%
            \def\PYGZcb{\discretionary{\char`\}}{\Wrappedafterbreak}{\char`\}}}%
            \def\PYGZca{\discretionary{\char`\^}{\Wrappedafterbreak}{\char`\^}}%
            \def\PYGZam{\discretionary{\char`\&}{\Wrappedafterbreak}{\char`\&}}%
            \def\PYGZlt{\discretionary{}{\Wrappedafterbreak\char`\<}{\char`\<}}%
            \def\PYGZgt{\discretionary{\char`\>}{\Wrappedafterbreak}{\char`\>}}%
            \def\PYGZsh{\discretionary{}{\Wrappedafterbreak\char`\#}{\char`\#}}%
            \def\PYGZpc{\discretionary{}{\Wrappedafterbreak\char`\%}{\char`\%}}%
            \def\PYGZdl{\discretionary{}{\Wrappedafterbreak\char`\$}{\char`\$}}%
            \def\PYGZhy{\discretionary{\char`\-}{\Wrappedafterbreak}{\char`\-}}%
            \def\PYGZsq{\discretionary{}{\Wrappedafterbreak\textquotesingle}{\textquotesingle}}%
            \def\PYGZdq{\discretionary{}{\Wrappedafterbreak\char`\"}{\char`\"}}%
            \def\PYGZti{\discretionary{\char`\~}{\Wrappedafterbreak}{\char`\~}}%
        }
        % Some characters . , ; ? ! / are not pygmentized.
        % This macro makes them "active" and they will insert potential linebreaks
        \newcommand*\Wrappedbreaksatpunct {%
            \lccode`\~`\.\lowercase{\def~}{\discretionary{\hbox{\char`\.}}{\Wrappedafterbreak}{\hbox{\char`\.}}}%
            \lccode`\~`\,\lowercase{\def~}{\discretionary{\hbox{\char`\,}}{\Wrappedafterbreak}{\hbox{\char`\,}}}%
            \lccode`\~`\;\lowercase{\def~}{\discretionary{\hbox{\char`\;}}{\Wrappedafterbreak}{\hbox{\char`\;}}}%
            \lccode`\~`\:\lowercase{\def~}{\discretionary{\hbox{\char`\:}}{\Wrappedafterbreak}{\hbox{\char`\:}}}%
            \lccode`\~`\?\lowercase{\def~}{\discretionary{\hbox{\char`\?}}{\Wrappedafterbreak}{\hbox{\char`\?}}}%
            \lccode`\~`\!\lowercase{\def~}{\discretionary{\hbox{\char`\!}}{\Wrappedafterbreak}{\hbox{\char`\!}}}%
            \lccode`\~`\/\lowercase{\def~}{\discretionary{\hbox{\char`\/}}{\Wrappedafterbreak}{\hbox{\char`\/}}}%
            \catcode`\.\active
            \catcode`\,\active
            \catcode`\;\active
            \catcode`\:\active
            \catcode`\?\active
            \catcode`\!\active
            \catcode`\/\active
            \lccode`\~`\~
        }
    \makeatother

    \let\OriginalVerbatim=\Verbatim
    \makeatletter
    \renewcommand{\Verbatim}[1][1]{%
        %\parskip\z@skip
        \sbox\Wrappedcontinuationbox {\Wrappedcontinuationsymbol}%
        \sbox\Wrappedvisiblespacebox {\FV@SetupFont\Wrappedvisiblespace}%
        \def\FancyVerbFormatLine ##1{\hsize\linewidth
            \vtop{\raggedright\hyphenpenalty\z@\exhyphenpenalty\z@
                \doublehyphendemerits\z@\finalhyphendemerits\z@
                \strut ##1\strut}%
        }%
        % If the linebreak is at a space, the latter will be displayed as visible
        % space at end of first line, and a continuation symbol starts next line.
        % Stretch/shrink are however usually zero for typewriter font.
        \def\FV@Space {%
            \nobreak\hskip\z@ plus\fontdimen3\font minus\fontdimen4\font
            \discretionary{\copy\Wrappedvisiblespacebox}{\Wrappedafterbreak}
            {\kern\fontdimen2\font}%
        }%

        % Allow breaks at special characters using \PYG... macros.
        \Wrappedbreaksatspecials
        % Breaks at punctuation characters . , ; ? ! and / need catcode=\active
        \OriginalVerbatim[#1,codes*=\Wrappedbreaksatpunct]%
    }
    \makeatother

    % Exact colors from NB
    \definecolor{incolor}{HTML}{303F9F}
    \definecolor{outcolor}{HTML}{D84315}
    \definecolor{cellborder}{HTML}{CFCFCF}
    \definecolor{cellbackground}{HTML}{F7F7F7}

    % prompt
    \makeatletter
    \newcommand{\boxspacing}{\kern\kvtcb@left@rule\kern\kvtcb@boxsep}
    \makeatother
    \newcommand{\prompt}[4]{
        {\ttfamily\llap{{\color{#2}[#3]:\hspace{3pt}#4}}\vspace{-\baselineskip}}
    }
    

    
    % Prevent overflowing lines due to hard-to-break entities
    \sloppy
    % Setup hyperref package
    \hypersetup{
      breaklinks=true,  % so long urls are correctly broken across lines
      colorlinks=true,
      urlcolor=urlcolor,
      linkcolor=linkcolor,
      citecolor=citecolor,
      }
    % Slightly bigger margins than the latex defaults
    
    \geometry{verbose,tmargin=1in,bmargin=1in,lmargin=1in,rmargin=1in}
    
    

\pagenumbering{gobble}
\begin{document}
    
    
    

    
    \hypertarget{drfpmi-frequency-response}{%
\section{DRFPMI frequency response*}\label{drfpmi-frequency-response}}

    \begin{tcolorbox}[breakable, size=fbox, boxrule=1pt, pad at break*=1mm,colback=cellbackground, colframe=cellborder]
\prompt{In}{incolor}{1}{\boxspacing}
\begin{Verbatim}[commandchars=\\\{\}]
\PY{k+kn}{import} \PY{n+nn}{numpy} \PY{k}{as} \PY{n+nn}{np} 
\PY{k+kn}{import} \PY{n+nn}{matplotlib}\PY{n+nn}{.}\PY{n+nn}{pyplot} \PY{k}{as} \PY{n+nn}{plt}
\PY{k+kn}{import} \PY{n+nn}{scipy}\PY{n+nn}{.}\PY{n+nn}{signal} \PY{k}{as} \PY{n+nn}{sig}
\PY{k+kn}{import} \PY{n+nn}{os}
\PY{k+kn}{import} \PY{n+nn}{ifo\PYZus{}configs} \PY{k}{as} \PY{n+nn}{ifco}
\PY{n}{plt\PYZus{}style\PYZus{}dir} \PY{o}{=} \PY{l+s+s1}{\PYZsq{}}\PY{l+s+s1}{stash/}\PY{l+s+s1}{\PYZsq{}}
\PY{k}{if} \PY{n}{os}\PY{o}{.}\PY{n}{path}\PY{o}{.}\PY{n}{isdir}\PY{p}{(}\PY{n}{plt\PYZus{}style\PYZus{}dir}\PY{p}{)} \PY{o}{==} \PY{k+kc}{True}\PY{p}{:}
    \PY{n}{plt}\PY{o}{.}\PY{n}{style}\PY{o}{.}\PY{n}{use}\PY{p}{(}\PY{n}{plt\PYZus{}style\PYZus{}dir} \PY{o}{+} \PY{l+s+s1}{\PYZsq{}}\PY{l+s+s1}{ppt2latexsubfig.mplstyle}\PY{l+s+s1}{\PYZsq{}}\PY{p}{)}
\PY{n}{plt}\PY{o}{.}\PY{n}{rcParams}\PY{p}{[}\PY{l+s+s2}{\PYZdq{}}\PY{l+s+s2}{font.family}\PY{l+s+s2}{\PYZdq{}}\PY{p}{]} \PY{o}{=} \PY{l+s+s2}{\PYZdq{}}\PY{l+s+s2}{serif}\PY{l+s+s2}{\PYZdq{}}
\PY{n}{plt}\PY{o}{.}\PY{n}{rcParams}\PY{p}{[}\PY{l+s+s2}{\PYZdq{}}\PY{l+s+s2}{font.serif}\PY{l+s+s2}{\PYZdq{}}\PY{p}{]} \PY{o}{=} \PY{p}{[}\PY{l+s+s2}{\PYZdq{}}\PY{l+s+s2}{Times New Roman}\PY{l+s+s2}{\PYZdq{}}\PY{p}{]} \PY{o}{+} \PY{n}{plt}\PY{o}{.}\PY{n}{rcParams}\PY{p}{[}\PY{l+s+s2}{\PYZdq{}}\PY{l+s+s2}{font.serif}\PY{l+s+s2}{\PYZdq{}}\PY{p}{]}
\PY{n}{line\PYZus{}width}\PY{o}{=}\PY{l+m+mf}{7.5}
\end{Verbatim}
\end{tcolorbox}

    Up to this point we can understand how the FPMI repsonse function works:
\[ H_{FPMI}(\omega_g) = \frac{2 \Delta \phi_r(\omega_g)}{h(\omega_g)} =  \frac{t_1^2r_2}{(t_1^2 + r_1^2)r_2 -r_1} \frac{H_{\mathrm{MI}}(\omega_g, L)}{1-r_1r_2e^{-2i \omega_g L /c }}  \]

    \begin{tcolorbox}[breakable, size=fbox, boxrule=1pt, pad at break*=1mm,colback=cellbackground, colframe=cellborder]
\prompt{In}{incolor}{2}{\boxspacing}
\begin{Verbatim}[commandchars=\\\{\}]
\PY{c+c1}{\PYZsh{} Some parameters}
\PY{n}{cee} \PY{o}{=} \PY{n}{np}\PY{o}{.}\PY{n}{float64}\PY{p}{(}\PY{l+m+mi}{299792458}\PY{p}{)}
\PY{n}{OMEG} \PY{o}{=} \PY{n}{np}\PY{o}{.}\PY{n}{float64}\PY{p}{(}\PY{l+m+mi}{2}\PY{o}{*}\PY{n}{np}\PY{o}{.}\PY{n}{pi}\PY{o}{*}\PY{n}{cee}\PY{o}{/}\PY{p}{(}\PY{l+m+mf}{1064.0}\PY{o}{*}\PY{l+m+mf}{1e\PYZhy{}9}\PY{p}{)}\PY{p}{)}
\PY{n}{L} \PY{o}{=} \PY{n}{np}\PY{o}{.}\PY{n}{float64}\PY{p}{(}\PY{l+m+mf}{4000.0}\PY{p}{)}
\PY{n}{nu} \PY{o}{=} \PY{n}{np}\PY{o}{.}\PY{n}{arange}\PY{p}{(}\PY{l+m+mi}{1}\PY{p}{,} \PY{l+m+mi}{1000000}\PY{p}{,} \PY{l+m+mi}{1}\PY{p}{)}
\PY{n}{nat\PYZus{}nu} \PY{o}{=} \PY{p}{[}\PY{n}{np}\PY{o}{.}\PY{n}{float64}\PY{p}{(}\PY{n}{i}\PY{o}{*}\PY{l+m+mi}{2}\PY{o}{*}\PY{n}{np}\PY{o}{.}\PY{n}{pi}\PY{p}{)} \PY{k}{for} \PY{n}{i} \PY{o+ow}{in} \PY{n}{nu}\PY{p}{]}
\PY{n}{h\PYZus{}0} \PY{o}{=} \PY{n}{np}\PY{o}{.}\PY{n}{float64}\PY{p}{(}\PY{l+m+mi}{1}\PY{p}{)}

\PY{n}{T\PYZus{}1} \PY{o}{=} \PY{l+m+mf}{.014}
\PY{c+c1}{\PYZsh{}T\PYZus{}1 = 25e\PYZhy{}6 }
\PY{n}{T\PYZus{}2} \PY{o}{=} \PY{l+m+mf}{50e\PYZhy{}6}
\PY{n}{R\PYZus{}1} \PY{o}{=} \PY{l+m+mi}{1}\PY{o}{\PYZhy{}}\PY{n}{T\PYZus{}1}
\PY{n}{R\PYZus{}2} \PY{o}{=} \PY{l+m+mi}{1}\PY{o}{\PYZhy{}}\PY{n}{T\PYZus{}2}

\PY{n}{t\PYZus{}1} \PY{o}{=} \PY{n}{T\PYZus{}1}\PY{o}{*}\PY{o}{*}\PY{l+m+mf}{.5}
\PY{n}{r\PYZus{}1} \PY{o}{=} \PY{n}{R\PYZus{}1}\PY{o}{*}\PY{o}{*}\PY{l+m+mf}{.5}
\PY{n}{r\PYZus{}2} \PY{o}{=} \PY{n}{R\PYZus{}2}\PY{o}{*}\PY{o}{*}\PY{l+m+mf}{.5} 

\PY{n}{PHI\PYZus{}0} \PY{o}{=} \PY{n}{np}\PY{o}{.}\PY{n}{pi}\PY{o}{/}\PY{l+m+mi}{2} 
\PY{n}{P\PYZus{}IN} \PY{o}{=} \PY{l+m+mi}{25}
\end{Verbatim}
\end{tcolorbox}

    \hypertarget{power-recycling}{%
\section{POWER RECYCLING}\label{power-recycling}}

    \hypertarget{derivation}{%
\subsection{Derivation}\label{derivation}}

With all the power going to the symmetric port, the nominal operating
state of the FPMI involves a significant amount of dumped / wasted
power. Placing a mirror at the symmetric port can allow that power to be
recycled. Though considerations must be made to maximize the amount of
recycling gain you can acquire with your GW detector. This is dependent
on the placement of the power recycling mirror (PRM) and its
reflectivity, transmission, and loss coefficients.

But first, the field at the symmetric port:

\[E_\mathrm{SYM} = \frac{E_i}{2}e^{2ikl}(r_\mathrm{FP,X} + r_\mathrm{FP,Y}) \]

This is realized through observing the circulating power between the PRM
and the short Michelson:

\[ E_\mathrm{PRC} = \frac{t_\mathrm{PRM}}{1- r_\mathrm{PRM} r_\mathrm{FPMI} e^{2ik (L_\mathrm{PRC2BS} + L_\mathrm{SMICH})}}E_\mathrm{in} \]

Where:

\[ L_\mathrm{SMICH} = l_x + l_y \]

    Now let's observe the cavity reflection parameter:

\[ r_\mathrm{FP} = -r_1 + \frac{t_1^2 r_2 e^{i2kL}}{1-r_1 r_2 e^{i2kL}} = -\frac{\mathcal{F}}{\pi} \Big[-\Big(\frac{r_1}{r_2} \Big)^{1/2} + \Big(\frac{r_2}{r_1}\Big)^{1/2} (r_1^2 + t_1^2) \Big]\]

    But with loss considerations:

\[ r_\mathrm{FP} = -r_1 + \frac{t_1^2 r_2 e^{- t_\mathrm{RT}/\tau_\mathrm{loss}}  e^{i2kL}}{1-r_1 r_2 e^{- t_\mathrm{RT}/\tau_\mathrm{loss}} e^{i2kL}} \approx -\frac{\mathcal{F}}{\pi} \Big[\frac{-r_1 +  r_2(r_1^2 + t_1^2)(1-\mathscr{L}_\mathrm{RT})}{\sqrt{r_1 r_2}} \Big]\]

    we know that \(t_1^2 << r_1^2\):

\[r_\mathrm{FP} \approx -\frac{\mathcal{F}}{\pi} \Big[ \frac{r_1(-1 + (1 - \pi/\mathcal{F}) (1- \mathscr{L}_\mathrm{RT}))}{\sqrt{r_1 r_2}} \Big] \approx  -\Big(\frac{r_1}{r_2}\Big)^{1/2} \frac{\mathcal{F}}{\pi} \Big[- \pi/\mathcal{F} - \mathscr{L}_\mathrm{RT} + (\mathscr{L}_\mathrm{RT}\pi)/\mathcal{F}) \Big] \]

    And \(\mathscr{L}_\mathrm{RT} <<1\) with \(r_1 /r_2 \approx 1\) we get:

\[r_\mathrm{FP} \approx -1 + \frac{\mathcal{F}}{\pi} \mathscr{L}_\mathrm{RT}\]

    If we're operating at a dark fringe, at the symmetric port we see
superimposed fields:

\[ E_\mathrm{SYM}  = \frac{E_i}{2} \Big[ r_\mathrm{FPX}e^{2ik\mathscr{l}_x} + r_\mathrm{FPY}e^{2ik\mathscr{l}_y} \Big] \]

Where we assume that the short Michelson arms and reflection
coefficients are roughly equal (\(\mathscr{l}_x = \mathscr{l}_y\),
\(r_\mathrm{FPX} = r_\mathrm{FPY}\))

We also can average the short Michelson arm lengths
\((\mathscr{l}_x + \mathscr{l}_y)/2\) such that the effective reflection
coefficient is:
\(r_\mathrm{FPMI} = e^{2ik\mathscr{l}}(- 1 + \frac{\mathcal{F}}{\pi} \mathscr{L}_\mathrm{RT})\)

    Knowing this we create the following expression for the circulating
power within the cavity:

\[ P_\mathrm{PRC} = \frac{|t_\mathrm{PRM}|^2}{|1-r_\mathrm{PRM} r_\mathrm{FPMI} e^{2ik(L_\mathrm{PRC2BS} + L_\mathrm{SMICH})}|^2} P_\mathrm{in}\]

where \[t\_\mathrm{PRM}|^{2} = 1 -
r\_\mathrm{PRM}|^{2} \] and given a carrier resonance
condition we want to maximize the power with a variable PRM
reflectivity:

\[\frac{\partial P_\mathrm{PRC}}{\partial r_\mathrm{PRM}} = \frac{2r_\mathrm{PRM}^2(r_\mathrm{FPMI} - r_\mathrm{PRM})}{(1 - r_\mathrm{PRM} r_\mathrm{FPMI})^3} = 0\]

which sets $ r\_\mathrm{PRM} = r\_\mathrm{FPMI} $

    On resonance, the power recyling gain
(\(G_\mathrm{PR} = \frac{P_\mathrm{PRC}}{P_\mathrm{in}}\)):

\[ G_\mathrm{PR} = \frac{\pi}{2 \mathcal{F} \mathscr{L}_\mathrm{RT}} \Bigg[ \frac{1}{1- \frac{\mathcal{F}\mathscr{L}_\mathrm{RT}}{2 \pi}} \Bigg] \]

    \begin{tcolorbox}[breakable, size=fbox, boxrule=1pt, pad at break*=1mm,colback=cellbackground, colframe=cellborder]
\prompt{In}{incolor}{3}{\boxspacing}
\begin{Verbatim}[commandchars=\\\{\}]
\PY{n}{r\PYZus{}FPMI} \PY{o}{=} \PY{o}{\PYZhy{}}\PY{n}{r\PYZus{}1} \PY{o}{+} \PY{p}{(}\PY{n}{T\PYZus{}1}\PY{o}{*}\PY{n}{r\PYZus{}2}\PY{p}{)}\PY{o}{/}\PY{p}{(}\PY{l+m+mi}{1}\PY{o}{\PYZhy{}}\PY{n}{r\PYZus{}1}\PY{o}{*}\PY{n}{r\PYZus{}2}\PY{p}{)}
\PY{n}{T\PYZus{}PRM} \PY{o}{=} \PY{l+m+mf}{.03}
\PY{n}{R\PYZus{}PRM} \PY{o}{=} \PY{l+m+mi}{1}\PY{o}{\PYZhy{}}\PY{n}{T\PYZus{}PRM}
\PY{n}{t\PYZus{}PRM} \PY{o}{=} \PY{p}{(}\PY{n}{T\PYZus{}PRM}\PY{p}{)}\PY{o}{*}\PY{o}{*}\PY{l+m+mf}{.5}
\PY{n}{r\PYZus{}PRM} \PY{o}{=} \PY{p}{(}\PY{n}{R\PYZus{}PRM}\PY{p}{)}\PY{o}{*}\PY{o}{*}\PY{l+m+mf}{.5}
\PY{n}{G\PYZus{}PRC} \PY{o}{=} \PY{l+m+mi}{1}\PY{o}{/}\PY{p}{(}\PY{l+m+mi}{1}\PY{o}{\PYZhy{}}\PY{n}{r\PYZus{}PRM}\PY{o}{*}\PY{p}{(}\PY{n}{r\PYZus{}FPMI}\PY{p}{)}\PY{p}{)}
\end{Verbatim}
\end{tcolorbox}

    \begin{tcolorbox}[breakable, size=fbox, boxrule=1pt, pad at break*=1mm,colback=cellbackground, colframe=cellborder]
\prompt{In}{incolor}{4}{\boxspacing}
\begin{Verbatim}[commandchars=\\\{\}]
\PY{n}{L\PYZus{}rt} \PY{o}{=} \PY{l+m+mf}{75e\PYZhy{}6}
\PY{n}{Finn} \PY{o}{=} \PY{p}{(}\PY{n}{np}\PY{o}{.}\PY{n}{pi}\PY{o}{*}\PY{n}{np}\PY{o}{.}\PY{n}{sqrt}\PY{p}{(}\PY{n}{r\PYZus{}1}\PY{o}{*}\PY{n}{r\PYZus{}2}\PY{p}{)}\PY{p}{)}\PY{o}{/}\PY{p}{(}\PY{l+m+mi}{1}\PY{o}{\PYZhy{}}\PY{n}{r\PYZus{}1}\PY{o}{*}\PY{n}{r\PYZus{}2}\PY{p}{)}
\PY{n+nb}{print}\PY{p}{(}\PY{n}{Finn}\PY{p}{)}
\end{Verbatim}
\end{tcolorbox}

    \begin{Verbatim}[commandchars=\\\{\}]
444.0741558169753
    \end{Verbatim}

    \begin{tcolorbox}[breakable, size=fbox, boxrule=1pt, pad at break*=1mm,colback=cellbackground, colframe=cellborder]
\prompt{In}{incolor}{5}{\boxspacing}
\begin{Verbatim}[commandchars=\\\{\}]
\PY{n}{r\PYZus{}FPMI\PYZus{}approx} \PY{o}{=} \PY{p}{(}\PY{l+m+mi}{1} \PY{o}{\PYZhy{}} \PY{n}{Finn}\PY{o}{*}\PY{n}{L\PYZus{}rt}\PY{o}{/}\PY{n}{np}\PY{o}{.}\PY{n}{pi}\PY{p}{)}
\end{Verbatim}
\end{tcolorbox}

    \begin{tcolorbox}[breakable, size=fbox, boxrule=1pt, pad at break*=1mm,colback=cellbackground, colframe=cellborder]
\prompt{In}{incolor}{6}{\boxspacing}
\begin{Verbatim}[commandchars=\\\{\}]
\PY{n}{r\PYZus{}range} \PY{o}{=} \PY{n}{np}\PY{o}{.}\PY{n}{arange}\PY{p}{(}\PY{l+m+mf}{.9}\PY{p}{,}\PY{l+m+mi}{1}\PY{p}{,}\PY{l+m+mi}{1}\PY{o}{/}\PY{p}{(}\PY{l+m+mi}{2}\PY{o}{*}\PY{o}{*}\PY{l+m+mi}{16}\PY{p}{)}\PY{p}{)}
\end{Verbatim}
\end{tcolorbox}

    \begin{tcolorbox}[breakable, size=fbox, boxrule=1pt, pad at break*=1mm,colback=cellbackground, colframe=cellborder]
\prompt{In}{incolor}{7}{\boxspacing}
\begin{Verbatim}[commandchars=\\\{\}]
\PY{n}{G\PYZus{}PRC\PYZus{}} \PY{o}{=} \PY{n}{ifco}\PY{o}{.}\PY{n}{PRG}\PY{p}{(}\PY{n}{L\PYZus{}rt}\PY{p}{,} \PY{n}{Finn}\PY{p}{,} \PY{n}{r\PYZus{}range}\PY{p}{,} \PY{n+nb}{max}\PY{o}{=}\PY{l+m+mi}{0}\PY{p}{)} 
\end{Verbatim}
\end{tcolorbox}

    \begin{tcolorbox}[breakable, size=fbox, boxrule=1pt, pad at break*=1mm,colback=cellbackground, colframe=cellborder]
\prompt{In}{incolor}{8}{\boxspacing}
\begin{Verbatim}[commandchars=\\\{\}]
ifco.PRG\PY{o}{?}
\end{Verbatim}
\end{tcolorbox}

    
    \begin{Verbatim}[commandchars=\\\{\}]
\textcolor{ansi-red}{Signature:} ifco\textcolor{ansi-blue}{.}PRG\textcolor{ansi-blue}{(}L\_rt\textcolor{ansi-blue}{,} Finn\textcolor{ansi-blue}{,} r\_PRM\textcolor{ansi-blue}{,} max\textcolor{ansi-blue}{=}\textcolor{ansi-cyan}{0}\textcolor{ansi-blue}{)}
\textcolor{ansi-red}{Docstring:}
POWER RECYCLING GAIN (@ optimal reflectivity)
* Assuming a FPMI with symmetric arms *
L\_rt : Round trip loss
Finn : Cavity finesse
\textcolor{ansi-red}{File:}      \textasciitilde{}/Documents/git/SU/dissertation/code/ifo\_configs.py
\textcolor{ansi-red}{Type:}      function

    \end{Verbatim}

    
    \begin{tcolorbox}[breakable, size=fbox, boxrule=1pt, pad at break*=1mm,colback=cellbackground, colframe=cellborder]
\prompt{In}{incolor}{9}{\boxspacing}
\begin{Verbatim}[commandchars=\\\{\}]
\PY{n}{G\PYZus{}PRC\PYZus{}opt} \PY{o}{=}  \PY{n}{ifco}\PY{o}{.}\PY{n}{PRG}\PY{p}{(}\PY{n}{L\PYZus{}rt}\PY{p}{,} \PY{n}{Finn}\PY{p}{,} \PY{n}{r\PYZus{}FPMI}\PY{p}{,} \PY{n+nb}{max}\PY{o}{=}\PY{l+m+mi}{1}\PY{p}{)}
\end{Verbatim}
\end{tcolorbox}

    \begin{tcolorbox}[breakable, size=fbox, boxrule=1pt, pad at break*=1mm,colback=cellbackground, colframe=cellborder]
\prompt{In}{incolor}{10}{\boxspacing}
\begin{Verbatim}[commandchars=\\\{\}]
\PY{n}{plt}\PY{o}{.}\PY{n}{plot}\PY{p}{(}\PY{n}{r\PYZus{}range}\PY{p}{,} \PY{n}{G\PYZus{}PRC\PYZus{}}\PY{p}{,} \PY{n}{linewidth}\PY{o}{=}\PY{n}{line\PYZus{}width}\PY{p}{)}
\PY{n}{plt}\PY{o}{.}\PY{n}{axhline}\PY{p}{(}\PY{n}{G\PYZus{}PRC\PYZus{}opt}\PY{p}{,} \PY{n}{linestyle}\PY{o}{=}\PY{l+s+s1}{\PYZsq{}}\PY{l+s+s1}{\PYZhy{}\PYZhy{}}\PY{l+s+s1}{\PYZsq{}}\PY{p}{,}\PY{n}{linewidth}\PY{o}{=}\PY{n}{line\PYZus{}width}\PY{p}{,} \PY{n}{color}\PY{o}{=}\PY{l+s+s1}{\PYZsq{}}\PY{l+s+s1}{r}\PY{l+s+s1}{\PYZsq{}}\PY{p}{)}
\PY{n}{plt}\PY{o}{.}\PY{n}{xlim}\PY{p}{(}\PY{n}{r\PYZus{}range}\PY{p}{[}\PY{l+m+mi}{0}\PY{p}{]}\PY{p}{,} \PY{n}{r\PYZus{}range}\PY{p}{[}\PY{o}{\PYZhy{}}\PY{l+m+mi}{1}\PY{p}{]}\PY{p}{)}
\PY{n}{plt}\PY{o}{.}\PY{n}{xlabel}\PY{p}{(}\PY{l+s+s1}{\PYZsq{}}\PY{l+s+s1}{\PYZdl{}}\PY{l+s+s1}{\PYZbs{}}\PY{l+s+s1}{mathdefault}\PY{l+s+s1}{\PYZob{}}\PY{l+s+s1}{r\PYZus{}}\PY{l+s+si}{\PYZob{}PRM\PYZcb{}}\PY{l+s+s1}{\PYZcb{}\PYZdl{} [arb]}\PY{l+s+s1}{\PYZsq{}}\PY{p}{)}
\PY{n}{plt}\PY{o}{.}\PY{n}{ylabel}\PY{p}{(}\PY{l+s+s1}{\PYZsq{}}\PY{l+s+s1}{\PYZdl{}}\PY{l+s+s1}{\PYZbs{}}\PY{l+s+s1}{mathdefault}\PY{l+s+s1}{\PYZob{}}\PY{l+s+s1}{G\PYZus{}}\PY{l+s+si}{\PYZob{}PRC\PYZcb{}}\PY{l+s+s1}{\PYZcb{}\PYZdl{} [arb]}\PY{l+s+s1}{\PYZsq{}}\PY{p}{)}
\end{Verbatim}
\end{tcolorbox}

            \begin{tcolorbox}[breakable, size=fbox, boxrule=.5pt, pad at break*=1mm, opacityfill=0]
\prompt{Out}{outcolor}{10}{\boxspacing}
\begin{Verbatim}[commandchars=\\\{\}]
Text(0, 0.5, '$\textbackslash{}\textbackslash{}mathdefault\{G\_\{PRC\}\}$ [arb]')
\end{Verbatim}
\end{tcolorbox}
        
    \begin{center}
    \adjustimage{max size={0.9\linewidth}{0.9\paperheight}}{drfpmi_fr_files/drfpmi_fr_21_1.png}
    \end{center}
    { \hspace*{\fill} \\}
    
    \begin{tcolorbox}[breakable, size=fbox, boxrule=1pt, pad at break*=1mm,colback=cellbackground, colframe=cellborder]
\prompt{In}{incolor}{11}{\boxspacing}
\begin{Verbatim}[commandchars=\\\{\}]
\PY{n}{G\PYZus{}PRC\PYZus{}actual} \PY{o}{=} \PY{n}{ifco}\PY{o}{.}\PY{n}{PRG}\PY{p}{(}\PY{n}{L\PYZus{}rt}\PY{p}{,} \PY{n}{Finn}\PY{p}{,} \PY{n}{r\PYZus{}PRM}\PY{p}{,} \PY{n+nb}{max}\PY{o}{=}\PY{l+m+mi}{1}\PY{p}{)}
\end{Verbatim}
\end{tcolorbox}

    \begin{tcolorbox}[breakable, size=fbox, boxrule=1pt, pad at break*=1mm,colback=cellbackground, colframe=cellborder]
\prompt{In}{incolor}{12}{\boxspacing}
\begin{Verbatim}[commandchars=\\\{\}]
ifco.fpmi\PYZus{}freq\PYZus{}resp\PY{o}{?}
\end{Verbatim}
\end{tcolorbox}

    
    \begin{Verbatim}[commandchars=\\\{\}]
\textcolor{ansi-red}{Signature:}
ifco\textcolor{ansi-blue}{.}fpmi\_freq\_resp\textcolor{ansi-blue}{(}
    freq\textcolor{ansi-blue}{,}
    r\_1\textcolor{ansi-blue}{,}
    t\_1\textcolor{ansi-blue}{,}
    r\_2\textcolor{ansi-blue}{,}
    L\textcolor{ansi-blue}{,}
    phi\_0\textcolor{ansi-blue}{,}
    P\_in\textcolor{ansi-blue}{,}
    OMEGA\textcolor{ansi-blue}{,}
    low\_pass\textcolor{ansi-blue}{=}\textcolor{ansi-green}{False}\textcolor{ansi-blue}{,}
\textcolor{ansi-blue}{)}
\textcolor{ansi-red}{Docstring:}
FABRY PEROT MICHELSON FREQUENCY RESPONSE CALCULATOR
freq : standard (gravitational wave) frequency [Hz]
r\_1, t\_1, r\_2: Assuming arm symmetry where the ITM has r\_1, t\_1 coefficients and the ETM has a r\_2 reflectivity coefficient. Also assumes no loss. [arb]
OMEGA: OPTICAL angular frequency [rad Hz]
Length: Michelson ifo arm length [m]
phi\_0 : static differential arm length tuning phase [rad]
\textcolor{ansi-red}{File:}      \textasciitilde{}/Documents/git/SU/dissertation/code/ifo\_configs.py
\textcolor{ansi-red}{Type:}      function

    \end{Verbatim}

    
    \begin{tcolorbox}[breakable, size=fbox, boxrule=1pt, pad at break*=1mm,colback=cellbackground, colframe=cellborder]
\prompt{In}{incolor}{13}{\boxspacing}
\begin{Verbatim}[commandchars=\\\{\}]
\PY{n}{H\PYZus{}FPMI} \PY{o}{=} \PY{n}{ifco}\PY{o}{.}\PY{n}{fpmi\PYZus{}freq\PYZus{}resp}\PY{p}{(}\PY{n}{nu}\PY{p}{,} \PY{n}{r\PYZus{}1}\PY{p}{,} \PY{n}{t\PYZus{}1}\PY{p}{,} \PY{n}{r\PYZus{}2}\PY{p}{,} \PY{n}{L}\PY{p}{,} \PY{n}{PHI\PYZus{}0}\PY{p}{,} \PY{n}{P\PYZus{}IN}\PY{p}{,} \PY{n}{OMEG}\PY{p}{)}
\PY{n}{H\PYZus{}FPMI\PYZus{}LP} \PY{o}{=} \PY{n}{ifco}\PY{o}{.}\PY{n}{fpmi\PYZus{}freq\PYZus{}resp}\PY{p}{(}\PY{n}{nu}\PY{p}{,} \PY{n}{r\PYZus{}1}\PY{p}{,} \PY{n}{t\PYZus{}1}\PY{p}{,} \PY{n}{r\PYZus{}2}\PY{p}{,} \PY{n}{L}\PY{p}{,} \PY{n}{PHI\PYZus{}0}\PY{p}{,} \PY{n}{P\PYZus{}IN}\PY{p}{,} \PY{n}{OMEG}\PY{p}{,} \PY{n}{low\PYZus{}pass}\PY{o}{=}\PY{l+s+s1}{\PYZsq{}}\PY{l+s+s1}{True}\PY{l+s+s1}{\PYZsq{}}\PY{p}{)}
\end{Verbatim}
\end{tcolorbox}

    \begin{tcolorbox}[breakable, size=fbox, boxrule=1pt, pad at break*=1mm,colback=cellbackground, colframe=cellborder]
\prompt{In}{incolor}{14}{\boxspacing}
\begin{Verbatim}[commandchars=\\\{\}]
\PY{n}{H\PYZus{}PRFPMI} \PY{o}{=} \PY{p}{(}\PY{p}{(}\PY{n}{G\PYZus{}PRC\PYZus{}actual}\PY{p}{)}\PY{o}{*}\PY{o}{*}\PY{l+m+mf}{.5}\PY{p}{)}\PY{o}{*}\PY{n}{H\PYZus{}FPMI}
\end{Verbatim}
\end{tcolorbox}

    We estimate the FP's pole frequency
\[  1 - r_1 r_2 e^{-2i \omega_g L / c} = 0 \] therefore when:
\[ e^{-i \omega_g L / c} = \frac{1}{\sqrt{r_1 r_2}} \] we acquire the
pole frequency \(\omega_\mathrm{pole}\) as indicated in the low pass
\[ f_\mathrm{pole} = \frac{1}{4\pi \tau_{s}} =  \frac{c}{4 \pi L} \frac{1- r_1 r_2}{\sqrt{r_1 r_2}} = \frac{\nu_\mathrm{FSR}}{2 \pi} \frac{1- r_1 r_2}{\sqrt{r_1 r_2}} = \frac{\nu_\mathrm{FSR}}{\mathcal{F}} \]

    \begin{tcolorbox}[breakable, size=fbox, boxrule=1pt, pad at break*=1mm,colback=cellbackground, colframe=cellborder]
\prompt{In}{incolor}{15}{\boxspacing}
\begin{Verbatim}[commandchars=\\\{\}]
ifco.mich\PYZus{}freq\PYZus{}resp\PY{o}{?}
\end{Verbatim}
\end{tcolorbox}

    
    \begin{Verbatim}[commandchars=\\\{\}]
\textcolor{ansi-red}{Signature:} ifco\textcolor{ansi-blue}{.}mich\_freq\_resp\textcolor{ansi-blue}{(}freq\textcolor{ansi-blue}{,} Length\textcolor{ansi-blue}{,} phi\_0\textcolor{ansi-blue}{,} P\_in\textcolor{ansi-blue}{,} OMEGA\textcolor{ansi-blue}{)}
\textcolor{ansi-red}{Docstring:}
MICHELSON FREQEUNCY RESPONSE CALCULATOR
freq : standard (gravitational wave) frequency [Hz]
Length : Michelson ifo arm length [m]
phi\_0 : static differential arm length tuning phase [rad]
P\_in : input power [W] 
\textcolor{ansi-red}{File:}      \textasciitilde{}/Documents/git/SU/dissertation/code/ifo\_configs.py
\textcolor{ansi-red}{Type:}      function

    \end{Verbatim}

    
    Might as well compare it to our Michelson response:
\[ H_{\mathrm{MI}}(\omega_g) = \frac{2 L \Omega}{c}e^{-i L \omega / c} \frac{\mathrm{sin}(L \omega /c)}{L \omega /c} \]

    \begin{tcolorbox}[breakable, size=fbox, boxrule=1pt, pad at break*=1mm,colback=cellbackground, colframe=cellborder]
\prompt{In}{incolor}{18}{\boxspacing}
\begin{Verbatim}[commandchars=\\\{\}]
\PY{n}{H\PYZus{}MI} \PY{o}{=} \PY{n}{ifco}\PY{o}{.}\PY{n}{mich\PYZus{}freq\PYZus{}resp}\PY{p}{(}\PY{n}{nu}\PY{p}{,} \PY{n}{L}\PY{p}{,} \PY{n}{PHI\PYZus{}0}\PY{p}{,} \PY{n}{P\PYZus{}IN}\PY{p}{,} \PY{n}{OMEG}\PY{p}{)}
\end{Verbatim}
\end{tcolorbox}

    \begin{tcolorbox}[breakable, size=fbox, boxrule=1pt, pad at break*=1mm,colback=cellbackground, colframe=cellborder]
\prompt{In}{incolor}{19}{\boxspacing}
\begin{Verbatim}[commandchars=\\\{\}]
\PY{n}{plt}\PY{o}{.}\PY{n}{loglog}\PY{p}{(}\PY{n}{nu}\PY{p}{,} \PY{n}{ifco}\PY{o}{.}\PY{n}{bode\PYZus{}amp}\PY{p}{(}\PY{n}{H\PYZus{}MI}\PY{p}{)}\PY{p}{,} \PY{n}{label}\PY{o}{=} \PY{l+s+s1}{\PYZsq{}}\PY{l+s+s1}{MICH}\PY{l+s+s1}{\PYZsq{}}\PY{p}{,} \PY{n}{linewidth}\PY{o}{=} \PY{n}{line\PYZus{}width}\PY{p}{,} \PY{n}{alpha}\PY{o}{=}\PY{l+m+mf}{.3}\PY{p}{)}
\PY{n}{plt}\PY{o}{.}\PY{n}{loglog}\PY{p}{(}\PY{n}{nu}\PY{p}{,} \PY{n}{ifco}\PY{o}{.}\PY{n}{bode\PYZus{}amp}\PY{p}{(}\PY{n}{H\PYZus{}FPMI}\PY{p}{)}\PY{p}{,} \PY{n}{label}\PY{o}{=}\PY{l+s+s1}{\PYZsq{}}\PY{l+s+s1}{FPMI}\PY{l+s+s1}{\PYZsq{}}\PY{p}{,} \PY{n}{linewidth}\PY{o}{=}\PY{n}{line\PYZus{}width}\PY{p}{,} \PY{n}{alpha}\PY{o}{=}\PY{l+m+mf}{.3}\PY{p}{)}
\PY{n}{plt}\PY{o}{.}\PY{n}{loglog}\PY{p}{(}\PY{n}{nu}\PY{p}{,} \PY{n}{ifco}\PY{o}{.}\PY{n}{bode\PYZus{}amp}\PY{p}{(}\PY{n}{H\PYZus{}PRFPMI}\PY{p}{)}\PY{p}{,} \PY{n}{label}\PY{o}{=}\PY{l+s+s1}{\PYZsq{}}\PY{l+s+s1}{PRFPMI}\PY{l+s+s1}{\PYZsq{}}\PY{p}{,} \PY{n}{linewidth} \PY{o}{=} \PY{n}{line\PYZus{}width}\PY{p}{)}
\PY{c+c1}{\PYZsh{}plt.axvline (x=f\PYZus{}pole,ymin=1e\PYZhy{}11, color=\PYZsq{}red\PYZsq{}, linestyle=\PYZsq{}dotted\PYZsq{}, linewidth=3)}
\PY{n}{plt}\PY{o}{.}\PY{n}{xlim}\PY{p}{(}\PY{p}{[}\PY{l+m+mf}{1e0}\PY{p}{,} \PY{l+m+mf}{1e5}\PY{p}{]}\PY{p}{)}
\PY{n}{plt}\PY{o}{.}\PY{n}{ylim}\PY{p}{(}\PY{p}{[}\PY{l+m+mf}{1e9}\PY{p}{,}\PY{l+m+mf}{2e15}\PY{p}{]}\PY{p}{)}
\PY{n}{plt}\PY{o}{.}\PY{n}{xlabel}\PY{p}{(}\PY{l+s+s1}{\PYZsq{}}\PY{l+s+s1}{frequency [Hz]}\PY{l+s+s1}{\PYZsq{}}\PY{p}{)}
\PY{n}{plt}\PY{o}{.}\PY{n}{ylabel}\PY{p}{(}\PY{l+s+s1}{\PYZsq{}}\PY{l+s+s1}{H(f) [\PYZdl{}}\PY{l+s+s1}{\PYZbs{}}\PY{l+s+s1}{mathdefault}\PY{l+s+s1}{\PYZob{}}\PY{l+s+s1}{W/m\PYZcb{}\PYZdl{}]}\PY{l+s+s1}{\PYZsq{}}\PY{p}{)}
\PY{n}{lgd}\PY{o}{=}\PY{n}{plt}\PY{o}{.}\PY{n}{legend}\PY{p}{(}\PY{p}{)}
\PY{n}{plt}\PY{o}{.}\PY{n}{savefig}\PY{p}{(}\PY{l+s+s1}{\PYZsq{}}\PY{l+s+s1}{../figs/INTRO/prfpmi\PYZus{}fr.pdf}\PY{l+s+s1}{\PYZsq{}}\PY{p}{,} \PY{n}{dpi}\PY{o}{=}\PY{l+m+mi}{300}\PY{p}{,} \PY{n}{bbox\PYZus{}inches}\PY{o}{=}\PY{l+s+s1}{\PYZsq{}}\PY{l+s+s1}{tight}\PY{l+s+s1}{\PYZsq{}}\PY{p}{)}
\end{Verbatim}
\end{tcolorbox}

    \begin{center}
    \adjustimage{max size={0.9\linewidth}{0.9\paperheight}}{drfpmi_fr_files/drfpmi_fr_30_0.png}
    \end{center}
    { \hspace*{\fill} \\}
    
    \begin{tcolorbox}[breakable, size=fbox, boxrule=1pt, pad at break*=1mm,colback=cellbackground, colframe=cellborder]
\prompt{In}{incolor}{20}{\boxspacing}
\begin{Verbatim}[commandchars=\\\{\}]
\PY{n}{plt}\PY{o}{.}\PY{n}{semilogx}\PY{p}{(}\PY{n}{nu}\PY{p}{,}\PY{p}{(}\PY{l+m+mi}{180}\PY{o}{/}\PY{n}{np}\PY{o}{.}\PY{n}{pi}\PY{p}{)}\PY{o}{*}\PY{n}{np}\PY{o}{.}\PY{n}{arctan}\PY{p}{(}\PY{n}{np}\PY{o}{.}\PY{n}{imag}\PY{p}{(}\PY{n}{H\PYZus{}FPMI}\PY{p}{)}\PY{o}{/}\PY{n}{np}\PY{o}{.}\PY{n}{real}\PY{p}{(}\PY{n}{H\PYZus{}FPMI}\PY{p}{)}\PY{p}{)}\PY{p}{,}\PY{l+s+s1}{\PYZsq{}}\PY{l+s+s1}{\PYZhy{}\PYZhy{}}\PY{l+s+s1}{\PYZsq{}}\PY{p}{,} \PY{n}{linewidth}\PY{o}{=}\PY{n}{line\PYZus{}width}\PY{p}{)}
\PY{n}{plt}\PY{o}{.}\PY{n}{semilogx}\PY{p}{(}\PY{n}{nu}\PY{p}{,}\PY{p}{(}\PY{l+m+mi}{180}\PY{o}{/}\PY{n}{np}\PY{o}{.}\PY{n}{pi}\PY{p}{)}\PY{o}{*}\PY{n}{np}\PY{o}{.}\PY{n}{arctan}\PY{p}{(}\PY{n}{np}\PY{o}{.}\PY{n}{imag}\PY{p}{(}\PY{n}{H\PYZus{}MI}\PY{p}{)}\PY{o}{/}\PY{n}{np}\PY{o}{.}\PY{n}{real}\PY{p}{(}\PY{n}{H\PYZus{}MI}\PY{p}{)}\PY{p}{)}\PY{p}{,} \PY{l+s+s1}{\PYZsq{}}\PY{l+s+s1}{\PYZhy{}\PYZhy{}}\PY{l+s+s1}{\PYZsq{}}\PY{p}{,} \PY{n}{linewidth}\PY{o}{=}\PY{n}{line\PYZus{}width}\PY{p}{)}
\PY{n}{plt}\PY{o}{.}\PY{n}{semilogx}\PY{p}{(}\PY{n}{nu}\PY{p}{,}\PY{p}{(}\PY{l+m+mi}{180}\PY{o}{/}\PY{n}{np}\PY{o}{.}\PY{n}{pi}\PY{p}{)}\PY{o}{*}\PY{n}{np}\PY{o}{.}\PY{n}{arctan}\PY{p}{(}\PY{n}{np}\PY{o}{.}\PY{n}{imag}\PY{p}{(}\PY{n}{H\PYZus{}PRFPMI}\PY{p}{)}\PY{o}{/}\PY{n}{np}\PY{o}{.}\PY{n}{real}\PY{p}{(}\PY{n}{H\PYZus{}PRFPMI}\PY{p}{)}\PY{p}{)}\PY{p}{,}\PY{n}{linestyle}\PY{o}{=}\PY{l+s+s1}{\PYZsq{}}\PY{l+s+s1}{\PYZhy{}\PYZhy{}}\PY{l+s+s1}{\PYZsq{}}\PY{p}{,} \PY{n}{linewidth}\PY{o}{=}\PY{n}{line\PYZus{}width}\PY{p}{,}\PY{n}{dashes}\PY{o}{=}\PY{p}{(}\PY{l+m+mi}{3}\PY{p}{,}\PY{l+m+mi}{10}\PY{p}{)}\PY{p}{)}
\PY{n}{plt}\PY{o}{.}\PY{n}{xlim}\PY{p}{(}\PY{p}{[}\PY{l+m+mi}{1}\PY{p}{,}\PY{l+m+mi}{100000}\PY{p}{]}\PY{p}{)}
\PY{n}{plt}\PY{o}{.}\PY{n}{ylabel}\PY{p}{(}\PY{l+s+s1}{\PYZsq{}}\PY{l+s+s1}{phase [deg]}\PY{l+s+s1}{\PYZsq{}}\PY{p}{)}
\PY{n}{plt}\PY{o}{.}\PY{n}{xlabel}\PY{p}{(}\PY{l+s+s1}{\PYZsq{}}\PY{l+s+s1}{Frequency [Hz]}\PY{l+s+s1}{\PYZsq{}}\PY{p}{)}
\end{Verbatim}
\end{tcolorbox}

            \begin{tcolorbox}[breakable, size=fbox, boxrule=.5pt, pad at break*=1mm, opacityfill=0]
\prompt{Out}{outcolor}{20}{\boxspacing}
\begin{Verbatim}[commandchars=\\\{\}]
Text(0.5, 0, 'Frequency [Hz]')
\end{Verbatim}
\end{tcolorbox}
        
    \begin{center}
    \adjustimage{max size={0.9\linewidth}{0.9\paperheight}}{drfpmi_fr_files/drfpmi_fr_31_1.png}
    \end{center}
    { \hspace*{\fill} \\}
    
    \hypertarget{signal-recycling}{%
\section{SIGNAL RECYCLING}\label{signal-recycling}}

    Initially not used in early iterations of LIGO (intial LIGO and enhanced
LIGO) signal recycling imagines using a partially reflective mirror at
the anti-symmetric port. And at first glance it seems to not very much
make sense to have a mirror at detector output as you would potentially
attenuate gravitational wave signals by said mirror reflection
coefficient.

While true, it is important to analyze the multi-state configurations
offered by such a mirror with various microscopic length tuning
configurations. What do I mean by this? Well, it helps to start
imagining by analogy of couple cavity relationship as established in the
power recycling discussion. The relationship of the differential signal
output of the PRFPMI with respect to the newly placed mirror at the
anti-symmetric port is represented by the following:

\[ t_\mathrm{SRC} = \frac{t_\mathrm{ITM}t_\mathrm{SRM} e^{i  (k + \Omega/c) \mathscr{l}_\mathrm{SRC}}}{1- r_\mathrm{ITM}r_\mathrm{SRM} e^{2i  (k + \Omega/c) \mathscr{l}_\mathrm{SRC}}}\]

\[ r_\mathrm{SRC} = \frac{r_\mathrm{ITM} - r_\mathrm{SRM} e^{2i  (k + \Omega/c) l_\mathrm{SRC}}}{1- r_\mathrm{ITM}r_\mathrm{SRM} e^{2i  (k + \Omega/c) \mathscr{l}_\mathrm{SRC}}}\]

as \(k >> \Omega_\mathrm{gw}/c\) for $1 \textless{}
\Omega\_\mathrm{gw} \textless{} 5 \cdot 10^{3} $

Therefore with a pre-defined
\(T_\mathrm{ITM} + R_\mathrm{ITM} + L_\mathrm{ITM} = 1\) the coupled
cavity pole AND gain is a function of the SRM reflectivity and
microscopic length tuning:

\[ t_\mathrm{SRC} = \frac{t_\mathrm{ITM}t_\mathrm{SRM} e^{i k \mathscr{l}_\mathrm{SRC}}}{1- r_\mathrm{ITM}r_\mathrm{SRM} e^{2i k \mathscr{l}_\mathrm{SRC}}}\]

\[ r_\mathrm{SRC} = \frac{r_\mathrm{ITM} - r_\mathrm{SRM} e^{2i k l_\mathrm{SRC}}}{1- r_\mathrm{ITM}r_\mathrm{SRM} e^{2i k \mathscr{l}_\mathrm{SRC}}}\]

    We now observe the tuning extrema: - On resonance
\(2ik \mathscr{l}_\mathrm{SRC} = 2i\phi_\mathrm{SRC} = 0\):
\[ r_\mathrm{SRC \; , \; \phi_{SRC} = 0} = \frac{r_\mathrm{ITM} - r_\mathrm{SRM}}{1- r_\mathrm{ITM}r_\mathrm{SRM}}\]
- On resonance \[ 2ik \mathscr{l}_\mathrm{SRC} = 2i\phi_\mathrm{SRC}
= \frac{\pi}{2} \]:
\[ r_\mathrm{SRC \; , \; \phi_{SRC} = \pi} = \frac{r_\mathrm{ITM} + r_\mathrm{SRM}}{1+ r_\mathrm{ITM}r_\mathrm{SRM}}\]

    \[\mathrm{H}_\mathrm{DRFPMI} = \mathrm{G}_\mathrm{PR} \mathrm{P}_\mathrm{in} L \Omega \bigg[ \frac{ t_\mathrm{ITM}^2 r_\mathrm{ETM}}{(t_\mathrm{ITM}^2 + r_\mathrm{ITM}^2)r_\mathrm{ETM} - r_\mathrm{ITM}} \frac{t_\mathrm{SRM} t_\mathrm{ITM} e^{i\phi_\mathrm{SRC}}}{1-r_\mathrm{ITM} r_\mathrm{SRM} e^{i2\phi_\mathrm{SRC}}} \frac{e^{-i 2 \pi L f / c} \mathrm{sin}( 2 \pi f / c)}{ 2 \pi L f } \frac{\mathrm{sin}(\phi_0)}{1- [(r_\mathrm{ITM} - r_\mathrm{SRM} e^{i2\phi_\mathrm{SRC}})/(1-r_\mathrm{ITM} r_\mathrm{SRM} e^{i2\phi_\mathrm{SRC}})] r_\mathrm{ETM} e^{-i 4 \pi L f / c}} \bigg]\]

    \begin{tcolorbox}[breakable, size=fbox, boxrule=1pt, pad at break*=1mm,colback=cellbackground, colframe=cellborder]
\prompt{In}{incolor}{21}{\boxspacing}
\begin{Verbatim}[commandchars=\\\{\}]
ifco.drfpmi\PYZus{}freq\PYZus{}resp\PY{o}{?}
\end{Verbatim}
\end{tcolorbox}

    
    \begin{Verbatim}[commandchars=\\\{\}]
\textcolor{ansi-red}{Signature:}
ifco\textcolor{ansi-blue}{.}drfpmi\_freq\_resp\textcolor{ansi-blue}{(}
    freq\textcolor{ansi-blue}{,}
    G\_PRC\_opt\textcolor{ansi-blue}{,}
    r\_1\textcolor{ansi-blue}{,}
    t\_1\textcolor{ansi-blue}{,}
    r\_2\textcolor{ansi-blue}{,}
    r\_SRM\textcolor{ansi-blue}{,}
    t\_SRM\textcolor{ansi-blue}{,}
    phi\_SRC\textcolor{ansi-blue}{,}
    L\textcolor{ansi-blue}{,}
    phi\_0\textcolor{ansi-blue}{,}
    P\_in\textcolor{ansi-blue}{,}
    OMEGA\textcolor{ansi-blue}{,}
\textcolor{ansi-blue}{)}
\textcolor{ansi-red}{Docstring:}
 DUAL RECYCLED FABRY PEROT MICHELSON FREQUENCY RESPONSE CALCULATOR

 freq: standard (gravitational wave) frequency [Hz]
 G\_PRC\_opt: maximum power recycling gain (optimal) [arb]
 r\_1: ITM reflection coefficient [arb]
 t\_1: ITM transmission coefficient [arb]
 r\_2: ETM reflection coefficient [arb]
 r\_SRM: Signal recycling mirror reflection coefficient [arb]
 t\_SRM: Signal recycling mirror transmission coefficient [arb]
 L: Length of the Fabry-Perot arms [m]
 OMEGA: OPTICAL angular frequency [rad Hz]
 
\textcolor{ansi-red}{File:}      \textasciitilde{}/Documents/git/SU/dissertation/code/ifo\_configs.py
\textcolor{ansi-red}{Type:}      function

    \end{Verbatim}

    \begin{tcolorbox}[breakable, size=fbox, boxrule=1pt, pad at break*=1mm,colback=cellbackground, colframe=cellborder]
\prompt{In}{incolor}{22}{\boxspacing}
\begin{Verbatim}[commandchars=\\\{\}]
\PY{n}{l\PYZus{}SRC} \PY{o}{=} \PY{l+m+mi}{56}  \PY{c+c1}{\PYZsh{}[m]}

\PY{n}{T\PYZus{}SRM} \PY{o}{=} \PY{l+m+mf}{.30}
\PY{n}{R\PYZus{}SRM} \PY{o}{=} \PY{l+m+mi}{1}\PY{o}{\PYZhy{}}\PY{n}{T\PYZus{}SRM}
\PY{n}{t\PYZus{}SRM} \PY{o}{=} \PY{n}{T\PYZus{}SRM}\PY{o}{*}\PY{o}{*}\PY{l+m+mf}{.5}
\PY{n}{r\PYZus{}SRM} \PY{o}{=} \PY{n}{R\PYZus{}SRM}\PY{o}{*}\PY{o}{*}\PY{l+m+mf}{.5}

\PY{n}{phi\PYZus{}SRC} \PY{o}{=} \PY{n}{np}\PY{o}{.}\PY{n}{pi}
\end{Verbatim}
\end{tcolorbox}

    \begin{tcolorbox}[breakable, size=fbox, boxrule=1pt, pad at break*=1mm,colback=cellbackground, colframe=cellborder]
\prompt{In}{incolor}{23}{\boxspacing}
\begin{Verbatim}[commandchars=\\\{\}]
\PY{n}{H\PYZus{}DRFPMI} \PY{o}{=} \PY{n}{ifco}\PY{o}{.}\PY{n}{drfpmi\PYZus{}freq\PYZus{}resp}\PY{p}{(}\PY{n}{nu}\PY{p}{,} \PY{n}{G\PYZus{}PRC\PYZus{}opt}\PY{p}{,} \PY{n}{r\PYZus{}1}\PY{p}{,} \PY{n}{t\PYZus{}1}\PY{p}{,} \PY{n}{r\PYZus{}2}\PY{p}{,} \PY{n}{r\PYZus{}SRM}\PY{p}{,} \PY{n}{t\PYZus{}SRM}\PY{p}{,} \PY{n}{phi\PYZus{}SRC}\PY{p}{,} \PY{n}{L}\PY{p}{,} \PY{n}{PHI\PYZus{}0}\PY{p}{,} \PY{n}{P\PYZus{}IN}\PY{p}{,} \PY{n}{OMEG}\PY{p}{)}
\end{Verbatim}
\end{tcolorbox}

    \begin{tcolorbox}[breakable, size=fbox, boxrule=1pt, pad at break*=1mm,colback=cellbackground, colframe=cellborder]
\prompt{In}{incolor}{24}{\boxspacing}
\begin{Verbatim}[commandchars=\\\{\}]
\PY{n}{bode\PYZus{}test}\PY{o}{=}\PY{k+kc}{False}
\PY{k}{if} \PY{n}{bode\PYZus{}test}\PY{p}{:}
    \PY{n}{fig}\PY{p}{,} \PY{n}{ax1} \PY{o}{=} \PY{n}{plt}\PY{o}{.}\PY{n}{subplots}\PY{p}{(}\PY{p}{)}
    \PY{n}{ax1}\PY{o}{.}\PY{n}{set\PYZus{}xlabel}\PY{p}{(}\PY{l+s+s1}{\PYZsq{}}\PY{l+s+s1}{frequency [Hz]}\PY{l+s+s1}{\PYZsq{}}\PY{p}{)}
    \PY{n}{ax1}\PY{o}{.}\PY{n}{set\PYZus{}ylabel}\PY{p}{(}\PY{l+s+s1}{\PYZsq{}}\PY{l+s+s1}{H\PYZdl{}\PYZus{}}\PY{l+s+s1}{\PYZbs{}}\PY{l+s+s1}{mathdefault}\PY{l+s+si}{\PYZob{}FPMI\PYZcb{}}\PY{l+s+s1}{\PYZdl{}  [\PYZdl{}}\PY{l+s+s1}{\PYZbs{}}\PY{l+s+s1}{mathdefault}\PY{l+s+s1}{\PYZob{}}\PY{l+s+s1}{W/m\PYZcb{}\PYZdl{}]  }\PY{l+s+s1}{\PYZsq{}}\PY{p}{,} \PY{n}{color}\PY{o}{=}\PY{l+s+s1}{\PYZsq{}}\PY{l+s+s1}{C0}\PY{l+s+s1}{\PYZsq{}}\PY{p}{)}
    \PY{c+c1}{\PYZsh{}ax1.plot(w/(FSR), F\PYZus{}w\PYZus{}cc\PYZus{}modsq*100)}
    \PY{n}{ax1}\PY{o}{.}\PY{n}{loglog}\PY{p}{(}\PY{n}{nu}\PY{p}{,} \PY{n}{ifco}\PY{o}{.}\PY{n}{bode\PYZus{}amp}\PY{p}{(}\PY{n}{H\PYZus{}FPMI}\PY{p}{)}\PY{p}{,} \PY{n}{label}\PY{o}{=}\PY{l+s+s1}{\PYZsq{}}\PY{l+s+s1}{FPMI}\PY{l+s+s1}{\PYZsq{}}\PY{p}{,} \PY{n}{linewidth}\PY{o}{=}\PY{n}{line\PYZus{}width}\PY{p}{,} \PY{n}{linestyle}\PY{o}{=}\PY{l+s+s1}{\PYZsq{}}\PY{l+s+s1}{:}\PY{l+s+s1}{\PYZsq{}}\PY{p}{,}\PY{n}{color}\PY{o}{=}\PY{l+s+s1}{\PYZsq{}}\PY{l+s+s1}{C0}\PY{l+s+s1}{\PYZsq{}}\PY{p}{)}
    \PY{n}{ax1}\PY{o}{.}\PY{n}{loglog}\PY{p}{(}\PY{n}{nu}\PY{p}{,} \PY{n}{ifco}\PY{o}{.}\PY{n}{bode\PYZus{}amp}\PY{p}{(}\PY{n}{H\PYZus{}PRFPMI}\PY{p}{)}\PY{p}{,} \PY{n}{label}\PY{o}{=}\PY{l+s+s1}{\PYZsq{}}\PY{l+s+s1}{PRFPMI}\PY{l+s+s1}{\PYZsq{}}\PY{p}{,}  \PY{n}{linewidth}\PY{o}{=}\PY{n}{line\PYZus{}width}\PY{p}{,} \PY{n}{color}\PY{o}{=}\PY{l+s+s1}{\PYZsq{}}\PY{l+s+s1}{C0}\PY{l+s+s1}{\PYZsq{}}\PY{p}{)}
    \PY{n}{ax1}\PY{o}{.}\PY{n}{loglog}\PY{p}{(}\PY{n}{nu}\PY{p}{,} \PY{n}{ifco}\PY{o}{.}\PY{n}{bode\PYZus{}amp}\PY{p}{(}\PY{n}{H\PYZus{}DRFPMI}\PY{p}{)}\PY{p}{,} \PY{n}{label}\PY{o}{=}\PY{l+s+s1}{\PYZsq{}}\PY{l+s+s1}{DRFPMI}\PY{l+s+s1}{\PYZsq{}}\PY{p}{,}  \PY{n}{linewidth}\PY{o}{=}\PY{n}{line\PYZus{}width}\PY{p}{,} \PY{n}{color}\PY{o}{=}\PY{l+s+s1}{\PYZsq{}}\PY{l+s+s1}{C1}\PY{l+s+s1}{\PYZsq{}}\PY{p}{)}
    \PY{n}{ax1}\PY{o}{.}\PY{n}{legend}\PY{p}{(}\PY{p}{)}
    \PY{c+c1}{\PYZsh{}ax1.loglog(w,H\PYZus{}MI\PYZus{}modsq, label= \PYZsq{}MICH\PYZsq{}, linewidth= 5)}
    \PY{c+c1}{\PYZsh{}ax1.loglog(w,H\PYZus{}FPMI\PYZus{}LP\PYZus{}modsq*H\PYZus{}FPMI\PYZus{}modsq[0], label=\PYZsq{}FPMI LP\PYZsq{}, linewidth = 20.0, alpha=0.25,color=\PYZsq{}C2\PYZsq{})}
    \PY{c+c1}{\PYZsh{}ax1.axvline (x=f\PYZus{}pole,ymin=1e\PYZhy{}13, color=\PYZsq{}red\PYZsq{}, linestyle=\PYZsq{}dotted\PYZsq{}, linewidth=3)}
    \PY{n}{ax2} \PY{o}{=} \PY{n}{ax1}\PY{o}{.}\PY{n}{twinx}\PY{p}{(}\PY{p}{)}
    \PY{n}{ax2}\PY{o}{.}\PY{n}{semilogx}\PY{p}{(}\PY{n}{nu}\PY{p}{,} \PY{n}{ifco}\PY{o}{.}\PY{n}{bode\PYZus{}ph}\PY{p}{(}\PY{n}{H\PYZus{}FPMI}\PY{p}{)}\PY{p}{,}\PY{l+s+s1}{\PYZsq{}}\PY{l+s+s1}{\PYZhy{}\PYZhy{}}\PY{l+s+s1}{\PYZsq{}}\PY{p}{,} \PY{n}{linewidth}\PY{o}{=}\PY{l+m+mf}{7.5}\PY{p}{,} \PY{n}{color}\PY{o}{=}\PY{l+s+s1}{\PYZsq{}}\PY{l+s+s1}{C0}\PY{l+s+s1}{\PYZsq{}}\PY{p}{,} \PY{n}{alpha}\PY{o}{=}\PY{l+m+mf}{.3}\PY{p}{)}
    \PY{n}{ax2}\PY{o}{.}\PY{n}{semilogx}\PY{p}{(}\PY{n}{nu}\PY{p}{,} \PY{n}{ifco}\PY{o}{.}\PY{n}{bode\PYZus{}ph}\PY{p}{(}\PY{n}{H\PYZus{}DRFPMI}\PY{p}{)}\PY{p}{,}\PY{l+s+s1}{\PYZsq{}}\PY{l+s+s1}{\PYZhy{}\PYZhy{}}\PY{l+s+s1}{\PYZsq{}}\PY{p}{,} \PY{n}{linewidth}\PY{o}{=}\PY{l+m+mf}{7.5}\PY{p}{,} \PY{n}{color}\PY{o}{=}\PY{l+s+s1}{\PYZsq{}}\PY{l+s+s1}{C1}\PY{l+s+s1}{\PYZsq{}}\PY{p}{,} \PY{n}{alpha}\PY{o}{=}\PY{l+m+mf}{.3}\PY{p}{)}
    \PY{n}{ax2}\PY{o}{.}\PY{n}{grid}\PY{p}{(}\PY{n}{b}\PY{o}{=}\PY{k+kc}{False}\PY{p}{,} \PY{n}{which}\PY{o}{=}\PY{l+s+s1}{\PYZsq{}}\PY{l+s+s1}{both}\PY{l+s+s1}{\PYZsq{}}\PY{p}{,} \PY{n}{axis}\PY{o}{=}\PY{l+s+s1}{\PYZsq{}}\PY{l+s+s1}{y}\PY{l+s+s1}{\PYZsq{}}\PY{p}{)}
    \PY{c+c1}{\PYZsh{}ax2.semilogx(w,(180/np.pi)*np.arctan(np.imag(H\PYZus{}MI)/np.real(H\PYZus{}MI)), \PYZsq{}\PYZhy{}\PYZhy{}\PYZsq{})}
    \PY{c+c1}{\PYZsh{}ax2.semilogx(w,(180/np.pi)*np.arctan(np.imag(H\PYZus{}FPMI\PYZus{}LP)/np.real(H\PYZus{}FPMI\PYZus{}LP)),linestyle=\PYZsq{}\PYZhy{}\PYZhy{}\PYZsq{}, linewidth=20.0,dashes=(4,10),alpha=.25, color=\PYZsq{}C2\PYZsq{})}
    \PY{n}{plt}\PY{o}{.}\PY{n}{xlim}\PY{p}{(}\PY{p}{[}\PY{l+m+mi}{1}\PY{p}{,}\PY{l+m+mf}{1e5}\PY{p}{]}\PY{p}{)}
    \PY{n}{plt}\PY{o}{.}\PY{n}{ylabel}\PY{p}{(}\PY{l+s+s1}{\PYZsq{}}\PY{l+s+s1}{phase [deg]}\PY{l+s+s1}{\PYZsq{}}\PY{p}{,} \PY{n}{color}\PY{o}{=}\PY{l+s+s1}{\PYZsq{}}\PY{l+s+s1}{C1}\PY{l+s+s1}{\PYZsq{}}\PY{p}{,} \PY{n}{alpha}\PY{o}{=}\PY{l+m+mf}{.5}\PY{p}{)}
\end{Verbatim}
\end{tcolorbox}

    \begin{tcolorbox}[breakable, size=fbox, boxrule=1pt, pad at break*=1mm,colback=cellbackground, colframe=cellborder]
\prompt{In}{incolor}{25}{\boxspacing}
\begin{Verbatim}[commandchars=\\\{\}]
\PY{n}{plt}\PY{o}{.}\PY{n}{loglog}\PY{p}{(}\PY{n}{nu}\PY{p}{,} \PY{n}{ifco}\PY{o}{.}\PY{n}{bode\PYZus{}amp}\PY{p}{(}\PY{n}{H\PYZus{}MI}\PY{p}{)}\PY{p}{,} \PY{n}{label}\PY{o}{=} \PY{l+s+s1}{\PYZsq{}}\PY{l+s+s1}{MICH}\PY{l+s+s1}{\PYZsq{}}\PY{p}{,} \PY{n}{linewidth}\PY{o}{=} \PY{n}{line\PYZus{}width}\PY{p}{,} \PY{n}{alpha}\PY{o}{=}\PY{l+m+mf}{.4}\PY{p}{)}
\PY{n}{plt}\PY{o}{.}\PY{n}{loglog}\PY{p}{(}\PY{n}{nu}\PY{p}{,} \PY{n}{ifco}\PY{o}{.}\PY{n}{bode\PYZus{}amp}\PY{p}{(}\PY{n}{H\PYZus{}FPMI}\PY{p}{)}\PY{p}{,} \PY{n}{label}\PY{o}{=}\PY{l+s+s1}{\PYZsq{}}\PY{l+s+s1}{FPMI}\PY{l+s+s1}{\PYZsq{}}\PY{p}{,} \PY{n}{linewidth}\PY{o}{=}\PY{n}{line\PYZus{}width}\PY{p}{,} \PY{n}{alpha}\PY{o}{=}\PY{l+m+mf}{.4}\PY{p}{)}
\PY{n}{plt}\PY{o}{.}\PY{n}{loglog}\PY{p}{(}\PY{n}{nu}\PY{p}{,} \PY{n}{ifco}\PY{o}{.}\PY{n}{bode\PYZus{}amp}\PY{p}{(}\PY{n}{H\PYZus{}PRFPMI}\PY{p}{)}\PY{p}{,} \PY{n}{label}\PY{o}{=}\PY{l+s+s1}{\PYZsq{}}\PY{l+s+s1}{PRFPMI}\PY{l+s+s1}{\PYZsq{}}\PY{p}{,} \PY{n}{linewidth} \PY{o}{=} \PY{n}{line\PYZus{}width}\PY{p}{,} \PY{n}{alpha}\PY{o}{=}\PY{l+m+mf}{.4}\PY{p}{)}
\PY{n}{plt}\PY{o}{.}\PY{n}{loglog}\PY{p}{(}\PY{n}{nu}\PY{p}{,} \PY{n}{ifco}\PY{o}{.}\PY{n}{bode\PYZus{}amp}\PY{p}{(}\PY{n}{H\PYZus{}DRFPMI}\PY{p}{)}\PY{p}{,} \PY{n}{label}\PY{o}{=}\PY{l+s+s1}{\PYZsq{}}\PY{l+s+s1}{DRFPMI}\PY{l+s+s1}{\PYZsq{}}\PY{p}{,} \PY{n}{linewidth} \PY{o}{=} \PY{n}{line\PYZus{}width}\PY{p}{)}
\PY{c+c1}{\PYZsh{}plt.axvline (x=f\PYZus{}pole,ymin=1e\PYZhy{}11, color=\PYZsq{}red\PYZsq{}, linestyle=\PYZsq{}dotted\PYZsq{}, linewidth=3)}
\PY{n}{plt}\PY{o}{.}\PY{n}{xlim}\PY{p}{(}\PY{p}{[}\PY{l+m+mf}{1e0}\PY{p}{,} \PY{l+m+mf}{1e5}\PY{p}{]}\PY{p}{)}
\PY{n}{plt}\PY{o}{.}\PY{n}{ylim}\PY{p}{(}\PY{p}{[}\PY{l+m+mf}{1e9}\PY{p}{,} \PY{l+m+mf}{2e15}\PY{p}{]}\PY{p}{)}
\PY{n}{plt}\PY{o}{.}\PY{n}{xlabel}\PY{p}{(}\PY{l+s+s1}{\PYZsq{}}\PY{l+s+s1}{frequency [Hz]}\PY{l+s+s1}{\PYZsq{}}\PY{p}{)}
\PY{n}{plt}\PY{o}{.}\PY{n}{ylabel}\PY{p}{(}\PY{l+s+s1}{\PYZsq{}}\PY{l+s+s1}{H(f) [\PYZdl{}}\PY{l+s+s1}{\PYZbs{}}\PY{l+s+s1}{mathdefault}\PY{l+s+s1}{\PYZob{}}\PY{l+s+s1}{W/m\PYZcb{}\PYZdl{}]}\PY{l+s+s1}{\PYZsq{}}\PY{p}{)}
\PY{n}{lgd}\PY{o}{=}\PY{n}{plt}\PY{o}{.}\PY{n}{legend}\PY{p}{(}\PY{p}{)}
\PY{n}{plt}\PY{o}{.}\PY{n}{savefig}\PY{p}{(}\PY{l+s+s1}{\PYZsq{}}\PY{l+s+s1}{../figs/INTRO/drfpmi\PYZus{}fr.pdf}\PY{l+s+s1}{\PYZsq{}}\PY{p}{,} \PY{n}{dpi}\PY{o}{=}\PY{l+m+mi}{300}\PY{p}{,} \PY{n}{bbox\PYZus{}inches}\PY{o}{=}\PY{l+s+s1}{\PYZsq{}}\PY{l+s+s1}{tight}\PY{l+s+s1}{\PYZsq{}}\PY{p}{)}
\end{Verbatim}
\end{tcolorbox}

    \begin{center}
    \adjustimage{max size={0.9\linewidth}{0.9\paperheight}}{drfpmi_fr_files/drfpmi_fr_40_0.png}
    \end{center}
    { \hspace*{\fill} \\}
    
    \begin{tcolorbox}[breakable, size=fbox, boxrule=1pt, pad at break*=1mm,colback=cellbackground, colframe=cellborder]
\prompt{In}{incolor}{26}{\boxspacing}
\begin{Verbatim}[commandchars=\\\{\}]
\PY{n}{plt}\PY{o}{.}\PY{n}{semilogx}\PY{p}{(}\PY{n}{nu}\PY{p}{,}\PY{n}{ifco}\PY{o}{.}\PY{n}{bode\PYZus{}ph}\PY{p}{(}\PY{n}{H\PYZus{}MI}\PY{p}{)}\PY{p}{,} \PY{l+s+s1}{\PYZsq{}}\PY{l+s+s1}{\PYZhy{}\PYZhy{}}\PY{l+s+s1}{\PYZsq{}}\PY{p}{,} \PY{n}{linewidth}\PY{o}{=}\PY{n}{line\PYZus{}width}\PY{p}{,} \PY{n}{alpha}\PY{o}{=}\PY{l+m+mf}{.4}\PY{p}{,} \PY{n}{label}\PY{o}{=}\PY{l+s+s1}{\PYZsq{}}\PY{l+s+s1}{MICH}\PY{l+s+s1}{\PYZsq{}}\PY{p}{)}
\PY{n}{plt}\PY{o}{.}\PY{n}{semilogx}\PY{p}{(}\PY{n}{nu}\PY{p}{,}\PY{n}{ifco}\PY{o}{.}\PY{n}{bode\PYZus{}ph}\PY{p}{(}\PY{n}{H\PYZus{}FPMI}\PY{p}{)}\PY{p}{,}\PY{l+s+s1}{\PYZsq{}}\PY{l+s+s1}{\PYZhy{}\PYZhy{}}\PY{l+s+s1}{\PYZsq{}}\PY{p}{,} \PY{n}{linewidth}\PY{o}{=}\PY{n}{line\PYZus{}width}\PY{p}{,} \PY{n}{alpha}\PY{o}{=}\PY{l+m+mf}{.4}\PY{p}{,} \PY{n}{label}\PY{o}{=}\PY{l+s+s1}{\PYZsq{}}\PY{l+s+s1}{FPMI}\PY{l+s+s1}{\PYZsq{}}\PY{p}{)}
\PY{n}{plt}\PY{o}{.}\PY{n}{semilogx}\PY{p}{(}\PY{n}{nu}\PY{p}{,}\PY{n}{ifco}\PY{o}{.}\PY{n}{bode\PYZus{}ph}\PY{p}{(}\PY{n}{H\PYZus{}PRFPMI}\PY{p}{)}\PY{p}{,}\PY{n}{linestyle}\PY{o}{=}\PY{l+s+s1}{\PYZsq{}}\PY{l+s+s1}{\PYZhy{}\PYZhy{}}\PY{l+s+s1}{\PYZsq{}}\PY{p}{,} \PY{n}{linewidth}\PY{o}{=}\PY{n}{line\PYZus{}width}\PY{p}{,}\PY{n}{dashes}\PY{o}{=}\PY{p}{(}\PY{l+m+mi}{3}\PY{p}{,}\PY{l+m+mi}{10}\PY{p}{)}\PY{p}{,} \PY{n}{alpha}\PY{o}{=}\PY{l+m+mf}{.4}\PY{p}{,} \PY{n}{label}\PY{o}{=}\PY{l+s+s1}{\PYZsq{}}\PY{l+s+s1}{PRFPMI}\PY{l+s+s1}{\PYZsq{}}\PY{p}{)}
\PY{n}{plt}\PY{o}{.}\PY{n}{semilogx}\PY{p}{(}\PY{n}{nu}\PY{p}{,}\PY{n}{ifco}\PY{o}{.}\PY{n}{bode\PYZus{}ph}\PY{p}{(}\PY{n}{H\PYZus{}DRFPMI}\PY{p}{)}\PY{p}{,}\PY{l+s+s1}{\PYZsq{}}\PY{l+s+s1}{\PYZhy{}\PYZhy{}}\PY{l+s+s1}{\PYZsq{}}\PY{p}{,} \PY{n}{linewidth}\PY{o}{=}\PY{n}{line\PYZus{}width}\PY{p}{,} \PY{n}{label}\PY{o}{=}\PY{l+s+s1}{\PYZsq{}}\PY{l+s+s1}{DRFPMI}\PY{l+s+s1}{\PYZsq{}}\PY{p}{)}
\PY{n}{plt}\PY{o}{.}\PY{n}{xlim}\PY{p}{(}\PY{p}{[}\PY{l+m+mi}{1}\PY{p}{,}\PY{l+m+mi}{100000}\PY{p}{]}\PY{p}{)}
\PY{n}{plt}\PY{o}{.}\PY{n}{ylim}\PY{p}{(}\PY{p}{[}\PY{o}{\PYZhy{}}\PY{l+m+mi}{91}\PY{p}{,}\PY{l+m+mi}{91}\PY{p}{]}\PY{p}{)}
\PY{n}{plt}\PY{o}{.}\PY{n}{ylabel}\PY{p}{(}\PY{l+s+s1}{\PYZsq{}}\PY{l+s+s1}{phase [deg]}\PY{l+s+s1}{\PYZsq{}}\PY{p}{)}
\PY{n}{plt}\PY{o}{.}\PY{n}{xlabel}\PY{p}{(}\PY{l+s+s1}{\PYZsq{}}\PY{l+s+s1}{Frequency [Hz]}\PY{l+s+s1}{\PYZsq{}}\PY{p}{)}
\PY{n}{plt}\PY{o}{.}\PY{n}{legend}\PY{p}{(}\PY{p}{)}
\end{Verbatim}
\end{tcolorbox}

            \begin{tcolorbox}[breakable, size=fbox, boxrule=.5pt, pad at break*=1mm, opacityfill=0]
\prompt{Out}{outcolor}{26}{\boxspacing}
\begin{Verbatim}[commandchars=\\\{\}]
<matplotlib.legend.Legend at 0x17f3bec70>
\end{Verbatim}
\end{tcolorbox}
        
    \begin{center}
    \adjustimage{max size={0.9\linewidth}{0.9\paperheight}}{drfpmi_fr_files/drfpmi_fr_41_1.png}
    \end{center}
    { \hspace*{\fill} \\}
    
    \begin{tcolorbox}[breakable, size=fbox, boxrule=1pt, pad at break*=1mm,colback=cellbackground, colframe=cellborder]
\prompt{In}{incolor}{27}{\boxspacing}
\begin{Verbatim}[commandchars=\\\{\}]
ifco.N\PYZus{}shot\PY{o}{?}
\end{Verbatim}
\end{tcolorbox}

    
    \begin{Verbatim}[commandchars=\\\{\}]
\textcolor{ansi-red}{Signature:} ifco\textcolor{ansi-blue}{.}N\_shot\textcolor{ansi-blue}{(}OMEGA\textcolor{ansi-blue}{,} P\_in\textcolor{ansi-blue}{)}
\textcolor{ansi-red}{Docstring:}
Interferometer shot noise calculator
OMEG: OPTICAL angular frequency [rad Hz]
Length : ifo arm length [m]
phi\_0 : static differential arm length tuning phase [rad]
P\_in : Input power [W]
\textcolor{ansi-red}{File:}      \textasciitilde{}/Documents/git/SU/dissertation/code/ifo\_configs.py
\textcolor{ansi-red}{Type:}      function

    \end{Verbatim}

    
    \begin{tcolorbox}[breakable, size=fbox, boxrule=1pt, pad at break*=1mm,colback=cellbackground, colframe=cellborder]
\prompt{In}{incolor}{28}{\boxspacing}
\begin{Verbatim}[commandchars=\\\{\}]
\PY{n}{Sn} \PY{o}{=} \PY{n}{ifco}\PY{o}{.}\PY{n}{N\PYZus{}shot}\PY{p}{(}\PY{n}{OMEG}\PY{p}{,} \PY{n}{P\PYZus{}IN}\PY{p}{)}
\end{Verbatim}
\end{tcolorbox}

    \begin{tcolorbox}[breakable, size=fbox, boxrule=1pt, pad at break*=1mm,colback=cellbackground, colframe=cellborder]
\prompt{In}{incolor}{29}{\boxspacing}
\begin{Verbatim}[commandchars=\\\{\}]
\PY{n}{plt}\PY{o}{.}\PY{n}{loglog}\PY{p}{(}\PY{n}{nu}\PY{p}{,} \PY{n}{Sn}\PY{o}{/}\PY{n}{ifco}\PY{o}{.}\PY{n}{bode\PYZus{}amp}\PY{p}{(}\PY{n}{H\PYZus{}MI}\PY{p}{)}\PY{p}{,} \PY{n}{label}\PY{o}{=} \PY{l+s+s1}{\PYZsq{}}\PY{l+s+s1}{MICH}\PY{l+s+s1}{\PYZsq{}}\PY{p}{,} \PY{n}{linewidth}\PY{o}{=}\PY{n}{line\PYZus{}width}\PY{p}{)}
\PY{n}{plt}\PY{o}{.}\PY{n}{loglog}\PY{p}{(}\PY{n}{nu}\PY{p}{,} \PY{n}{Sn}\PY{o}{/}\PY{n}{ifco}\PY{o}{.}\PY{n}{bode\PYZus{}amp}\PY{p}{(}\PY{n}{H\PYZus{}FPMI}\PY{p}{)}\PY{p}{,} \PY{n}{label}\PY{o}{=}\PY{l+s+s1}{\PYZsq{}}\PY{l+s+s1}{FPMI}\PY{l+s+s1}{\PYZsq{}}\PY{p}{,} \PY{n}{linewidth}\PY{o}{=}\PY{n}{line\PYZus{}width}\PY{p}{)}
\PY{n}{plt}\PY{o}{.}\PY{n}{loglog}\PY{p}{(}\PY{n}{nu}\PY{p}{,} \PY{n}{Sn}\PY{o}{/}\PY{n}{ifco}\PY{o}{.}\PY{n}{bode\PYZus{}amp}\PY{p}{(}\PY{n}{H\PYZus{}PRFPMI}\PY{p}{)}\PY{p}{,} \PY{n}{label}\PY{o}{=}\PY{l+s+s1}{\PYZsq{}}\PY{l+s+s1}{PRFPMI}\PY{l+s+s1}{\PYZsq{}}\PY{p}{,} \PY{n}{linewidth}\PY{o}{=}\PY{n}{line\PYZus{}width}\PY{p}{)}
\PY{n}{plt}\PY{o}{.}\PY{n}{loglog}\PY{p}{(}\PY{n}{nu}\PY{p}{,} \PY{n}{Sn}\PY{o}{/}\PY{n}{ifco}\PY{o}{.}\PY{n}{bode\PYZus{}amp}\PY{p}{(}\PY{n}{H\PYZus{}DRFPMI}\PY{p}{)}\PY{p}{,} \PY{n}{label}\PY{o}{=}\PY{l+s+s1}{\PYZsq{}}\PY{l+s+s1}{DRFPMI}\PY{l+s+s1}{\PYZsq{}}\PY{p}{,} \PY{n}{linewidth}\PY{o}{=}\PY{n}{line\PYZus{}width}\PY{p}{)}
\PY{c+c1}{\PYZsh{}plt.axvline (x=f\PYZus{}pole,ymin=1e\PYZhy{}11, color=\PYZsq{}red\PYZsq{}, linestyle=\PYZsq{}dotted\PYZsq{}, linewidth=3)}
\PY{n}{plt}\PY{o}{.}\PY{n}{ylim}\PY{p}{(}\PY{p}{[}\PY{l+m+mf}{1e\PYZhy{}24}\PY{p}{,}\PY{l+m+mf}{2e\PYZhy{}19}\PY{p}{]}\PY{p}{)}
\PY{n}{plt}\PY{o}{.}\PY{n}{xlim}\PY{p}{(}\PY{p}{[}\PY{l+m+mf}{1e0}\PY{p}{,} \PY{l+m+mf}{1e5}\PY{p}{]}\PY{p}{)}
\PY{n}{plt}\PY{o}{.}\PY{n}{xlabel}\PY{p}{(}\PY{l+s+s1}{\PYZsq{}}\PY{l+s+s1}{frequency [Hz]}\PY{l+s+s1}{\PYZsq{}}\PY{p}{)}
\PY{n}{plt}\PY{o}{.}\PY{n}{ylabel}\PY{p}{(}\PY{l+s+s1}{\PYZsq{}}\PY{l+s+s1}{\PYZdl{}}\PY{l+s+s1}{\PYZbs{}}\PY{l+s+s1}{mathdefault}\PY{l+s+s1}{\PYZob{}}\PY{l+s+s1}{[ 1 / }\PY{l+s+s1}{\PYZbs{}}\PY{l+s+s1}{sqrt}\PY{l+s+s1}{\PYZob{}}\PY{l+s+s1}{\PYZbs{}}\PY{l+s+s1}{mathdefault}\PY{l+s+si}{\PYZob{}Hz\PYZcb{}}\PY{l+s+s1}{\PYZcb{}]\PYZcb{}\PYZdl{}}\PY{l+s+s1}{\PYZsq{}}\PY{p}{)}
\PY{n}{lgd}\PY{o}{=}\PY{n}{plt}\PY{o}{.}\PY{n}{legend}\PY{p}{(}\PY{p}{)}
\PY{n}{plt}\PY{o}{.}\PY{n}{savefig}\PY{p}{(}\PY{l+s+s1}{\PYZsq{}}\PY{l+s+s1}{../figs/INTRO/strain\PYZus{}compare.pdf}\PY{l+s+s1}{\PYZsq{}}\PY{p}{,} \PY{n}{dpi}\PY{o}{=}\PY{l+m+mi}{300}\PY{p}{,} \PY{n}{bbox\PYZus{}inches}\PY{o}{=}\PY{l+s+s1}{\PYZsq{}}\PY{l+s+s1}{tight}\PY{l+s+s1}{\PYZsq{}}\PY{p}{)}
\end{Verbatim}
\end{tcolorbox}

    \begin{center}
    \adjustimage{max size={0.9\linewidth}{0.9\paperheight}}{drfpmi_fr_files/drfpmi_fr_44_0.png}
    \end{center}
    { \hspace*{\fill} \\}
    
    \begin{tcolorbox}[breakable, size=fbox, boxrule=1pt, pad at break*=1mm,colback=cellbackground, colframe=cellborder]
\prompt{In}{incolor}{ }{\boxspacing}
\begin{Verbatim}[commandchars=\\\{\}]

\end{Verbatim}
\end{tcolorbox}


    % Add a bibliography block to the postdoc
    
    
    
\end{document}
