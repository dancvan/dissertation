\documentclass[11pt]{article}
    \usepackage[breakable]{tcolorbox}
    \usepackage{parskip} % Stop auto-indenting (to mimic markdown behaviour)
    

    % Basic figure setup, for now with no caption control since it's done
    % automatically by Pandoc (which extracts ![](path) syntax from Markdown).
    \usepackage{graphicx}
    % Maintain compatibility with old templates. Remove in nbconvert 6.0
    \let\Oldincludegraphics\includegraphics
    % Ensure that by default, figures have no caption (until we provide a
    % proper Figure object with a Caption API and a way to capture that
    % in the conversion process - todo).
    \usepackage{caption}
    \DeclareCaptionFormat{nocaption}{}
    \captionsetup{format=nocaption,aboveskip=0pt,belowskip=0pt}

    \usepackage{float}
    \floatplacement{figure}{H} % forces figures to be placed at the correct location
    \usepackage{xcolor} % Allow colors to be defined
    \usepackage{enumerate} % Needed for markdown enumerations to work
    \usepackage{geometry} % Used to adjust the document margins
    \usepackage{amsmath} % Equations
    \usepackage{amssymb} % Equations
    \usepackage{textcomp} % defines textquotesingle
    % Hack from http://tex.stackexchange.com/a/47451/13684:
    \AtBeginDocument{%
        \def\PYZsq{\textquotesingle}% Upright quotes in Pygmentized code
    }
    \usepackage{upquote} % Upright quotes for verbatim code
    \usepackage{eurosym} % defines \euro

    \usepackage{iftex}
    \ifPDFTeX
        \usepackage[T1]{fontenc}
        \IfFileExists{alphabeta.sty}{
              \usepackage{alphabeta}
          }{
              \usepackage[mathletters]{ucs}
              \usepackage[utf8x]{inputenc}
          }
    \else
        \usepackage{fontspec}
        \usepackage{unicode-math}
    \fi

    \usepackage{fancyvrb} % verbatim replacement that allows latex
    \usepackage{grffile} % extends the file name processing of package graphics
                         % to support a larger range
    \makeatletter % fix for old versions of grffile with XeLaTeX
    \@ifpackagelater{grffile}{2019/11/01}
    {
      % Do nothing on new versions
    }
    {
      \def\Gread@@xetex#1{%
        \IfFileExists{"\Gin@base".bb}%
        {\Gread@eps{\Gin@base.bb}}%
        {\Gread@@xetex@aux#1}%
      }
    }
    \makeatother
    \usepackage[Export]{adjustbox} % Used to constrain images to a maximum size
    \adjustboxset{max size={0.9\linewidth}{0.9\paperheight}}

    % The hyperref package gives us a pdf with properly built
    % internal navigation ('pdf bookmarks' for the table of contents,
    % internal cross-reference links, web links for URLs, etc.)
    \usepackage{hyperref}
    % The default LaTeX title has an obnoxious amount of whitespace. By default,
    % titling removes some of it. It also provides customization options.
    \usepackage{titling}
    \usepackage{longtable} % longtable support required by pandoc >1.10
    \usepackage{booktabs}  % table support for pandoc > 1.12.2
    \usepackage{array}     % table support for pandoc >= 2.11.3
    \usepackage{calc}      % table minipage width calculation for pandoc >= 2.11.1
    \usepackage[inline]{enumitem} % IRkernel/repr support (it uses the enumerate* environment)
    \usepackage[normalem]{ulem} % ulem is needed to support strikethroughs (\sout)
                                % normalem makes italics be italics, not underlines
    \usepackage{soul}      % strikethrough (\st) support for pandoc >= 3.0.0
    \usepackage{mathrsfs}
    

    
    % Colors for the hyperref package
    \definecolor{urlcolor}{rgb}{0,.145,.698}
    \definecolor{linkcolor}{rgb}{.71,0.21,0.01}
    \definecolor{citecolor}{rgb}{.12,.54,.11}

    % ANSI colors
    \definecolor{ansi-black}{HTML}{3E424D}
    \definecolor{ansi-black-intense}{HTML}{282C36}
    \definecolor{ansi-red}{HTML}{E75C58}
    \definecolor{ansi-red-intense}{HTML}{B22B31}
    \definecolor{ansi-green}{HTML}{00A250}
    \definecolor{ansi-green-intense}{HTML}{007427}
    \definecolor{ansi-yellow}{HTML}{DDB62B}
    \definecolor{ansi-yellow-intense}{HTML}{B27D12}
    \definecolor{ansi-blue}{HTML}{208FFB}
    \definecolor{ansi-blue-intense}{HTML}{0065CA}
    \definecolor{ansi-magenta}{HTML}{D160C4}
    \definecolor{ansi-magenta-intense}{HTML}{A03196}
    \definecolor{ansi-cyan}{HTML}{60C6C8}
    \definecolor{ansi-cyan-intense}{HTML}{258F8F}
    \definecolor{ansi-white}{HTML}{C5C1B4}
    \definecolor{ansi-white-intense}{HTML}{A1A6B2}
    \definecolor{ansi-default-inverse-fg}{HTML}{FFFFFF}
    \definecolor{ansi-default-inverse-bg}{HTML}{000000}

    % common color for the border for error outputs.
    \definecolor{outerrorbackground}{HTML}{FFDFDF}

    % commands and environments needed by pandoc snippets
    % extracted from the output of `pandoc -s`
    \providecommand{\tightlist}{%
      \setlength{\itemsep}{0pt}\setlength{\parskip}{0pt}}
    \DefineVerbatimEnvironment{Highlighting}{Verbatim}{commandchars=\\\{\}}
    % Add ',fontsize=\small' for more characters per line
    \newenvironment{Shaded}{}{}
    \newcommand{\KeywordTok}[1]{\textcolor[rgb]{0.00,0.44,0.13}{\textbf{{#1}}}}
    \newcommand{\DataTypeTok}[1]{\textcolor[rgb]{0.56,0.13,0.00}{{#1}}}
    \newcommand{\DecValTok}[1]{\textcolor[rgb]{0.25,0.63,0.44}{{#1}}}
    \newcommand{\BaseNTok}[1]{\textcolor[rgb]{0.25,0.63,0.44}{{#1}}}
    \newcommand{\FloatTok}[1]{\textcolor[rgb]{0.25,0.63,0.44}{{#1}}}
    \newcommand{\CharTok}[1]{\textcolor[rgb]{0.25,0.44,0.63}{{#1}}}
    \newcommand{\StringTok}[1]{\textcolor[rgb]{0.25,0.44,0.63}{{#1}}}
    \newcommand{\CommentTok}[1]{\textcolor[rgb]{0.38,0.63,0.69}{\textit{{#1}}}}
    \newcommand{\OtherTok}[1]{\textcolor[rgb]{0.00,0.44,0.13}{{#1}}}
    \newcommand{\AlertTok}[1]{\textcolor[rgb]{1.00,0.00,0.00}{\textbf{{#1}}}}
    \newcommand{\FunctionTok}[1]{\textcolor[rgb]{0.02,0.16,0.49}{{#1}}}
    \newcommand{\RegionMarkerTok}[1]{{#1}}
    \newcommand{\ErrorTok}[1]{\textcolor[rgb]{1.00,0.00,0.00}{\textbf{{#1}}}}
    \newcommand{\NormalTok}[1]{{#1}}

    % Additional commands for more recent versions of Pandoc
    \newcommand{\ConstantTok}[1]{\textcolor[rgb]{0.53,0.00,0.00}{{#1}}}
    \newcommand{\SpecialCharTok}[1]{\textcolor[rgb]{0.25,0.44,0.63}{{#1}}}
    \newcommand{\VerbatimStringTok}[1]{\textcolor[rgb]{0.25,0.44,0.63}{{#1}}}
    \newcommand{\SpecialStringTok}[1]{\textcolor[rgb]{0.73,0.40,0.53}{{#1}}}
    \newcommand{\ImportTok}[1]{{#1}}
    \newcommand{\DocumentationTok}[1]{\textcolor[rgb]{0.73,0.13,0.13}{\textit{{#1}}}}
    \newcommand{\AnnotationTok}[1]{\textcolor[rgb]{0.38,0.63,0.69}{\textbf{\textit{{#1}}}}}
    \newcommand{\CommentVarTok}[1]{\textcolor[rgb]{0.38,0.63,0.69}{\textbf{\textit{{#1}}}}}
    \newcommand{\VariableTok}[1]{\textcolor[rgb]{0.10,0.09,0.49}{{#1}}}
    \newcommand{\ControlFlowTok}[1]{\textcolor[rgb]{0.00,0.44,0.13}{\textbf{{#1}}}}
    \newcommand{\OperatorTok}[1]{\textcolor[rgb]{0.40,0.40,0.40}{{#1}}}
    \newcommand{\BuiltInTok}[1]{{#1}}
    \newcommand{\ExtensionTok}[1]{{#1}}
    \newcommand{\PreprocessorTok}[1]{\textcolor[rgb]{0.74,0.48,0.00}{{#1}}}
    \newcommand{\AttributeTok}[1]{\textcolor[rgb]{0.49,0.56,0.16}{{#1}}}
    \newcommand{\InformationTok}[1]{\textcolor[rgb]{0.38,0.63,0.69}{\textbf{\textit{{#1}}}}}
    \newcommand{\WarningTok}[1]{\textcolor[rgb]{0.38,0.63,0.69}{\textbf{\textit{{#1}}}}}


    % Define a nice break command that doesn't care if a line doesn't already
    % exist.
    \def\br{\hspace*{\fill} \\* }
    % Math Jax compatibility definitions
    \def\gt{>}
    \def\lt{<}
    \let\Oldtex\TeX
    \let\Oldlatex\LaTeX
    \renewcommand{\TeX}{\textrm{\Oldtex}}
    \renewcommand{\LaTeX}{\textrm{\Oldlatex}}
    % Document parameters
    % Document title
    \title{fpmi\_fr}
    
    
    
    
    
    
    
% Pygments definitions
\makeatletter
\def\PY@reset{\let\PY@it=\relax \let\PY@bf=\relax%
    \let\PY@ul=\relax \let\PY@tc=\relax%
    \let\PY@bc=\relax \let\PY@ff=\relax}
\def\PY@tok#1{\csname PY@tok@#1\endcsname}
\def\PY@toks#1+{\ifx\relax#1\empty\else%
    \PY@tok{#1}\expandafter\PY@toks\fi}
\def\PY@do#1{\PY@bc{\PY@tc{\PY@ul{%
    \PY@it{\PY@bf{\PY@ff{#1}}}}}}}
\def\PY#1#2{\PY@reset\PY@toks#1+\relax+\PY@do{#2}}

\@namedef{PY@tok@w}{\def\PY@tc##1{\textcolor[rgb]{0.73,0.73,0.73}{##1}}}
\@namedef{PY@tok@c}{\let\PY@it=\textit\def\PY@tc##1{\textcolor[rgb]{0.24,0.48,0.48}{##1}}}
\@namedef{PY@tok@cp}{\def\PY@tc##1{\textcolor[rgb]{0.61,0.40,0.00}{##1}}}
\@namedef{PY@tok@k}{\let\PY@bf=\textbf\def\PY@tc##1{\textcolor[rgb]{0.00,0.50,0.00}{##1}}}
\@namedef{PY@tok@kp}{\def\PY@tc##1{\textcolor[rgb]{0.00,0.50,0.00}{##1}}}
\@namedef{PY@tok@kt}{\def\PY@tc##1{\textcolor[rgb]{0.69,0.00,0.25}{##1}}}
\@namedef{PY@tok@o}{\def\PY@tc##1{\textcolor[rgb]{0.40,0.40,0.40}{##1}}}
\@namedef{PY@tok@ow}{\let\PY@bf=\textbf\def\PY@tc##1{\textcolor[rgb]{0.67,0.13,1.00}{##1}}}
\@namedef{PY@tok@nb}{\def\PY@tc##1{\textcolor[rgb]{0.00,0.50,0.00}{##1}}}
\@namedef{PY@tok@nf}{\def\PY@tc##1{\textcolor[rgb]{0.00,0.00,1.00}{##1}}}
\@namedef{PY@tok@nc}{\let\PY@bf=\textbf\def\PY@tc##1{\textcolor[rgb]{0.00,0.00,1.00}{##1}}}
\@namedef{PY@tok@nn}{\let\PY@bf=\textbf\def\PY@tc##1{\textcolor[rgb]{0.00,0.00,1.00}{##1}}}
\@namedef{PY@tok@ne}{\let\PY@bf=\textbf\def\PY@tc##1{\textcolor[rgb]{0.80,0.25,0.22}{##1}}}
\@namedef{PY@tok@nv}{\def\PY@tc##1{\textcolor[rgb]{0.10,0.09,0.49}{##1}}}
\@namedef{PY@tok@no}{\def\PY@tc##1{\textcolor[rgb]{0.53,0.00,0.00}{##1}}}
\@namedef{PY@tok@nl}{\def\PY@tc##1{\textcolor[rgb]{0.46,0.46,0.00}{##1}}}
\@namedef{PY@tok@ni}{\let\PY@bf=\textbf\def\PY@tc##1{\textcolor[rgb]{0.44,0.44,0.44}{##1}}}
\@namedef{PY@tok@na}{\def\PY@tc##1{\textcolor[rgb]{0.41,0.47,0.13}{##1}}}
\@namedef{PY@tok@nt}{\let\PY@bf=\textbf\def\PY@tc##1{\textcolor[rgb]{0.00,0.50,0.00}{##1}}}
\@namedef{PY@tok@nd}{\def\PY@tc##1{\textcolor[rgb]{0.67,0.13,1.00}{##1}}}
\@namedef{PY@tok@s}{\def\PY@tc##1{\textcolor[rgb]{0.73,0.13,0.13}{##1}}}
\@namedef{PY@tok@sd}{\let\PY@it=\textit\def\PY@tc##1{\textcolor[rgb]{0.73,0.13,0.13}{##1}}}
\@namedef{PY@tok@si}{\let\PY@bf=\textbf\def\PY@tc##1{\textcolor[rgb]{0.64,0.35,0.47}{##1}}}
\@namedef{PY@tok@se}{\let\PY@bf=\textbf\def\PY@tc##1{\textcolor[rgb]{0.67,0.36,0.12}{##1}}}
\@namedef{PY@tok@sr}{\def\PY@tc##1{\textcolor[rgb]{0.64,0.35,0.47}{##1}}}
\@namedef{PY@tok@ss}{\def\PY@tc##1{\textcolor[rgb]{0.10,0.09,0.49}{##1}}}
\@namedef{PY@tok@sx}{\def\PY@tc##1{\textcolor[rgb]{0.00,0.50,0.00}{##1}}}
\@namedef{PY@tok@m}{\def\PY@tc##1{\textcolor[rgb]{0.40,0.40,0.40}{##1}}}
\@namedef{PY@tok@gh}{\let\PY@bf=\textbf\def\PY@tc##1{\textcolor[rgb]{0.00,0.00,0.50}{##1}}}
\@namedef{PY@tok@gu}{\let\PY@bf=\textbf\def\PY@tc##1{\textcolor[rgb]{0.50,0.00,0.50}{##1}}}
\@namedef{PY@tok@gd}{\def\PY@tc##1{\textcolor[rgb]{0.63,0.00,0.00}{##1}}}
\@namedef{PY@tok@gi}{\def\PY@tc##1{\textcolor[rgb]{0.00,0.52,0.00}{##1}}}
\@namedef{PY@tok@gr}{\def\PY@tc##1{\textcolor[rgb]{0.89,0.00,0.00}{##1}}}
\@namedef{PY@tok@ge}{\let\PY@it=\textit}
\@namedef{PY@tok@gs}{\let\PY@bf=\textbf}
\@namedef{PY@tok@ges}{\let\PY@bf=\textbf\let\PY@it=\textit}
\@namedef{PY@tok@gp}{\let\PY@bf=\textbf\def\PY@tc##1{\textcolor[rgb]{0.00,0.00,0.50}{##1}}}
\@namedef{PY@tok@go}{\def\PY@tc##1{\textcolor[rgb]{0.44,0.44,0.44}{##1}}}
\@namedef{PY@tok@gt}{\def\PY@tc##1{\textcolor[rgb]{0.00,0.27,0.87}{##1}}}
\@namedef{PY@tok@err}{\def\PY@bc##1{{\setlength{\fboxsep}{\string -\fboxrule}\fcolorbox[rgb]{1.00,0.00,0.00}{1,1,1}{\strut ##1}}}}
\@namedef{PY@tok@kc}{\let\PY@bf=\textbf\def\PY@tc##1{\textcolor[rgb]{0.00,0.50,0.00}{##1}}}
\@namedef{PY@tok@kd}{\let\PY@bf=\textbf\def\PY@tc##1{\textcolor[rgb]{0.00,0.50,0.00}{##1}}}
\@namedef{PY@tok@kn}{\let\PY@bf=\textbf\def\PY@tc##1{\textcolor[rgb]{0.00,0.50,0.00}{##1}}}
\@namedef{PY@tok@kr}{\let\PY@bf=\textbf\def\PY@tc##1{\textcolor[rgb]{0.00,0.50,0.00}{##1}}}
\@namedef{PY@tok@bp}{\def\PY@tc##1{\textcolor[rgb]{0.00,0.50,0.00}{##1}}}
\@namedef{PY@tok@fm}{\def\PY@tc##1{\textcolor[rgb]{0.00,0.00,1.00}{##1}}}
\@namedef{PY@tok@vc}{\def\PY@tc##1{\textcolor[rgb]{0.10,0.09,0.49}{##1}}}
\@namedef{PY@tok@vg}{\def\PY@tc##1{\textcolor[rgb]{0.10,0.09,0.49}{##1}}}
\@namedef{PY@tok@vi}{\def\PY@tc##1{\textcolor[rgb]{0.10,0.09,0.49}{##1}}}
\@namedef{PY@tok@vm}{\def\PY@tc##1{\textcolor[rgb]{0.10,0.09,0.49}{##1}}}
\@namedef{PY@tok@sa}{\def\PY@tc##1{\textcolor[rgb]{0.73,0.13,0.13}{##1}}}
\@namedef{PY@tok@sb}{\def\PY@tc##1{\textcolor[rgb]{0.73,0.13,0.13}{##1}}}
\@namedef{PY@tok@sc}{\def\PY@tc##1{\textcolor[rgb]{0.73,0.13,0.13}{##1}}}
\@namedef{PY@tok@dl}{\def\PY@tc##1{\textcolor[rgb]{0.73,0.13,0.13}{##1}}}
\@namedef{PY@tok@s2}{\def\PY@tc##1{\textcolor[rgb]{0.73,0.13,0.13}{##1}}}
\@namedef{PY@tok@sh}{\def\PY@tc##1{\textcolor[rgb]{0.73,0.13,0.13}{##1}}}
\@namedef{PY@tok@s1}{\def\PY@tc##1{\textcolor[rgb]{0.73,0.13,0.13}{##1}}}
\@namedef{PY@tok@mb}{\def\PY@tc##1{\textcolor[rgb]{0.40,0.40,0.40}{##1}}}
\@namedef{PY@tok@mf}{\def\PY@tc##1{\textcolor[rgb]{0.40,0.40,0.40}{##1}}}
\@namedef{PY@tok@mh}{\def\PY@tc##1{\textcolor[rgb]{0.40,0.40,0.40}{##1}}}
\@namedef{PY@tok@mi}{\def\PY@tc##1{\textcolor[rgb]{0.40,0.40,0.40}{##1}}}
\@namedef{PY@tok@il}{\def\PY@tc##1{\textcolor[rgb]{0.40,0.40,0.40}{##1}}}
\@namedef{PY@tok@mo}{\def\PY@tc##1{\textcolor[rgb]{0.40,0.40,0.40}{##1}}}
\@namedef{PY@tok@ch}{\let\PY@it=\textit\def\PY@tc##1{\textcolor[rgb]{0.24,0.48,0.48}{##1}}}
\@namedef{PY@tok@cm}{\let\PY@it=\textit\def\PY@tc##1{\textcolor[rgb]{0.24,0.48,0.48}{##1}}}
\@namedef{PY@tok@cpf}{\let\PY@it=\textit\def\PY@tc##1{\textcolor[rgb]{0.24,0.48,0.48}{##1}}}
\@namedef{PY@tok@c1}{\let\PY@it=\textit\def\PY@tc##1{\textcolor[rgb]{0.24,0.48,0.48}{##1}}}
\@namedef{PY@tok@cs}{\let\PY@it=\textit\def\PY@tc##1{\textcolor[rgb]{0.24,0.48,0.48}{##1}}}

\def\PYZbs{\char`\\}
\def\PYZus{\char`\_}
\def\PYZob{\char`\{}
\def\PYZcb{\char`\}}
\def\PYZca{\char`\^}
\def\PYZam{\char`\&}
\def\PYZlt{\char`\<}
\def\PYZgt{\char`\>}
\def\PYZsh{\char`\#}
\def\PYZpc{\char`\%}
\def\PYZdl{\char`\$}
\def\PYZhy{\char`\-}
\def\PYZsq{\char`\'}
\def\PYZdq{\char`\"}
\def\PYZti{\char`\~}
% for compatibility with earlier versions
\def\PYZat{@}
\def\PYZlb{[}
\def\PYZrb{]}
\makeatother


    % For linebreaks inside Verbatim environment from package fancyvrb.
    \makeatletter
        \newbox\Wrappedcontinuationbox
        \newbox\Wrappedvisiblespacebox
        \newcommand*\Wrappedvisiblespace {\textcolor{red}{\textvisiblespace}}
        \newcommand*\Wrappedcontinuationsymbol {\textcolor{red}{\llap{\tiny$\m@th\hookrightarrow$}}}
        \newcommand*\Wrappedcontinuationindent {3ex }
        \newcommand*\Wrappedafterbreak {\kern\Wrappedcontinuationindent\copy\Wrappedcontinuationbox}
        % Take advantage of the already applied Pygments mark-up to insert
        % potential linebreaks for TeX processing.
        %        {, <, #, %, $, ' and ": go to next line.
        %        _, }, ^, &, >, - and ~: stay at end of broken line.
        % Use of \textquotesingle for straight quote.
        \newcommand*\Wrappedbreaksatspecials {%
            \def\PYGZus{\discretionary{\char`\_}{\Wrappedafterbreak}{\char`\_}}%
            \def\PYGZob{\discretionary{}{\Wrappedafterbreak\char`\{}{\char`\{}}%
            \def\PYGZcb{\discretionary{\char`\}}{\Wrappedafterbreak}{\char`\}}}%
            \def\PYGZca{\discretionary{\char`\^}{\Wrappedafterbreak}{\char`\^}}%
            \def\PYGZam{\discretionary{\char`\&}{\Wrappedafterbreak}{\char`\&}}%
            \def\PYGZlt{\discretionary{}{\Wrappedafterbreak\char`\<}{\char`\<}}%
            \def\PYGZgt{\discretionary{\char`\>}{\Wrappedafterbreak}{\char`\>}}%
            \def\PYGZsh{\discretionary{}{\Wrappedafterbreak\char`\#}{\char`\#}}%
            \def\PYGZpc{\discretionary{}{\Wrappedafterbreak\char`\%}{\char`\%}}%
            \def\PYGZdl{\discretionary{}{\Wrappedafterbreak\char`\$}{\char`\$}}%
            \def\PYGZhy{\discretionary{\char`\-}{\Wrappedafterbreak}{\char`\-}}%
            \def\PYGZsq{\discretionary{}{\Wrappedafterbreak\textquotesingle}{\textquotesingle}}%
            \def\PYGZdq{\discretionary{}{\Wrappedafterbreak\char`\"}{\char`\"}}%
            \def\PYGZti{\discretionary{\char`\~}{\Wrappedafterbreak}{\char`\~}}%
        }
        % Some characters . , ; ? ! / are not pygmentized.
        % This macro makes them "active" and they will insert potential linebreaks
        \newcommand*\Wrappedbreaksatpunct {%
            \lccode`\~`\.\lowercase{\def~}{\discretionary{\hbox{\char`\.}}{\Wrappedafterbreak}{\hbox{\char`\.}}}%
            \lccode`\~`\,\lowercase{\def~}{\discretionary{\hbox{\char`\,}}{\Wrappedafterbreak}{\hbox{\char`\,}}}%
            \lccode`\~`\;\lowercase{\def~}{\discretionary{\hbox{\char`\;}}{\Wrappedafterbreak}{\hbox{\char`\;}}}%
            \lccode`\~`\:\lowercase{\def~}{\discretionary{\hbox{\char`\:}}{\Wrappedafterbreak}{\hbox{\char`\:}}}%
            \lccode`\~`\?\lowercase{\def~}{\discretionary{\hbox{\char`\?}}{\Wrappedafterbreak}{\hbox{\char`\?}}}%
            \lccode`\~`\!\lowercase{\def~}{\discretionary{\hbox{\char`\!}}{\Wrappedafterbreak}{\hbox{\char`\!}}}%
            \lccode`\~`\/\lowercase{\def~}{\discretionary{\hbox{\char`\/}}{\Wrappedafterbreak}{\hbox{\char`\/}}}%
            \catcode`\.\active
            \catcode`\,\active
            \catcode`\;\active
            \catcode`\:\active
            \catcode`\?\active
            \catcode`\!\active
            \catcode`\/\active
            \lccode`\~`\~
        }
    \makeatother

    \let\OriginalVerbatim=\Verbatim
    \makeatletter
    \renewcommand{\Verbatim}[1][1]{%
        %\parskip\z@skip
        \sbox\Wrappedcontinuationbox {\Wrappedcontinuationsymbol}%
        \sbox\Wrappedvisiblespacebox {\FV@SetupFont\Wrappedvisiblespace}%
        \def\FancyVerbFormatLine ##1{\hsize\linewidth
            \vtop{\raggedright\hyphenpenalty\z@\exhyphenpenalty\z@
                \doublehyphendemerits\z@\finalhyphendemerits\z@
                \strut ##1\strut}%
        }%
        % If the linebreak is at a space, the latter will be displayed as visible
        % space at end of first line, and a continuation symbol starts next line.
        % Stretch/shrink are however usually zero for typewriter font.
        \def\FV@Space {%
            \nobreak\hskip\z@ plus\fontdimen3\font minus\fontdimen4\font
            \discretionary{\copy\Wrappedvisiblespacebox}{\Wrappedafterbreak}
            {\kern\fontdimen2\font}%
        }%

        % Allow breaks at special characters using \PYG... macros.
        \Wrappedbreaksatspecials
        % Breaks at punctuation characters . , ; ? ! and / need catcode=\active
        \OriginalVerbatim[#1,codes*=\Wrappedbreaksatpunct]%
    }
    \makeatother

    % Exact colors from NB
    \definecolor{incolor}{HTML}{303F9F}
    \definecolor{outcolor}{HTML}{D84315}
    \definecolor{cellborder}{HTML}{CFCFCF}
    \definecolor{cellbackground}{HTML}{F7F7F7}

    % prompt
    \makeatletter
    \newcommand{\boxspacing}{\kern\kvtcb@left@rule\kern\kvtcb@boxsep}
    \makeatother
    \newcommand{\prompt}[4]{
        {\ttfamily\llap{{\color{#2}[#3]:\hspace{3pt}#4}}\vspace{-\baselineskip}}
    }
    

    
    % Prevent overflowing lines due to hard-to-break entities
    \sloppy
    % Setup hyperref package
    \hypersetup{
      breaklinks=true,  % so long urls are correctly broken across lines
      colorlinks=true,
      urlcolor=urlcolor,
      linkcolor=linkcolor,
      citecolor=citecolor,
      }
    % Slightly bigger margins than the latex defaults
    
    \geometry{verbose,tmargin=1in,bmargin=1in,lmargin=1in,rmargin=1in}
    
    

\pagenumbering{gobble}
\begin{document}
    
    
    

    
    \hypertarget{fpmi-frequency-response-to-gravitational-wave-perturbation}{%
\section{FPMI frequency response to gravitational wave
perturbation*}\label{fpmi-frequency-response-to-gravitational-wave-perturbation}}

    \begin{tcolorbox}[breakable, size=fbox, boxrule=1pt, pad at break*=1mm,colback=cellbackground, colframe=cellborder]
\prompt{In}{incolor}{1}{\boxspacing}
\begin{Verbatim}[commandchars=\\\{\}]
\PY{k+kn}{import} \PY{n+nn}{numpy} \PY{k}{as} \PY{n+nn}{np} 
\PY{k+kn}{import} \PY{n+nn}{matplotlib}\PY{n+nn}{.}\PY{n+nn}{pyplot} \PY{k}{as} \PY{n+nn}{plt}
\PY{k+kn}{import} \PY{n+nn}{scipy}\PY{n+nn}{.}\PY{n+nn}{signal} \PY{k}{as} \PY{n+nn}{sig}
\PY{k+kn}{import} \PY{n+nn}{os}
\PY{k+kn}{from} \PY{n+nn}{ifo\PYZus{}configs} \PY{k+kn}{import} \PY{n}{mich\PYZus{}freq\PYZus{}resp} \PY{k}{as} \PY{n}{MICH}
\PY{k+kn}{from} \PY{n+nn}{ifo\PYZus{}configs} \PY{k+kn}{import} \PY{n}{fpmi\PYZus{}freq\PYZus{}resp} \PY{k}{as} \PY{n}{FPMI}
\PY{k+kn}{from} \PY{n+nn}{ifo\PYZus{}configs} \PY{k+kn}{import} \PY{n}{N\PYZus{}shot}\PY{p}{,} \PY{n}{bode\PYZus{}amp}\PY{p}{,} \PY{n}{bode\PYZus{}ph}
\PY{n}{plt\PYZus{}style\PYZus{}dir} \PY{o}{=} \PY{l+s+s1}{\PYZsq{}}\PY{l+s+s1}{stash/}\PY{l+s+s1}{\PYZsq{}}
\PY{k}{if} \PY{n}{os}\PY{o}{.}\PY{n}{path}\PY{o}{.}\PY{n}{isdir}\PY{p}{(}\PY{n}{plt\PYZus{}style\PYZus{}dir}\PY{p}{)} \PY{o}{==} \PY{k+kc}{True}\PY{p}{:}
    \PY{n}{plt}\PY{o}{.}\PY{n}{style}\PY{o}{.}\PY{n}{use}\PY{p}{(}\PY{n}{plt\PYZus{}style\PYZus{}dir} \PY{o}{+} \PY{l+s+s1}{\PYZsq{}}\PY{l+s+s1}{ppt2latexsubfig.mplstyle}\PY{l+s+s1}{\PYZsq{}}\PY{p}{)}
\PY{n}{plt}\PY{o}{.}\PY{n}{rcParams}\PY{p}{[}\PY{l+s+s2}{\PYZdq{}}\PY{l+s+s2}{font.family}\PY{l+s+s2}{\PYZdq{}}\PY{p}{]} \PY{o}{=} \PY{l+s+s2}{\PYZdq{}}\PY{l+s+s2}{Times New Roman}\PY{l+s+s2}{\PYZdq{}}
\PY{n}{line\PYZus{}width}\PY{o}{=}\PY{l+m+mf}{7.5}
\end{Verbatim}
\end{tcolorbox}

    I often find it helpful to revisit that which we will need to build upon
before moving forward, especially if one has not done this derivation
more than a couple of times. Sooooo\ldots{}

\hypertarget{last-time-when-working-with-ifo-configs}{%
\subsection{Last time when working with IFO
configs:}\label{last-time-when-working-with-ifo-configs}}

\textbf{Michelson frequency response to gravitational wave}

The Michelson interferometer by it's design is a measurement device that
can detect small changes in light phase between it's two perpendicular
arms.

A gravitational wave offers a unique phase differential that can be
characterized mathematically by the following:

\[
\phi_X - \phi_Y = \int_{t-2L/c}^{t} \Omega \bigg[1 + \frac{1}{2}h(t)\bigg]dt - \int_{t-2L/c}^{t} \Omega \bigg[1 - \frac{1}{2}h(t)\bigg]dt \label{eq1}\tag{1}
\]

This eventually led us to the time independent phase response to a
monochromatic gravitational wave (\(h(t)\)):

\[
\Delta \phi (\omega) = h_0\frac{2 L \Omega}{c}e^{-i L \omega / c} \frac{\mathrm{sin}(L \omega /c)}{L \omega /c} \label{eq2}\tag{2}
\]

Now, we are going to look at how Fabry Perót cavitites can help our
Michelson become a gravitational wave observatory

    \hypertarget{derivation}{%
\section{Derivation}\label{derivation}}

Let's start with the simple Fabry Perót cavity. The following are
equations that characterize the circulating and reflected fields (both
critical to measuring the phase response of the FP cavity to GWs):

\[
E(t) = t_1 E_{in} + r_1 r_2 E(t - 2T) e^{-i \Delta \phi(t)} \label{eq3}\tag{3}
\]

\[
E_r(t) = -r_1 E_{in} + t_1 r_2 E(t - 2T) e^{-i \Delta \phi(t)} \label{eq4}\tag{4}
\]

\(T = L/c\) is the time it takes light to reach the end of the cavity
and \(\Delta \phi(t)\) is the phase rotation.

We can define the static phase rotation (no GW passing through) as :
\[\Delta \phi = 2kL = 4 \pi L /\lambda_{opt}  \label{eq5}\tag{5}\]

And if L is tuned just right \(2kL = 2 \pi n\) so the cavity is just
tuned for resonance

If we put a gravitational wave in the mix we redefine this phase
rotation as such that:
\[\Delta \phi =  \frac{\omega_0}{2} \int_{t-\frac{2L}{c}}^{t} h(t')dt'  \label{eq6}\tag{6}\]

This assumes that the static phase rotation satisfies
\(2\omega_0L/c = 2 \pi n\). Which is the same thing that we said above
but with different symbols (because we're fancy ;D )

Say that we have something that does throw the cavity slightly off
resonance.. doesn't have to be a gravitational wave\ldots{} but that's
what we hope for. ANYWAY\ldots{}

If the \(\Delta \phi\) becomes such that the cavity is thrown off
resonance we get a time dependent intra-cavity field:

\[ E(t) = \bar{E} + \delta E(t) \label{eq7}\tag{7} \]

and if the phase rotation (\(\Delta \phi\)) is super small\ldots{} which
is pretty much guaranteed with gravy waves, we can say:

\[ e^{i\Delta \phi} = 1- i \Delta \phi \label{eq8}\tag{8}\]

Using equations \ref{eq7} and \ref{eq8} in \ref{eq3} we get:

\[ \bar{E} + \delta E(t) = t_1 E_{in} -r_1r_2\bar{E} + r_1r_2 \delta E(t-2T) - ir_1r_2\bar{E}\Delta \phi(t)) \label{eq9}\tag{9} \]

We can parse this into time dependent and time independent terms:

\[ \bar{E} = t_1 E_{in} -r_1r_2\bar{E} \label{eq10}\tag{10} \]

\[ \delta E(t) = r_1r_2 \delta E(t-2T) - ir_1r_2\bar{E}\Delta \phi(t) \label{eq11}\tag{11} \]

Since the time dependent phase information is encoded in \ref{eq11} we
will take the laplace transform of this equation to yield:

\[\delta E(s) = -i \frac{r_1r_2 \bar{E}}{1-r_1r_2e^{-2sT}} \Delta \phi(s)\]

\textbf{YAS!} we are now one step closer to getting a useful expression
for the phase response. But again.. what does this last equation mean?
That last equation is how the change in the electric field directly
relates to a small perturbation in phase (which could be either a small
change in laser frequency or length modulation)

Now.. we're not done yet because that last expression does not tell us
the entire story yet.. we want to see how this effects the phase
differential with the \textbf{reflected} electric field.

To do this.. we have to combine equations \ref{eq3} and \ref{eq4}.
(\emph{an easy way to do this is to get rid of the \$ r\_2 E(t - 2T)
e\^{}\{-i \Delta \phi(t)\}\$ term}) :

\[ E_r(t) = \frac{t_1}{r_1}E(t) - \frac{t_1^2 + r_2^2}{r_1} E_{in}\]

if the cavity is unperturbed:

\[ \bar{E}_r = \bigg(\frac{r_2(r_1^2 + t_1^2) - r_1}{t_1} \bigg) \bar{E} \]

and if we perturb the cavity we see that the change in the intra-cavity
field is directly related to the change in the reflected field:

\[ \Delta \phi_r(s) \equiv \frac{\delta E(s)}{\bar{E}} = \frac{t_1^2r_2}{(t_1^2 + r_1^2)r_2 -r_1} \frac{\Delta \phi(s)}{1-r_1r_2e^{-2sT}}\]

This implies that there is an additional frequency dependent factor in
your phase shift and this translates into your FPMI transfer function
as:

\[ H_{FPMI}(\omega_g) = \frac{2 \Delta \phi_r(\omega_g)}{h(\omega_g)} =  \frac{t_1^2r_2}{(t_1^2 + r_1^2)r_2 -r_1} \frac{H_{\mathrm{MI}}(\omega_g, L)}{1-r_1r_2e^{-2i \omega_g L /c }}  \]

Whew\ldots. that was a lot\ldots. now let's code it up

    Since we can seperate the calculation into two.. I'm going to parse out
the calculation between the constant Fabry Perót term and the term with
the frequency dependence. But first, lets set up our parameters for our
FPMI:

    \begin{tcolorbox}[breakable, size=fbox, boxrule=1pt, pad at break*=1mm,colback=cellbackground, colframe=cellborder]
\prompt{In}{incolor}{2}{\boxspacing}
\begin{Verbatim}[commandchars=\\\{\}]
\PY{c+c1}{\PYZsh{} Some parameters}
\PY{n}{cee} \PY{o}{=} \PY{n}{np}\PY{o}{.}\PY{n}{float64}\PY{p}{(}\PY{l+m+mi}{299792458}\PY{p}{)}
\PY{n}{OMEG} \PY{o}{=} \PY{n}{np}\PY{o}{.}\PY{n}{float64}\PY{p}{(}\PY{l+m+mi}{2}\PY{o}{*}\PY{n}{np}\PY{o}{.}\PY{n}{pi}\PY{o}{*}\PY{n}{cee}\PY{o}{/}\PY{p}{(}\PY{l+m+mf}{1064.0}\PY{o}{*}\PY{l+m+mf}{1e\PYZhy{}9}\PY{p}{)}\PY{p}{)}
\PY{n}{L} \PY{o}{=} \PY{n}{np}\PY{o}{.}\PY{n}{float64}\PY{p}{(}\PY{l+m+mf}{4000.0}\PY{p}{)}
\PY{n}{nu} \PY{o}{=} \PY{n}{np}\PY{o}{.}\PY{n}{arange}\PY{p}{(}\PY{l+m+mi}{1}\PY{p}{,} \PY{l+m+mi}{1000000}\PY{p}{,} \PY{l+m+mi}{1}\PY{p}{)}
\PY{n}{nat\PYZus{}nu} \PY{o}{=} \PY{p}{[}\PY{n}{np}\PY{o}{.}\PY{n}{float64}\PY{p}{(}\PY{n}{i}\PY{o}{*}\PY{l+m+mi}{2}\PY{o}{*}\PY{n}{np}\PY{o}{.}\PY{n}{pi}\PY{p}{)} \PY{k}{for} \PY{n}{i} \PY{o+ow}{in} \PY{n}{nu}\PY{p}{]}
\PY{n}{h\PYZus{}0} \PY{o}{=} \PY{n}{np}\PY{o}{.}\PY{n}{float64}\PY{p}{(}\PY{l+m+mi}{1}\PY{p}{)}

\PY{n}{PHI\PYZus{}0} \PY{o}{=} \PY{n}{np}\PY{o}{.}\PY{n}{pi}\PY{o}{/}\PY{l+m+mi}{2} \PY{c+c1}{\PYZsh{}[rad]}
\PY{n}{P\PYZus{}IN} \PY{o}{=} \PY{l+m+mi}{25}

\PY{n}{T\PYZus{}1} \PY{o}{=} \PY{l+m+mf}{.014}
\PY{c+c1}{\PYZsh{}T\PYZus{}1 = 25e\PYZhy{}6 }
\PY{n}{T\PYZus{}2} \PY{o}{=} \PY{l+m+mf}{50e\PYZhy{}6}
\PY{n}{R\PYZus{}1} \PY{o}{=} \PY{l+m+mi}{1}\PY{o}{\PYZhy{}}\PY{n}{T\PYZus{}1}
\PY{n}{R\PYZus{}2} \PY{o}{=} \PY{l+m+mi}{1}\PY{o}{\PYZhy{}}\PY{n}{T\PYZus{}2}

\PY{n}{t\PYZus{}1} \PY{o}{=} \PY{n}{T\PYZus{}1}\PY{o}{*}\PY{o}{*}\PY{l+m+mf}{.5}
\PY{n}{r\PYZus{}1} \PY{o}{=} \PY{n}{R\PYZus{}1}\PY{o}{*}\PY{o}{*}\PY{l+m+mf}{.5}
\PY{n}{r\PYZus{}2} \PY{o}{=} \PY{n}{R\PYZus{}2}\PY{o}{*}\PY{o}{*}\PY{l+m+mf}{.5}
 
\end{Verbatim}
\end{tcolorbox}

    Now we can compute:
\[ H_{FPMI}(\omega_g) =  \frac{t_1^2r_2}{(t_1^2 + r_1^2)r_2 -r_1}\cdot \frac{H_{\mathrm{MI}}(\omega_g, L)}{1-r_1r_2e^{-2i \omega_g L /c }}  \]

    \begin{tcolorbox}[breakable, size=fbox, boxrule=1pt, pad at break*=1mm,colback=cellbackground, colframe=cellborder]
\prompt{In}{incolor}{3}{\boxspacing}
\begin{Verbatim}[commandchars=\\\{\}]
FPMI\PY{o}{?}
\end{Verbatim}
\end{tcolorbox}

    
    \begin{Verbatim}[commandchars=\\\{\}]
\textcolor{ansi-red}{Signature:} FPMI\textcolor{ansi-blue}{(}freq\textcolor{ansi-blue}{,} r\_1\textcolor{ansi-blue}{,} t\_1\textcolor{ansi-blue}{,} r\_2\textcolor{ansi-blue}{,} L\textcolor{ansi-blue}{,} phi\_0\textcolor{ansi-blue}{,} P\_in\textcolor{ansi-blue}{,} OMEGA\textcolor{ansi-blue}{,} low\_pass\textcolor{ansi-blue}{=}\textcolor{ansi-green}{False}\textcolor{ansi-blue}{)}
\textcolor{ansi-red}{Docstring:}
FABRY PEROT MICHELSON FREQUENCY RESPONSE CALCULATOR
freq : standard (gravitational wave) frequency [Hz]
r\_1, t\_1, r\_2: Assuming arm symmetry where the ITM has r\_1, t\_1 coefficients and the ETM has a r\_2 reflectivity coefficient. Also assumes no loss. [arb]
OMEGA: OPTICAL angular frequency [rad Hz]
Length: Michelson ifo arm length [m]
phi\_0 : static differential arm length tuning phase [rad]
\textcolor{ansi-red}{File:}      \textasciitilde{}/Documents/git/SU/dissertation/code/ifo\_configs.py
\textcolor{ansi-red}{Type:}      function

    \end{Verbatim}

    
    \begin{tcolorbox}[breakable, size=fbox, boxrule=1pt, pad at break*=1mm,colback=cellbackground, colframe=cellborder]
\prompt{In}{incolor}{4}{\boxspacing}
\begin{Verbatim}[commandchars=\\\{\}]
\PY{n}{H\PYZus{}FPMI} \PY{o}{=} \PY{n}{FPMI}\PY{p}{(}\PY{n}{nu}\PY{p}{,} \PY{n}{r\PYZus{}1}\PY{p}{,} \PY{n}{t\PYZus{}1}\PY{p}{,} \PY{n}{r\PYZus{}2}\PY{p}{,} \PY{n}{L}\PY{p}{,} \PY{n}{PHI\PYZus{}0}\PY{p}{,} \PY{n}{P\PYZus{}IN}\PY{p}{,} \PY{n}{OMEG}\PY{p}{)}
\end{Verbatim}
\end{tcolorbox}

    We estimate the FP's pole frequency
\[  1 - r_1 r_2 e^{-2i \omega_g L / c} = 0 \] therefore when:
\[ e^{-i \omega_g L / c} = \frac{1}{\sqrt{r_1 r_2}} \] we acquire the
pole frequency \(\omega_\mathrm{pole}\) as indicated in the low pass
\[ f_\mathrm{pole} = \frac{1}{4\pi \tau_{s}} =  \frac{c}{4 \pi L} \frac{1- r_1 r_2}{\sqrt{r_1 r_2}} = \frac{\nu_\mathrm{FSR}}{2 \pi} \frac{1- r_1 r_2}{\sqrt{r_1 r_2}} = \frac{\nu_\mathrm{FSR}}{\mathcal{F}} \]

ALso, understanding that the cavity Finesse can be defined as

\[ \mathcal{F} = \frac{\pi \sqrt{r_i r_e}}{1- r_i r_e} \]

we also can invert for a high value of finesse \[ \mathcal{F}
\textgreater\textgreater{} \pi \]:

\[ r_i r_e \approx 1 - \frac{\pi}{\mathcal{F}} \]

    \begin{tcolorbox}[breakable, size=fbox, boxrule=1pt, pad at break*=1mm,colback=cellbackground, colframe=cellborder]
\prompt{In}{incolor}{5}{\boxspacing}
\begin{Verbatim}[commandchars=\\\{\}]
\PY{n}{f\PYZus{}pole} \PY{o}{=} \PY{l+m+mi}{1}\PY{o}{/}\PY{p}{(}\PY{p}{(}\PY{p}{(}\PY{l+m+mi}{4}\PY{o}{*}\PY{n}{np}\PY{o}{.}\PY{n}{pi}\PY{o}{*}\PY{n}{L}\PY{p}{)}\PY{o}{*}\PY{n}{np}\PY{o}{.}\PY{n}{sqrt}\PY{p}{(}\PY{n}{r\PYZus{}1}\PY{o}{*}\PY{n}{r\PYZus{}2}\PY{p}{)}\PY{p}{)}\PY{o}{/}\PY{p}{(}\PY{n}{cee}\PY{o}{*}\PY{p}{(}\PY{l+m+mi}{1}\PY{o}{\PYZhy{}}\PY{n}{r\PYZus{}1}\PY{o}{*}\PY{n}{r\PYZus{}2}\PY{p}{)}\PY{p}{)}\PY{p}{)}
\PY{k}{def} \PY{n+nf}{fpmi\PYZus{}lp}\PY{p}{(}\PY{n}{freq}\PY{p}{,} \PY{n}{cav\PYZus{}pole}\PY{p}{)}\PY{p}{:}
    \PY{k}{return} \PY{l+m+mi}{1}\PY{o}{/}\PY{p}{(}\PY{l+m+mi}{1} \PY{o}{+} \PY{l+m+mi}{1}\PY{n}{j}\PY{o}{*}\PY{p}{(}\PY{n}{freq}\PY{o}{/}\PY{n}{cav\PYZus{}pole}\PY{p}{)}\PY{p}{)}\PY{c+c1}{\PYZsh{}*np.exp(1j*freq/cav\PYZus{}pole))}
\PY{n}{H\PYZus{}FPMI\PYZus{}LP} \PY{o}{=} \PY{n}{fpmi\PYZus{}lp}\PY{p}{(}\PY{n}{nu}\PY{p}{,} \PY{n}{f\PYZus{}pole}\PY{p}{)}
\end{Verbatim}
\end{tcolorbox}

    Might as well compare it to our Michelson response:
\[ H_{\mathrm{MI}}(\omega_g) = \frac{2 L \Omega}{c}e^{-i L \omega / c} \frac{\mathrm{sin}(L \omega /c)}{L \omega /c} \]

    \begin{tcolorbox}[breakable, size=fbox, boxrule=1pt, pad at break*=1mm,colback=cellbackground, colframe=cellborder]
\prompt{In}{incolor}{6}{\boxspacing}
\begin{Verbatim}[commandchars=\\\{\}]
MICH\PY{o}{?}
\end{Verbatim}
\end{tcolorbox}

    
    \begin{Verbatim}[commandchars=\\\{\}]
\textcolor{ansi-red}{Signature:} MICH\textcolor{ansi-blue}{(}freq\textcolor{ansi-blue}{,} Length\textcolor{ansi-blue}{,} phi\_0\textcolor{ansi-blue}{,} P\_in\textcolor{ansi-blue}{,} OMEGA\textcolor{ansi-blue}{)}
\textcolor{ansi-red}{Docstring:}
MICHELSON FREQEUNCY RESPONSE CALCULATOR
freq : standard (gravitational wave) frequency [Hz]
Length : Michelson ifo arm length [m]
phi\_0 : static differential arm length tuning phase [rad]
P\_in : input power [W] 
\textcolor{ansi-red}{File:}      \textasciitilde{}/Documents/git/SU/dissertation/code/ifo\_configs.py
\textcolor{ansi-red}{Type:}      function

    \end{Verbatim}
    
    \begin{tcolorbox}[breakable, size=fbox, boxrule=1pt, pad at break*=1mm,colback=cellbackground, colframe=cellborder]
\prompt{In}{incolor}{7}{\boxspacing}
\begin{Verbatim}[commandchars=\\\{\}]
\PY{n}{H\PYZus{}MICH} \PY{o}{=} \PY{n}{MICH}\PY{p}{(}\PY{n}{nu}\PY{p}{,} \PY{n}{L}\PY{p}{,} \PY{n}{PHI\PYZus{}0}\PY{p}{,} \PY{n}{P\PYZus{}IN}\PY{p}{,} \PY{n}{OMEG}\PY{p}{)}
\end{Verbatim}
\end{tcolorbox}

    \begin{tcolorbox}[breakable, size=fbox, boxrule=1pt, pad at break*=1mm,colback=cellbackground, colframe=cellborder]
\prompt{In}{incolor}{8}{\boxspacing}
\begin{Verbatim}[commandchars=\\\{\}]
\PY{n}{fig}\PY{p}{,} \PY{n}{ax1} \PY{o}{=} \PY{n}{plt}\PY{o}{.}\PY{n}{subplots}\PY{p}{(}\PY{p}{)}
\PY{n}{ax1}\PY{o}{.}\PY{n}{set\PYZus{}xlabel}\PY{p}{(}\PY{l+s+s1}{\PYZsq{}}\PY{l+s+s1}{frequency [Hz]}\PY{l+s+s1}{\PYZsq{}}\PY{p}{)}
\PY{n}{ax1}\PY{o}{.}\PY{n}{set\PYZus{}ylabel}\PY{p}{(}\PY{l+s+s1}{\PYZsq{}}\PY{l+s+s1}{H\PYZdl{}\PYZus{}}\PY{l+s+s1}{\PYZbs{}}\PY{l+s+s1}{mathdefault}\PY{l+s+si}{\PYZob{}FPMI\PYZcb{}}\PY{l+s+s1}{ }\PY{l+s+s1}{\PYZbs{}}\PY{l+s+s1}{; }\PY{l+s+s1}{\PYZbs{}}\PY{l+s+s1}{mathdefault}\PY{l+s+s1}{\PYZob{}}\PY{l+s+s1}{ [W / m] \PYZcb{} \PYZdl{} }\PY{l+s+s1}{\PYZsq{}}\PY{p}{,} \PY{n}{color}\PY{o}{=}\PY{l+s+s1}{\PYZsq{}}\PY{l+s+s1}{C0}\PY{l+s+s1}{\PYZsq{}}\PY{p}{)}
\PY{c+c1}{\PYZsh{}ax1.plot(w/(FSR), F\PYZus{}w\PYZus{}cc\PYZus{}modsq*100)}
\PY{n}{ax1}\PY{o}{.}\PY{n}{loglog}\PY{p}{(}\PY{n}{bode\PYZus{}amp}\PY{p}{(}\PY{n}{H\PYZus{}FPMI}\PY{p}{)}\PY{p}{,} \PY{n}{label}\PY{o}{=}\PY{l+s+s1}{\PYZsq{}}\PY{l+s+s1}{FPMI}\PY{l+s+s1}{\PYZsq{}}\PY{p}{,} \PY{n}{linewidth}\PY{o}{=}\PY{n}{line\PYZus{}width}\PY{p}{,}\PY{n}{color}\PY{o}{=}\PY{l+s+s1}{\PYZsq{}}\PY{l+s+s1}{C0}\PY{l+s+s1}{\PYZsq{}}\PY{p}{)}
\PY{c+c1}{\PYZsh{}ax1.loglog(w,H\PYZus{}MI\PYZus{}modsq, label= \PYZsq{}MICH\PYZsq{}, linewidth= 5)}
\PY{c+c1}{\PYZsh{}ax1.loglog(w,H\PYZus{}FPMI\PYZus{}LP\PYZus{}modsq*H\PYZus{}FPMI\PYZus{}modsq[0], label=\PYZsq{}FPMI LP\PYZsq{}, linewidth = 20.0, alpha=0.25,color=\PYZsq{}C2\PYZsq{})}
\PY{c+c1}{\PYZsh{}ax1.axvline (x=f\PYZus{}pole,ymin=1e\PYZhy{}13, color=\PYZsq{}red\PYZsq{}, linestyle=\PYZsq{}dotted\PYZsq{}, linewidth=3)}
\PY{n}{ax2} \PY{o}{=} \PY{n}{ax1}\PY{o}{.}\PY{n}{twinx}\PY{p}{(}\PY{p}{)}
\PY{n}{ax2}\PY{o}{.}\PY{n}{semilogx}\PY{p}{(}\PY{n}{nu}\PY{p}{,}\PY{n}{bode\PYZus{}ph}\PY{p}{(}\PY{n}{H\PYZus{}FPMI}\PY{p}{)}\PY{p}{,}\PY{l+s+s1}{\PYZsq{}}\PY{l+s+s1}{\PYZhy{}\PYZhy{}}\PY{l+s+s1}{\PYZsq{}}\PY{p}{,} \PY{n}{linewidth}\PY{o}{=}\PY{n}{line\PYZus{}width}\PY{p}{,} \PY{n}{color}\PY{o}{=}\PY{l+s+s1}{\PYZsq{}}\PY{l+s+s1}{C1}\PY{l+s+s1}{\PYZsq{}}\PY{p}{)}
\PY{c+c1}{\PYZsh{}ax2.semilogx(w,(180/np.pi)*np.arctan(np.imag(H\PYZus{}MI)/np.real(H\PYZus{}MI)), \PYZsq{}\PYZhy{}\PYZhy{}\PYZsq{})}
\PY{c+c1}{\PYZsh{}ax2.semilogx(w,(180/np.pi)*np.arctan(np.imag(H\PYZus{}FPMI\PYZus{}LP)/np.real(H\PYZus{}FPMI\PYZus{}LP)),linestyle=\PYZsq{}\PYZhy{}\PYZhy{}\PYZsq{}, linewidth=20.0,dashes=(4,10),alpha=.25, color=\PYZsq{}C2\PYZsq{})}
\PY{n}{plt}\PY{o}{.}\PY{n}{xlim}\PY{p}{(}\PY{p}{[}\PY{l+m+mi}{1}\PY{p}{,}\PY{l+m+mf}{1e5}\PY{p}{]}\PY{p}{)}
\PY{n}{plt}\PY{o}{.}\PY{n}{ylabel}\PY{p}{(}\PY{l+s+s1}{\PYZsq{}}\PY{l+s+s1}{phase [deg]}\PY{l+s+s1}{\PYZsq{}}\PY{p}{,} \PY{n}{color}\PY{o}{=}\PY{l+s+s1}{\PYZsq{}}\PY{l+s+s1}{C1}\PY{l+s+s1}{\PYZsq{}}\PY{p}{)}
\PY{c+c1}{\PYZsh{}fig.savefig(\PYZsq{}../figs/INTRO/fpmi\PYZus{}fr.pdf\PYZsq{}, dpi=300, bbox\PYZus{}inches=\PYZsq{}tight\PYZsq{})}
\end{Verbatim}
\end{tcolorbox}

            \begin{tcolorbox}[breakable, size=fbox, boxrule=.5pt, pad at break*=1mm, opacityfill=0]
\prompt{Out}{outcolor}{8}{\boxspacing}
\begin{Verbatim}[commandchars=\\\{\}]
Text(0, 0.5, 'phase [deg]')
\end{Verbatim}
\end{tcolorbox}
        
    \begin{center}
    \adjustimage{max size={0.9\linewidth}{0.9\paperheight}}{fpmi_fr_files/fpmi_fr_14_1.png}
    \end{center}
    { \hspace*{\fill} \\}
    
    \begin{tcolorbox}[breakable, size=fbox, boxrule=1pt, pad at break*=1mm,colback=cellbackground, colframe=cellborder]
\prompt{In}{incolor}{9}{\boxspacing}
\begin{Verbatim}[commandchars=\\\{\}]
\PY{n}{plt}\PY{o}{.}\PY{n}{loglog}\PY{p}{(}\PY{n}{nu}\PY{p}{,}\PY{n}{bode\PYZus{}amp}\PY{p}{(}\PY{n}{H\PYZus{}MICH}\PY{p}{)}\PY{p}{,} \PY{n}{label}\PY{o}{=} \PY{l+s+s1}{\PYZsq{}}\PY{l+s+s1}{MICH}\PY{l+s+s1}{\PYZsq{}}\PY{p}{,} \PY{n}{linewidth}\PY{o}{=} \PY{n}{line\PYZus{}width}\PY{p}{,} \PY{n}{alpha}\PY{o}{=}\PY{l+m+mf}{.5}\PY{p}{)}
\PY{n}{plt}\PY{o}{.}\PY{n}{loglog}\PY{p}{(}\PY{n}{nu}\PY{p}{,}\PY{n}{bode\PYZus{}amp}\PY{p}{(}\PY{n}{H\PYZus{}FPMI}\PY{p}{)}\PY{p}{,} \PY{n}{label}\PY{o}{=}\PY{l+s+s1}{\PYZsq{}}\PY{l+s+s1}{FPMI}\PY{l+s+s1}{\PYZsq{}}\PY{p}{,} \PY{n}{linewidth}\PY{o}{=}\PY{n}{line\PYZus{}width}\PY{p}{)}
\PY{c+c1}{\PYZsh{}plt.loglog(nu,bode\PYZus{}amp(H\PYZus{}FPMI\PYZus{}LP)*bode\PYZus{}amp(H\PYZus{}FPMI)[0], label=\PYZsq{}FPMI LP\PYZsq{}, linewidth = 20.0, alpha=0.25)}
\PY{n}{plt}\PY{o}{.}\PY{n}{axvline} \PY{p}{(}\PY{n}{x}\PY{o}{=}\PY{n}{f\PYZus{}pole}\PY{p}{,}\PY{n}{ymin}\PY{o}{=}\PY{l+m+mf}{1e\PYZhy{}11}\PY{p}{,} \PY{n}{color}\PY{o}{=}\PY{l+s+s1}{\PYZsq{}}\PY{l+s+s1}{red}\PY{l+s+s1}{\PYZsq{}}\PY{p}{,} \PY{n}{linestyle}\PY{o}{=}\PY{l+s+s1}{\PYZsq{}}\PY{l+s+s1}{dotted}\PY{l+s+s1}{\PYZsq{}}\PY{p}{,} \PY{n}{linewidth}\PY{o}{=}\PY{l+m+mf}{3.0}\PY{p}{)}
\PY{n}{plt}\PY{o}{.}\PY{n}{ylim}\PY{p}{(}\PY{p}{[}\PY{l+m+mf}{5e7}\PY{p}{,} \PY{l+m+mf}{5e14}\PY{p}{]}\PY{p}{)}
\PY{n}{plt}\PY{o}{.}\PY{n}{xlim}\PY{p}{(}\PY{p}{[}\PY{l+m+mf}{1e0}\PY{p}{,} \PY{l+m+mf}{1e5}\PY{p}{]}\PY{p}{)}
\PY{c+c1}{\PYZsh{}plt.grid(visible=True, which=\PYZsq{}minor\PYZsq{}, axis=\PYZsq{}y\PYZsq{})}
\PY{n}{plt}\PY{o}{.}\PY{n}{xlabel}\PY{p}{(}\PY{l+s+s1}{\PYZsq{}}\PY{l+s+s1}{frequency [Hz]}\PY{l+s+s1}{\PYZsq{}}\PY{p}{)}
\PY{n}{plt}\PY{o}{.}\PY{n}{ylabel}\PY{p}{(}\PY{l+s+s1}{\PYZsq{}}\PY{l+s+s1}{H(f) \PYZdl{}}\PY{l+s+s1}{\PYZbs{}}\PY{l+s+s1}{mathdefault}\PY{l+s+s1}{\PYZob{}}\PY{l+s+s1}{[W/m]\PYZcb{}\PYZdl{}}\PY{l+s+s1}{\PYZsq{}}\PY{p}{)}
\PY{n}{lgd}\PY{o}{=}\PY{n}{plt}\PY{o}{.}\PY{n}{legend}\PY{p}{(}\PY{p}{)}
\PY{n}{plt}\PY{o}{.}\PY{n}{savefig}\PY{p}{(}\PY{l+s+s1}{\PYZsq{}}\PY{l+s+s1}{../figs/INTRO/fpmi\PYZus{}fr.pdf}\PY{l+s+s1}{\PYZsq{}}\PY{p}{,} \PY{n}{dpi}\PY{o}{=}\PY{l+m+mi}{300}\PY{p}{,} \PY{n}{bbox\PYZus{}inches}\PY{o}{=}\PY{l+s+s1}{\PYZsq{}}\PY{l+s+s1}{tight}\PY{l+s+s1}{\PYZsq{}}\PY{p}{)}
\end{Verbatim}
\end{tcolorbox}

    \begin{center}
    \adjustimage{max size={0.9\linewidth}{0.9\paperheight}}{fpmi_fr_files/fpmi_fr_15_0.png}
    \end{center}
    { \hspace*{\fill} \\}
    
    You can clearly see that there is a clear increase in gain at lower
frequencies (below 5000 kHz) Doesn't exactly look like Kiwamu's but
close enough?

    \begin{tcolorbox}[breakable, size=fbox, boxrule=1pt, pad at break*=1mm,colback=cellbackground, colframe=cellborder]
\prompt{In}{incolor}{10}{\boxspacing}
\begin{Verbatim}[commandchars=\\\{\}]
\PY{n}{plt}\PY{o}{.}\PY{n}{semilogx}\PY{p}{(}\PY{n}{nu}\PY{p}{,}\PY{n}{bode\PYZus{}ph}\PY{p}{(}\PY{n}{H\PYZus{}MICH}\PY{p}{)}\PY{p}{,} \PY{l+s+s1}{\PYZsq{}}\PY{l+s+s1}{\PYZhy{}\PYZhy{}}\PY{l+s+s1}{\PYZsq{}}\PY{p}{,} \PY{n}{label}\PY{o}{=}\PY{l+s+s1}{\PYZsq{}}\PY{l+s+s1}{MICH}\PY{l+s+s1}{\PYZsq{}}\PY{p}{,} \PY{n}{linewidth}\PY{o}{=} \PY{n}{line\PYZus{}width}\PY{p}{,} \PY{n}{alpha}\PY{o}{=}\PY{l+m+mf}{.5}\PY{p}{)}
\PY{n}{plt}\PY{o}{.}\PY{n}{semilogx}\PY{p}{(}\PY{n}{nu}\PY{p}{,}\PY{n}{bode\PYZus{}ph}\PY{p}{(}\PY{n}{H\PYZus{}FPMI}\PY{p}{)}\PY{p}{,}\PY{l+s+s1}{\PYZsq{}}\PY{l+s+s1}{\PYZhy{}\PYZhy{}}\PY{l+s+s1}{\PYZsq{}}\PY{p}{,} \PY{n}{label}\PY{o}{=}\PY{l+s+s1}{\PYZsq{}}\PY{l+s+s1}{FPMI}\PY{l+s+s1}{\PYZsq{}}\PY{p}{,} \PY{n}{linewidth}\PY{o}{=} \PY{n}{line\PYZus{}width}\PY{p}{)}
\PY{c+c1}{\PYZsh{}plt.semilogx(nu,bode\PYZus{}ph(H\PYZus{}FPMI\PYZus{}LP),linestyle=\PYZsq{}\PYZhy{}\PYZhy{}\PYZsq{}, linewidth=3.0,dashes=(3,10))}
\PY{n}{plt}\PY{o}{.}\PY{n}{xlim}\PY{p}{(}\PY{p}{[}\PY{l+m+mi}{1}\PY{p}{,}\PY{l+m+mi}{100000}\PY{p}{]}\PY{p}{)}
\PY{n}{plt}\PY{o}{.}\PY{n}{ylabel}\PY{p}{(}\PY{l+s+s1}{\PYZsq{}}\PY{l+s+s1}{phase [deg]}\PY{l+s+s1}{\PYZsq{}}\PY{p}{)}
\PY{n}{plt}\PY{o}{.}\PY{n}{xlabel}\PY{p}{(}\PY{l+s+s1}{\PYZsq{}}\PY{l+s+s1}{Frequency [Hz]}\PY{l+s+s1}{\PYZsq{}} \PY{p}{)}
\PY{n}{lgd}\PY{o}{=}\PY{n}{plt}\PY{o}{.}\PY{n}{legend}\PY{p}{(}\PY{p}{)}
\end{Verbatim}
\end{tcolorbox}

    \begin{center}
    \adjustimage{max size={0.9\linewidth}{0.9\paperheight}}{fpmi_fr_files/fpmi_fr_17_0.png}
    \end{center}
    { \hspace*{\fill} \\}
    
    \begin{tcolorbox}[breakable, size=fbox, boxrule=1pt, pad at break*=1mm,colback=cellbackground, colframe=cellborder]
\prompt{In}{incolor}{11}{\boxspacing}
\begin{Verbatim}[commandchars=\\\{\}]
\PY{n}{Sh\PYZus{}noise} \PY{o}{=} \PY{n}{N\PYZus{}shot}\PY{p}{(}\PY{n}{OMEG}\PY{p}{,} \PY{n}{P\PYZus{}IN}\PY{p}{)}
\end{Verbatim}
\end{tcolorbox}

    \begin{tcolorbox}[breakable, size=fbox, boxrule=1pt, pad at break*=1mm,colback=cellbackground, colframe=cellborder]
\prompt{In}{incolor}{12}{\boxspacing}
\begin{Verbatim}[commandchars=\\\{\}]
\PY{n}{plt}\PY{o}{.}\PY{n}{loglog}\PY{p}{(}\PY{n}{nu}\PY{p}{,}\PY{n}{Sh\PYZus{}noise}\PY{o}{/}\PY{n}{bode\PYZus{}amp}\PY{p}{(}\PY{n}{H\PYZus{}MICH}\PY{p}{)}\PY{p}{,} \PY{n}{label}\PY{o}{=} \PY{l+s+s1}{\PYZsq{}}\PY{l+s+s1}{MICH}\PY{l+s+s1}{\PYZsq{}}\PY{p}{,} \PY{n}{linewidth}\PY{o}{=} \PY{n}{line\PYZus{}width}\PY{p}{,} \PY{n}{alpha}\PY{o}{=}\PY{l+m+mf}{.5}\PY{p}{)}
\PY{n}{plt}\PY{o}{.}\PY{n}{loglog}\PY{p}{(}\PY{n}{nu}\PY{p}{,}\PY{n}{Sh\PYZus{}noise}\PY{o}{/}\PY{n}{bode\PYZus{}amp}\PY{p}{(}\PY{n}{H\PYZus{}FPMI}\PY{p}{)}\PY{p}{,} \PY{n}{label}\PY{o}{=}\PY{l+s+s1}{\PYZsq{}}\PY{l+s+s1}{FPMI}\PY{l+s+s1}{\PYZsq{}}\PY{p}{,} \PY{n}{linewidth}\PY{o}{=}\PY{n}{line\PYZus{}width}\PY{p}{)}
\PY{c+c1}{\PYZsh{}plt.loglog(nu,Sh\PYZus{}noise/(bode\PYZus{}amp(H\PYZus{}FPMI\PYZus{}LP)*bode\PYZus{}amp(H\PYZus{}FPMI)[0]), label=\PYZsq{}FPMI LP\PYZsq{}, linewidth = 20.0, alpha=0.25)}
\PY{c+c1}{\PYZsh{}plt.axvline (x=f\PYZus{}pole,ymin=1e\PYZhy{}11, color=\PYZsq{}red\PYZsq{}, linestyle=\PYZsq{}dotted\PYZsq{}, linewidth=3)}
\PY{n}{plt}\PY{o}{.}\PY{n}{ylim}\PY{p}{(}\PY{p}{[}\PY{l+m+mf}{1e\PYZhy{}23}\PY{p}{,} \PY{l+m+mf}{1e\PYZhy{}16}\PY{p}{]}\PY{p}{)}
\PY{n}{plt}\PY{o}{.}\PY{n}{xlim}\PY{p}{(}\PY{p}{[}\PY{l+m+mf}{1e0}\PY{p}{,} \PY{l+m+mf}{1e5}\PY{p}{]}\PY{p}{)}
\PY{n}{plt}\PY{o}{.}\PY{n}{xlabel}\PY{p}{(}\PY{l+s+s1}{\PYZsq{}}\PY{l+s+s1}{frequency [Hz]}\PY{l+s+s1}{\PYZsq{}}\PY{p}{)}
\PY{n}{plt}\PY{o}{.}\PY{n}{ylabel}\PY{p}{(}\PY{l+s+s1}{\PYZsq{}}\PY{l+s+s1}{H(f) \PYZdl{}}\PY{l+s+s1}{\PYZbs{}}\PY{l+s+s1}{mathdefault}\PY{l+s+s1}{\PYZob{}}\PY{l+s+s1}{[1/}\PY{l+s+s1}{\PYZbs{}}\PY{l+s+s1}{sqrt}\PY{l+s+s1}{\PYZob{}}\PY{l+s+s1}{\PYZbs{}}\PY{l+s+s1}{mathdefault}\PY{l+s+si}{\PYZob{}Hz\PYZcb{}}\PY{l+s+s1}{\PYZcb{}]\PYZcb{}\PYZdl{}}\PY{l+s+s1}{\PYZsq{}}\PY{p}{)}
\PY{n}{lgd}\PY{o}{=}\PY{n}{plt}\PY{o}{.}\PY{n}{legend}\PY{p}{(}\PY{p}{)}
\PY{n}{fig}\PY{o}{.}\PY{n}{savefig}\PY{p}{(}\PY{l+s+s1}{\PYZsq{}}\PY{l+s+s1}{../figs/INTRO/fpmi\PYZus{}sensi.pdf}\PY{l+s+s1}{\PYZsq{}}\PY{p}{,} \PY{n}{dpi}\PY{o}{=}\PY{l+m+mi}{300}\PY{p}{,} \PY{n}{bbox\PYZus{}inches}\PY{o}{=}\PY{l+s+s1}{\PYZsq{}}\PY{l+s+s1}{tight}\PY{l+s+s1}{\PYZsq{}}\PY{p}{)}
\end{Verbatim}
\end{tcolorbox}

    \begin{Verbatim}[commandchars=\\\{\}]
/Users/daniel\_vander-hyde/anaconda3/envs/jupy/lib/python3.7/site-
packages/IPython/core/pylabtools.py:151: UserWarning: Creating legend with
loc="best" can be slow with large amounts of data.
  fig.canvas.print\_figure(bytes\_io, **kw)
    \end{Verbatim}

    \begin{center}
    \adjustimage{max size={0.9\linewidth}{0.9\paperheight}}{fpmi_fr_files/fpmi_fr_19_1.png}
    \end{center}
    { \hspace*{\fill} \\}
    
    \hypertarget{heavily-heavily-inspired-by-kiwamus-thesis-chapter-on-this-subject-httpsgwic.ligo.orgthesisprize2012izumi-thesis.pdf}{%
\subsubsection{*Heavily HEAVILY inspired by Kiwamu's thesis chapter on
this subject
(https://gwic.ligo.org/thesisprize/2012/izumi-thesis.pdf)}\label{heavily-heavily-inspired-by-kiwamus-thesis-chapter-on-this-subject-httpsgwic.ligo.orgthesisprize2012izumi-thesis.pdf}}

    Next it might be useful to see the frequency dependence of just the FP
alone

    \hypertarget{vs.-the-delay-line}{%
\subsection{Vs. the delay line}\label{vs.-the-delay-line}}

    \hypertarget{expressing-part-of-the-transfer-function-in-terms-of-cavity-storage-time-tau_mathrmstor.-this-helps-us-create-a-concept-of-a-equivalent-length-or-total-elapsed-travel-time-of-a-given-phasefront-in-the-cavity.}{%
\paragraph{\texorpdfstring{Expressing part of the transfer function in
terms of cavity storage time \(\tau_\mathrm{stor}\). This helps us
create a concept of a ``equivalent length'' or total elapsed travel time
of a given phasefront in the
cavity.}{Expressing part of the transfer function in terms of cavity storage time \textbackslash tau\_\textbackslash mathrm\{stor\}. This helps us create a concept of a ``equivalent length'' or total elapsed travel time of a given phasefront in the cavity.}}\label{expressing-part-of-the-transfer-function-in-terms-of-cavity-storage-time-tau_mathrmstor.-this-helps-us-create-a-concept-of-a-equivalent-length-or-total-elapsed-travel-time-of-a-given-phasefront-in-the-cavity.}}

    Revisiting our Michelson response:
\[ H_{\mathrm{MI}}(f_\mathrm{gw}) = \tau_\mathrm{stor}\frac{2 \pi c}{\lambda}e^{-i \pi f_\mathrm{gw} \tau_\mathrm{stor}} \frac{\mathrm{sin}(f_\mathrm{gw} \tau_\mathrm{stor})}{f_\mathrm{gw}\tau_\mathrm{stor}} \]

    \begin{tcolorbox}[breakable, size=fbox, boxrule=1pt, pad at break*=1mm,colback=cellbackground, colframe=cellborder]
\prompt{In}{incolor}{13}{\boxspacing}
\begin{Verbatim}[commandchars=\\\{\}]
\PY{c+c1}{\PYZsh{} Delay line storage time}
\PY{n}{N\PYZus{}rt} \PY{o}{=} \PY{l+m+mi}{10}
\PY{n}{tau\PYZus{}stor\PYZus{}dl} \PY{o}{=} \PY{n}{N\PYZus{}rt}\PY{o}{*}\PY{p}{(}\PY{l+m+mi}{2}\PY{o}{*}\PY{n}{L}\PY{p}{)}\PY{o}{/}\PY{n}{cee}
\end{Verbatim}
\end{tcolorbox}

    \begin{tcolorbox}[breakable, size=fbox, boxrule=1pt, pad at break*=1mm,colback=cellbackground, colframe=cellborder]
\prompt{In}{incolor}{24}{\boxspacing}
\begin{Verbatim}[commandchars=\\\{\}]
\PY{c+c1}{\PYZsh{} Fabry perot storage tim }
\PY{c+c1}{\PYZsh{}Finn = np.pi*np.sqrt(r\PYZus{}1*r\PYZus{}2)/(1\PYZhy{}(r\PYZus{}1*r\PYZus{}2))}
\PY{n}{Finn} \PY{o}{=} \PY{l+m+mi}{3}
\PY{n}{tau\PYZus{}stor\PYZus{}fp} \PY{o}{=} \PY{p}{(}\PY{n}{L}\PY{o}{*}\PY{n}{Finn}\PY{p}{)}\PY{o}{/}\PY{p}{(}\PY{n}{cee}\PY{o}{*}\PY{n}{np}\PY{o}{.}\PY{n}{pi}\PY{p}{)}
\end{Verbatim}
\end{tcolorbox}

    \begin{tcolorbox}[breakable, size=fbox, boxrule=1pt, pad at break*=1mm,colback=cellbackground, colframe=cellborder]
\prompt{In}{incolor}{15}{\boxspacing}
\begin{Verbatim}[commandchars=\\\{\}]
\PY{k}{def} \PY{n+nf}{mich\PYZus{}freq\PYZus{}resp\PYZus{}}\PY{p}{(}\PY{n}{freq}\PY{p}{,} \PY{n}{L}\PY{p}{,} \PY{n}{t\PYZus{}s}\PY{p}{,} \PY{n}{lambd}\PY{p}{,} \PY{n}{h0}\PY{p}{)}\PY{p}{:}
    \PY{k}{return} \PY{p}{(}\PY{p}{(}\PY{n}{h0}\PY{o}{*}\PY{n}{t\PYZus{}s}\PY{o}{*}\PY{l+m+mf}{2.0}\PY{o}{*}\PY{n}{np}\PY{o}{.}\PY{n}{pi}\PY{o}{*}\PY{n}{cee}\PY{p}{)}\PY{o}{/}\PY{n}{lambd}\PY{p}{)}\PY{o}{*}\PY{n}{np}\PY{o}{.}\PY{n}{exp}\PY{p}{(}\PY{p}{(}\PY{o}{\PYZhy{}}\PY{l+m+mi}{1}\PY{n}{j}\PY{o}{*}\PY{n}{np}\PY{o}{.}\PY{n}{pi}\PY{o}{*}\PY{l+m+mf}{2.0}\PY{o}{*}\PY{n}{np}\PY{o}{.}\PY{n}{pi}\PY{o}{*}\PY{n}{freq}\PY{p}{)}\PY{o}{/}\PY{n}{cee}\PY{p}{)}\PY{o}{*}\PY{n}{np}\PY{o}{.}\PY{n}{sin}\PY{p}{(}\PY{p}{(}\PY{n}{L}\PY{o}{*}\PY{l+m+mf}{2.0}\PY{o}{*}\PY{n}{np}\PY{o}{.}\PY{n}{pi}\PY{o}{*}\PY{n}{freq}\PY{p}{)}\PY{o}{/}\PY{n}{cee}\PY{p}{)}\PY{o}{/}\PY{p}{(}\PY{n}{L}\PY{o}{*}\PY{l+m+mf}{2.0}\PY{o}{*}\PY{n}{np}\PY{o}{.}\PY{n}{pi}\PY{o}{*}\PY{n}{freq}\PY{p}{)}
\end{Verbatim}
\end{tcolorbox}

    \begin{tcolorbox}[breakable, size=fbox, boxrule=1pt, pad at break*=1mm,colback=cellbackground, colframe=cellborder]
\prompt{In}{incolor}{ }{\boxspacing}
\begin{Verbatim}[commandchars=\\\{\}]
\PY{k}{def} \PY{n+nf}{mich\PYZus{}freq\PYZus{}resp\PYZus{}}\PY{p}{(}
\end{Verbatim}
\end{tcolorbox}

    \begin{tcolorbox}[breakable, size=fbox, boxrule=1pt, pad at break*=1mm,colback=cellbackground, colframe=cellborder]
\prompt{In}{incolor}{17}{\boxspacing}
\begin{Verbatim}[commandchars=\\\{\}]
\PY{n}{H\PYZus{}MI\PYZus{}sto} \PY{o}{=} \PY{n}{mich\PYZus{}freq\PYZus{}resp\PYZus{}}\PY{p}{(}\PY{n}{nu}\PY{p}{,} \PY{n}{L}\PY{p}{,} \PY{n}{tau\PYZus{}stor\PYZus{}dl}\PY{p}{,} \PY{l+m+mf}{1064e\PYZhy{}9}\PY{p}{,} \PY{n}{h\PYZus{}0}\PY{p}{)}
\end{Verbatim}
\end{tcolorbox}

    \begin{tcolorbox}[breakable, size=fbox, boxrule=1pt, pad at break*=1mm,colback=cellbackground, colframe=cellborder]
\prompt{In}{incolor}{25}{\boxspacing}
\begin{Verbatim}[commandchars=\\\{\}]
\PY{n}{H\PYZus{}FPMI\PYZus{}sto} \PY{o}{=} \PY{n}{mich\PYZus{}freq\PYZus{}resp\PYZus{}}\PY{p}{(}\PY{n}{nu}\PY{p}{,} \PY{n}{L}\PY{p}{,} \PY{n}{tau\PYZus{}stor\PYZus{}fp}\PY{p}{,} \PY{l+m+mf}{1064e\PYZhy{}9}\PY{p}{,} \PY{n}{h\PYZus{}0}\PY{p}{)}
\end{Verbatim}
\end{tcolorbox}

    \begin{tcolorbox}[breakable, size=fbox, boxrule=1pt, pad at break*=1mm,colback=cellbackground, colframe=cellborder]
\prompt{In}{incolor}{26}{\boxspacing}
\begin{Verbatim}[commandchars=\\\{\}]
\PY{n}{H\PYZus{}MI\PYZus{}sto\PYZus{}modsq} \PY{o}{=} \PY{n}{np}\PY{o}{.}\PY{n}{real}\PY{p}{(}\PY{n}{H\PYZus{}MI\PYZus{}sto}\PY{p}{)}\PY{o}{*}\PY{o}{*}\PY{l+m+mi}{2} \PY{o}{+} \PY{n}{np}\PY{o}{.}\PY{n}{imag}\PY{p}{(}\PY{n}{H\PYZus{}MI\PYZus{}sto}\PY{p}{)}\PY{o}{*}\PY{o}{*}\PY{l+m+mi}{2}
\PY{n}{H\PYZus{}FPMI\PYZus{}sto\PYZus{}modsq} \PY{o}{=} \PY{n}{np}\PY{o}{.}\PY{n}{real}\PY{p}{(}\PY{n}{H\PYZus{}FPMI\PYZus{}sto}\PY{p}{)}\PY{o}{*}\PY{o}{*}\PY{l+m+mi}{2} \PY{o}{+} \PY{n}{np}\PY{o}{.}\PY{n}{imag}\PY{p}{(}\PY{n}{H\PYZus{}FPMI\PYZus{}sto}\PY{p}{)}\PY{o}{*}\PY{o}{*}\PY{l+m+mi}{2}
\end{Verbatim}
\end{tcolorbox}

    \begin{tcolorbox}[breakable, size=fbox, boxrule=1pt, pad at break*=1mm,colback=cellbackground, colframe=cellborder]
\prompt{In}{incolor}{27}{\boxspacing}
\begin{Verbatim}[commandchars=\\\{\}]
\PY{n}{H\PYZus{}MI\PYZus{}sto\PYZus{}ph} \PY{o}{=} \PY{p}{(}\PY{l+m+mi}{180}\PY{o}{/}\PY{n}{np}\PY{o}{.}\PY{n}{pi}\PY{p}{)}\PY{o}{*}\PY{n}{np}\PY{o}{.}\PY{n}{arctan}\PY{p}{(}\PY{n}{np}\PY{o}{.}\PY{n}{imag}\PY{p}{(}\PY{n}{H\PYZus{}MI\PYZus{}sto}\PY{p}{)}\PY{o}{/}\PY{n}{np}\PY{o}{.}\PY{n}{real}\PY{p}{(}\PY{n}{H\PYZus{}MI\PYZus{}sto}\PY{p}{)}\PY{p}{)}
\PY{n}{H\PYZus{}FPMI\PYZus{}sto\PYZus{}ph} \PY{o}{=} \PY{p}{(}\PY{l+m+mi}{180}\PY{o}{/}\PY{n}{np}\PY{o}{.}\PY{n}{pi}\PY{p}{)}\PY{o}{*}\PY{n}{np}\PY{o}{.}\PY{n}{arctan}\PY{p}{(}\PY{n}{np}\PY{o}{.}\PY{n}{imag}\PY{p}{(}\PY{n}{H\PYZus{}FPMI\PYZus{}sto}\PY{p}{)}\PY{o}{/}\PY{n}{np}\PY{o}{.}\PY{n}{real}\PY{p}{(}\PY{n}{H\PYZus{}FPMI\PYZus{}sto}\PY{p}{)}\PY{p}{)}
\end{Verbatim}
\end{tcolorbox}

    \begin{tcolorbox}[breakable, size=fbox, boxrule=1pt, pad at break*=1mm,colback=cellbackground, colframe=cellborder]
\prompt{In}{incolor}{28}{\boxspacing}
\begin{Verbatim}[commandchars=\\\{\}]
\PY{n}{plt}\PY{o}{.}\PY{n}{loglog}\PY{p}{(}\PY{n}{nu}\PY{p}{,} \PY{n}{H\PYZus{}MI\PYZus{}sto\PYZus{}modsq}\PY{p}{)}
\PY{n}{plt}\PY{o}{.}\PY{n}{loglog}\PY{p}{(}\PY{n}{nu}\PY{p}{,} \PY{n}{H\PYZus{}FPMI\PYZus{}sto\PYZus{}modsq}\PY{p}{)}
\end{Verbatim}
\end{tcolorbox}

            \begin{tcolorbox}[breakable, size=fbox, boxrule=.5pt, pad at break*=1mm, opacityfill=0]
\prompt{Out}{outcolor}{28}{\boxspacing}
\begin{Verbatim}[commandchars=\\\{\}]
[<matplotlib.lines.Line2D at 0x177cc23d0>]
\end{Verbatim}
\end{tcolorbox}
        
    \begin{center}
    \adjustimage{max size={0.9\linewidth}{0.9\paperheight}}{fpmi_fr_files/fpmi_fr_33_1.png}
    \end{center}
    { \hspace*{\fill} \\}
    
    \begin{tcolorbox}[breakable, size=fbox, boxrule=1pt, pad at break*=1mm,colback=cellbackground, colframe=cellborder]
\prompt{In}{incolor}{20}{\boxspacing}
\begin{Verbatim}[commandchars=\\\{\}]
\PY{n}{plt}\PY{o}{.}\PY{n}{semilogx}\PY{p}{(}\PY{n}{nu}\PY{p}{,} \PY{n}{H\PYZus{}MI\PYZus{}sto\PYZus{}ph}\PY{p}{)}
\end{Verbatim}
\end{tcolorbox}

            \begin{tcolorbox}[breakable, size=fbox, boxrule=.5pt, pad at break*=1mm, opacityfill=0]
\prompt{Out}{outcolor}{20}{\boxspacing}
\begin{Verbatim}[commandchars=\\\{\}]
[<matplotlib.lines.Line2D at 0x2a0c078b0>]
\end{Verbatim}
\end{tcolorbox}
        
    \begin{center}
    \adjustimage{max size={0.9\linewidth}{0.9\paperheight}}{fpmi_fr_files/fpmi_fr_34_1.png}
    \end{center}
    { \hspace*{\fill} \\}
    
    \begin{tcolorbox}[breakable, size=fbox, boxrule=1pt, pad at break*=1mm,colback=cellbackground, colframe=cellborder]
\prompt{In}{incolor}{ }{\boxspacing}
\begin{Verbatim}[commandchars=\\\{\}]

\end{Verbatim}
\end{tcolorbox}


    % Add a bibliography block to the postdoc
    
    
    
\end{document}
