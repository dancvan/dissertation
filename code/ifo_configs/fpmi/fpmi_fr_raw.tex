\begin{lstlisting}[frame=single, language=Python]
import numpy as np 
import matplotlib.pyplot as plt
import scipy.signal as sig
import os
import sys
sys.path.insert(0,'../')
plt_style_dir = '../../stash/'
fig_exp_dir = '../../../figs/'
from ifo_configs import mich_freq_resp as MICH
from ifo_configs import fpmi_freq_resp as FPMI
from ifo_configs import N_shot, bode_amp, bode_ph
if os.path.isdir(plt_style_dir) == True:
    plt.style.use(plt_style_dir + 'ppt2latexsubfig.mplstyle')
plt.rcParams["font.family"] = "Times New Roman"
line_width=7.5
\end{lstlisting}

Let's start with the simple Fabry Perót cavity. The following are
equations that characterize the circulating and reflected fields (both
critical to measuring the phase response of the FP cavity to GWs):

\[
E(t) = t_1 E_{in} + r_1 r_2 E(t - 2T) e^{-i \Delta \phi(t)} 
\]

\[
E_r(t) = -r_1 E_{in} + t_1 r_2 E(t - 2T) e^{-i \Delta \phi(t)}
\]

\(T = L/c\) is the time it takes light to reach the end of the cavity
and \(\Delta \phi(t)\) is the phase rotation.

We can define the static phase rotation (no GW passing through) as :
\[\Delta \phi = 2kL = 4 \pi L /\lambda_{opt}  \]

And if L is tuned just right \(2kL = 2 \pi n\) so the cavity is just
tuned for resonance

If we put a gravitational wave in the mix we redefine this phase
rotation as such that:
\[\Delta \phi =  \frac{\omega_0}{2} \int_{t-\frac{2L}{c}}^{t} h(t')dt' \]

This assumes that the static phase rotation satisfies
\(2\omega_0L/c = 2 \pi n\). Which is the same thing that we said above
but with different symbols (because we're fancy ;D )

Say that we have something that does throw the cavity slightly off
resonance.. doesn't have to be a gravitational wave\ldots{} but that's
what we hope for. ANYWAY\ldots{}

If the \(\Delta \phi\) becomes such that the cavity is thrown off
resonance we get a time dependent intra-cavity field:

\[ E(t) = \bar{E} + \delta E(t) \]

and if the phase rotation (\(\Delta \phi\)) is super small\ldots{} which
is pretty much guaranteed with gravy waves, we can say:

\[ e^{i\Delta \phi} = 1- i \Delta \phi \]

Using equations \ref{eq7} and \ref{eq8} in \ref{eq3} we get:

\[ \bar{E} + \delta E(t) = t_1 E_{in} -r_1r_2\bar{E} + r_1r_2 \delta E(t-2T) - ir_1r_2\bar{E}\Delta \phi(t)) \]

We can parse this into time dependent and time independent terms:

\[ \bar{E} = t_1 E_{in} -r_1r_2\bar{E} \]

\[ \delta E(t) = r_1r_2 \delta E(t-2T) - ir_1r_2\bar{E}\Delta \phi(t) \]

Since the time dependent phase information is encoded in \ref{eq11} we
will take the laplace transform of this equation to yield:

\[\delta E(s) = -i \frac{r_1r_2 \bar{E}}{1-r_1r_2e^{-2sT}} \Delta \phi(s)\]

\textbf{YAS!} we are now one step closer to getting a useful expression
for the phase response. But again.. what does this last equation mean?
That last equation is how the change in the electric field directly
relates to a small perturbation in phase (which could be either a small
change in laser frequency or length modulation)

Now.. we're not done yet because that last expression does not tell us
the entire story yet.. we want to see how this effects the phase
differential with the \textbf{reflected} electric field.

To do this.. we have to combine equations \ref{eq3} and \ref{eq4}. (an
easy way to do this is to get rid of the $ r_2 E(t - 2T) e^{-i \Delta \phi(t)} $ ) :

\[ E_r(t) = \frac{t_1}{r_1}E(t) - \frac{t_1^2 + r_2^2}{r_1} E_{in}\]

if the cavity is unperturbed:

\[ \bar{E}_r = \bigg(\frac{r_2(r_1^2 + t_1^2) - r_1}{t_1} \bigg) \bar{E} \]

and if we perturb the cavity we see that the change in the intra-cavity
field is directly related to the change in the reflected field:

\[ \Delta \phi_r(s) \equiv \frac{\delta E(s)}{\bar{E}} = \frac{t_1^2r_2}{(t_1^2 + r_1^2)r_2 -r_1} \frac{\Delta \phi(s)}{1-r_1r_2e^{-2sT}}\]

This implies that there is an additional frequency dependent factor in
your phase shift and this translates into your FPMI transfer function
as:

\[ H_{FPMI}(\omega_g) = \frac{2 \Delta \phi_r(\omega_g)}{h(\omega_g)} =  \frac{t_1^2r_2}{(t_1^2 + r_1^2)r_2 -r_1} \frac{H_{\mathrm{MI}}(\omega_g, L)}{1-r_1r_2e^{-2i \omega_g L /c }}  \]

Whew\ldots. that was a lot\ldots. now let's code it up Since we can
seperate the calculation into two.. I'm going to parse out the
calculation between the constant Fabry Perót term and the term with the
frequency dependence. But first, lets set up our parameters for our
FPMI:

\begin{lstlisting}[frame=single, language=Python]
# Some parameters
cee = np.float64(299792458)
OMEG = np.float64(2*np.pi*cee/(1064.0*1e-9))
L = np.float64(4000.0)
nu = np.arange(1, 1000000, 1)
nat_nu = [np.float64(i*2*np.pi) for i in nu]
h_0 = np.float64(1)

PHI_0 = np.pi/2 #[rad]
P_IN = 25

T_1 = .014
#T_1 = 25e-6 
T_2 = 50e-6
R_1 = 1-T_1
R_2 = 1-T_2

t_1 = T_1**.5
r_1 = R_1**.5
r_2 = R_2**.5
\end{lstlisting}

Now we can compute:
\[ H_{FPMI}(\omega_g) =  \frac{t_1^2r_2}{(t_1^2 + r_1^2)r_2 -r_1}\cdot \frac{H_{\mathrm{MI}}(\omega_g, L)}{1-r_1r_2e^{-2i \omega_g L /c }}  \]

\begin{lstlisting}[frame=single, language=Python]
H_FPMI = FPMI(nu, r_1, t_1, r_2, L, PHI_0, P_IN, OMEG)
\end{lstlisting}

We estimate the FP's pole frequency
\[  1 - r_1 r_2 e^{-2i \omega_g L / c} = 0 \] therefore when:
\[ e^{-i \omega_g L / c} = \frac{1}{\sqrt{r_1 r_2}} \] we acquire the
pole frequency \(\omega_\mathrm{pole}\) as indicated in the low pass
\[ f_\mathrm{pole} = \frac{1}{4\pi \tau_{s}} =  \frac{c}{4 \pi L} \frac{1- r_1 r_2}{\sqrt{r_1 r_2}} = \frac{\nu_\mathrm{FSR}}{2 \pi} \frac{1- r_1 r_2}{\sqrt{r_1 r_2}} = \frac{\nu_\mathrm{FSR}}{\mathcal{F}} \]

ALso, understanding that the cavity Finesse can be defined as

\[ \mathcal{F} = \frac{\pi \sqrt{r_i r_e}}{1- r_i r_e} \]

we also can invert for a high value of finesse $ \mathcal{F} \textgreater\textgreater{} \pi $:

\[ r_i r_e \approx 1 - \frac{\pi}{\mathcal{F}} \]

\begin{lstlisting}[frame=single, language=Python]
f_pole = 1/(((4*np.pi*L)*np.sqrt(r_1*r_2))/(cee*(1-r_1*r_2)))
def fpmi_lp(freq, cav_pole):
    return 1/(1 + 1j*(freq/cav_pole))#*np.exp(1j*freq/cav_pole))
H_FPMI_LP = fpmi_lp(nu, f_pole)
\end{lstlisting}

Might as well compare it to our Michelson response:
\[ H_{\mathrm{MI}}(\omega_g) = \frac{2 L \Omega}{c}e^{-i L \omega / c} \frac{\mathrm{sin}(L \omega /c)}{L \omega /c} \]

\begin{lstlisting}[frame=single, language=Python]
H_MICH = MICH(nu, L, PHI_0, P_IN, OMEG)
\end{lstlisting}

\begin{lstlisting}[frame=single, language=Python]
fig, ax1 = plt.subplots()
ax1.set_xlabel('frequency [Hz]')
ax1.set_ylabel('H$_\mathdefault{FPMI} \; \mathdefault{ [W / m] } $ ', color='C0')
#ax1.plot(w/(FSR), F_w_cc_modsq*100)
ax1.loglog(bode_amp(H_FPMI), label='FPMI', linewidth=line_width,color='C0')
#ax1.loglog(w,H_MI_modsq, label= 'MICH', linewidth= 5)
#ax1.loglog(w,H_FPMI_LP_modsq*H_FPMI_modsq[0], label='FPMI LP', linewidth = 20.0, alpha=0.25,color='C2')
#ax1.axvline (x=f_pole,ymin=1e-13, color='red', linestyle='dotted', linewidth=3)
ax2 = ax1.twinx()
ax2.semilogx(nu,bode_ph(H_FPMI),'--', linewidth=line_width, color='C1')
#ax2.semilogx(w,(180/np.pi)*np.arctan(np.imag(H_MI)/np.real(H_MI)), '--')
#ax2.semilogx(w,(180/np.pi)*np.arctan(np.imag(H_FPMI_LP)/np.real(H_FPMI_LP)),linestyle='--', linewidth=20.0,dashes=(4,10),alpha=.25, color='C2')
plt.xlim([1,1e5])
plt.ylabel('phase [deg]', color='C1')
#fig.savefig('../figs/INTRO/fpmi_fr.pdf', dpi=300, bbox_inches='tight')
\end{lstlisting}

\begin{lstlisting}
Text(0, 0.5, 'phase [deg]')
\end{lstlisting}

\begin{lstlisting}[frame=single, language=Python]
plt.loglog(nu,bode_amp(H_MICH), label= 'MICH', linewidth= line_width, alpha=.5)
plt.loglog(nu,bode_amp(H_FPMI), label='FPMI', linewidth=line_width)
#plt.loglog(nu,bode_amp(H_FPMI_LP)*bode_amp(H_FPMI)[0], label='FPMI LP', linewidth = 20.0, alpha=0.25)
plt.axvline (x=f_pole,ymin=1e-11, color='red', linestyle='dotted', linewidth=3.0)
plt.ylim([5e7, 5e14])
plt.xlim([1e0, 1e5])
#plt.grid(visible=True, which='minor', axis='y')
plt.xlabel('frequency [Hz]')
plt.ylabel('H(f) $\mathdefault{[W/m]}$')
lgd=plt.legend()
plt.savefig('../figs/INTRO/fpmi_fr.pdf', dpi=300, bbox_inches='tight')
\end{lstlisting}

You can clearly see that there is a clear increase in gain at lower
frequencies (below 5000 kHz) Doesn't exactly look like Kiwamu's but
close enough?

\begin{lstlisting}[frame=single, language=Python]
plt.semilogx(nu,bode_ph(H_MICH), '--', label='MICH', linewidth= line_width, alpha=.5)
plt.semilogx(nu,bode_ph(H_FPMI),'--', label='FPMI', linewidth= line_width)
#plt.semilogx(nu,bode_ph(H_FPMI_LP),linestyle='--', linewidth=3.0,dashes=(3,10))
plt.xlim([1,100000])
plt.ylabel('phase [deg]')
plt.xlabel('Frequency [Hz]' )
lgd=plt.legend()
\end{lstlisting}

\begin{lstlisting}[frame=single, language=Python]
Sh_noise = N_shot(OMEG, P_IN)
\end{lstlisting}

\begin{lstlisting}[frame=single, language=Python]
plt.loglog(nu,Sh_noise/bode_amp(H_MICH), label= 'MICH', linewidth= line_width, alpha=.5)
plt.loglog(nu,Sh_noise/bode_amp(H_FPMI), label='FPMI', linewidth=line_width)
#plt.loglog(nu,Sh_noise/(bode_amp(H_FPMI_LP)*bode_amp(H_FPMI)[0]), label='FPMI LP', linewidth = 20.0, alpha=0.25)
#plt.axvline (x=f_pole,ymin=1e-11, color='red', linestyle='dotted', linewidth=3)
plt.ylim([1e-23, 1e-16])
plt.xlim([1e0, 1e5])
plt.xlabel('frequency [Hz]')
plt.ylabel('H(f) $\mathdefault{[1/\sqrt{\mathdefault{Hz}}]}$')
lgd=plt.legend()
fig.savefig('../figs/INTRO/fpmi_sensi.pdf', dpi=300, bbox_inches='tight')
\end{lstlisting}

\hypertarget{heavily-heavily-inspired-by-kiwamus-thesis-chapter-on-this-subject-httpsgwic.ligo.orgthesisprize2012izumi-thesis.pdf}{%
\subsubsection{*Heavily HEAVILY inspired by Kiwamu's thesis chapter on
this subject
(https://gwic.ligo.org/thesisprize/2012/izumi-thesis.pdf)}\label{heavily-heavily-inspired-by-kiwamus-thesis-chapter-on-this-subject-httpsgwic.ligo.orgthesisprize2012izumi-thesis.pdf}}
