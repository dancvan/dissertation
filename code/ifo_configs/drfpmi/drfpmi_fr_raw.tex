\begin{lstlisting}[frame=single, language=Python]
import numpy as np 
import matplotlib.pyplot as plt
import scipy.signal as sig
import os 
import sys
sys.path.insert(0,'../')
plt_style_dir = '../../stash/'
fig_exp_dir = '../../../figs/'
import ifo_configs as ifco
if os.path.isdir(plt_style_dir) == True:
    plt.style.use(plt_style_dir + 'ppt2latexsubfig.mplstyle')
plt.rcParams["font.family"] = "serif"
plt.rcParams["font.serif"] = ["Times New Roman"] + plt.rcParams["font.serif"]
line_width=7.5
\end{lstlisting}

Up to this point we can understand how the FPMI repsonse function works:
\[ H_{FPMI}(\omega_g) = \frac{2 \Delta \phi_r(\omega_g)}{h(\omega_g)} =  \frac{t_1^2r_2}{(t_1^2 + r_1^2)r_2 -r_1} \frac{H_{\mathrm{MI}}(\omega_g, L)}{1-r_1r_2e^{-2i \omega_g L /c }}  \]

\begin{lstlisting}[frame=single, language=Python]
# Some parameters
cee = np.float64(299792458)
OMEG = np.float64(2*np.pi*cee/(1064.0*1e-9))
L = np.float64(4000.0)
nu = np.arange(1, 1000000, 1)
nat_nu = [np.float64(i*2*np.pi) for i in nu]
h_0 = np.float64(1)

T_1 = .014
#T_1 = 25e-6 
T_2 = 50e-6
R_1 = 1-T_1
R_2 = 1-T_2

t_1 = T_1**.5
r_1 = R_1**.5
r_2 = R_2**.5 

PHI_0 = np.pi/2 
P_IN = 25
\end{lstlisting}

\subsection{POWER RECYCLING}

With all the power going to the symmetric port, the nominal operating
state of the FPMI involves a significant amount of dumped / wasted
power. Placing a mirror at the symmetric port can allow that power to be
recycled. Though considerations must be made to maximize the amount of
recycling gain you can acquire with your GW detector. This is dependent
on the placement of the power recycling mirror (PRM) and its
reflectivity, transmission, and loss coefficients.

But first, the field at the symmetric port:

\[E_\mathrm{SYM} = \frac{E_i}{2}e^{2ikl}(r_\mathrm{FP,X} + r_\mathrm{FP,Y}) \]

This is realized through observing the circulating power between the PRM
and the short Michelson:

\[ E_\mathrm{PRC} = \frac{t_\mathrm{PRM}}{1- r_\mathrm{PRM} r_\mathrm{FPMI} e^{2ik (L_\mathrm{PRC2BS} + L_\mathrm{SMICH})}}E_\mathrm{in} \]

Where:

\[ L_\mathrm{SMICH} = l_x + l_y \]

Now let's observe the cavity reflection parameter:

\[ r_\mathrm{FP} = -r_1 + \frac{t_1^2 r_2 e^{i2kL}}{1-r_1 r_2 e^{i2kL}} = -\frac{\mathcal{F}}{\pi} \Big[-\Big(\frac{r_1}{r_2} \Big)^{1/2} + \Big(\frac{r_2}{r_1}\Big)^{1/2} (r_1^2 + t_1^2) \Big]\]
But with loss considerations:

\[ r_\mathrm{FP} = -r_1 + \frac{t_1^2 r_2 e^{- t_\mathrm{RT}/\tau_\mathrm{loss}}  e^{i2kL}}{1-r_1 r_2 e^{- t_\mathrm{RT}/\tau_\mathrm{loss}} e^{i2kL}} \approx -\frac{\mathcal{F}}{\pi} \Big[\frac{-r_1 +  r_2(r_1^2 + t_1^2)(1-\mathscr{L}_\mathrm{RT})}{\sqrt{r_1 r_2}} \Big]\]
we know that \(t_1^2 << r_1^2\):

\[r_\mathrm{FP} \approx -\frac{\mathcal{F}}{\pi} \Big[ \frac{r_1(-1 + (1 - \pi/\mathcal{F}) (1- \mathscr{L}_\mathrm{RT}))}{\sqrt{r_1 r_2}} \Big] \approx  -\Big(\frac{r_1}{r_2}\Big)^{1/2} \frac{\mathcal{F}}{\pi} \Big[- \pi/\mathcal{F} - \mathscr{L}_\mathrm{RT} + (\mathscr{L}_\mathrm{RT}\pi)/\mathcal{F}) \Big] \]And
\(\mathscr{L}_\mathrm{RT} <<1\) with \(r_1 /r_2 \approx 1\) we get:

\[r_\mathrm{FP} \approx -1 + \frac{\mathcal{F}}{\pi} \mathscr{L}_\mathrm{RT}\]If
we're operating at a dark fringe, at the symmetric port we see
superimposed fields:

\[ E_\mathrm{SYM}  = \frac{E_i}{2} \Big[ r_\mathrm{FPX}e^{2ik\mathscr{l}_x} + r_\mathrm{FPY}e^{2ik\mathscr{l}_y} \Big] \]

Where we assume that the short Michelson arms and reflection
coefficients are roughly equal (\(\mathscr{l}_x = \mathscr{l}_y\),
\(r_\mathrm{FPX} = r_\mathrm{FPY}\))

We also can average the short Michelson arm lengths
\((\mathscr{l}_x + \mathscr{l}_y)/2\) such that the effective reflection
coefficient is:
\(r_\mathrm{FPMI} = e^{2ik\mathscr{l}}(- 1 + \frac{\mathcal{F}}{\pi} \mathscr{L}_\mathrm{RT})\)Knowing
this we create the following expression for the circulating power within
the cavity:

\[ P_\mathrm{PRC} = \frac{|t_\mathrm{PRM}| ^2}{|1-r_\mathrm{PRM} r_\mathrm{FPMI} e^{2ik(L_\mathrm{PRC2BS} + L_\mathrm{SMICH})}| ^2} P_\mathrm{in}\] where
$ | t\_\mathrm{PRM} | ^2 = 1 - | r\_\mathrm{PRM} | ^ 2 $ and given a carrier resonance
condition we want to maximize the power with a variable PRM
reflectivity:

\[\frac{\partial P_\mathrm{PRC}}{\partial r_\mathrm{PRM}} = \frac{2r_\mathrm{PRM}^2(r_\mathrm{FPMI} - r_\mathrm{PRM})}{(1 - r_\mathrm{PRM} r_\mathrm{FPMI})^3} = 0\]

which sets $ r\_\mathrm{PRM} = r\_\mathrm{FPMI} $ On resonance, the
power recyling gain
(\(G_\mathrm{PR} = \frac{P_\mathrm{PRC}}{P_\mathrm{in}}\)):

\[ G_\mathrm{PR} = \frac{\pi}{2 \mathcal{F} \mathscr{L}_\mathrm{RT}} \Bigg[ \frac{1}{1- \frac{\mathcal{F}\mathscr{L}_\mathrm{RT}}{2 \pi}} \Bigg] \]

\begin{lstlisting}[frame=single, language=Python]
r_FPMI = -r_1 + (T_1*r_2)/(1-r_1*r_2)
T_PRM = .03
R_PRM = 1-T_PRM
t_PRM = (T_PRM)**.5
r_PRM = (R_PRM)**.5
G_PRC = 1/(1-r_PRM*(r_FPMI))
\end{lstlisting}

\begin{lstlisting}[frame=single, language=Python]
L_rt = 75e-6
Finn = (np.pi*np.sqrt(r_1*r_2))/(1-r_1*r_2)
print(Finn)
\end{lstlisting}

\begin{lstlisting}
444.0741558169753
\end{lstlisting}

\begin{lstlisting}[frame=single, language=Python]
r_FPMI_approx = (1 - Finn*L_rt/np.pi)
\end{lstlisting}

\begin{lstlisting}[frame=single, language=Python]
r_range = np.arange(.9,1,1/(2**16))
\end{lstlisting}

\begin{lstlisting}[frame=single, language=Python]
G_PRC_ = ifco.PRG(L_rt, Finn, r_range, max=0) 
\end{lstlisting}

\begin{lstlisting}[frame=single, language=Python]
G_PRC_opt =  ifco.PRG(L_rt, Finn, r_FPMI, max=1)
\end{lstlisting}

\begin{lstlisting}[frame=single, language=Python]
plt.plot(r_range, G_PRC_, linewidth=line_width)
plt.axhline(G_PRC_opt, linestyle='--',linewidth=line_width, color='r')
plt.xlim(r_range[0], r_range[-1])
plt.xlabel('$\mathdefault{r_{PRM}}$ [arb]')
plt.ylabel('$\mathdefault{G_{PRC}}$ [arb]')
\end{lstlisting}

\begin{lstlisting}
Text(0, 0.5, '$\\mathdefault{G_{PRC}}$ [arb]')
\end{lstlisting}

\begin{lstlisting}[frame=single, language=Python]
G_PRC_actual = ifco.PRG(L_rt, Finn, r_PRM, max=1)
\end{lstlisting}

\begin{lstlisting}[frame=single, language=Python]
H_FPMI = ifco.fpmi_freq_resp(nu, r_1, t_1, r_2, L, PHI_0, P_IN, OMEG)
H_FPMI_LP = ifco.fpmi_freq_resp(nu, r_1, t_1, r_2, L, PHI_0, P_IN, OMEG, 
				low_pass='True')
\end{lstlisting}

\begin{lstlisting}[frame=single, language=Python]
H_PRFPMI = ((G_PRC_actual)**.5)*H_FPMI
\end{lstlisting}

We estimate the FP's pole frequency
\[  1 - r_1 r_2 e^{-2i \omega_g L / c} = 0 \] therefore when:
\[ e^{-i \omega_g L / c} = \frac{1}{\sqrt{r_1 r_2}} \] we acquire the
pole frequency \(\omega_\mathrm{pole}\) as indicated in the low pass
\[ f_\mathrm{pole} = \frac{1}{4\pi \tau_{s}} =  \frac{c}{4 \pi L} \frac{1- r_1 r_2}{\sqrt{r_1 r_2}} = \frac{\nu_\mathrm{FSR}}{2 \pi} \frac{1- r_1 r_2}{\sqrt{r_1 r_2}} = \frac{\nu_\mathrm{FSR}}{\mathcal{F}} \]Might
as well compare it to our Michelson response:
\[ H_{\mathrm{MI}}(\omega_g) = \frac{2 L \Omega}{c}e^{-i L \omega / c} \frac{\mathrm{sin}(L \omega /c)}{L \omega /c} \]

\begin{lstlisting}[frame=single, language=Python]
H_MI = ifco.mich_freq_resp(nu, L, PHI_0, P_IN, OMEG)
\end{lstlisting}

\begin{lstlisting}[frame=single, language=Python]
plt.loglog(nu, ifco.bode_amp(H_MI), label= 'MICH', linewidth= line_width, alpha=.3)
plt.loglog(nu, ifco.bode_amp(H_FPMI), label='FPMI', linewidth=line_width, alpha=.3)
plt.loglog(nu, ifco.bode_amp(H_PRFPMI), label='PRFPMI', linewidth = line_width)
#plt.axvline (x=f_pole,ymin=1e-11, color='red', linestyle='dotted', linewidth=3)
plt.xlim([1e0, 1e5])
plt.ylim([1e9,2e15])
plt.xlabel('frequency [Hz]')
plt.ylabel('H(f) [$\mathdefault{W/m}$]')
lgd=plt.legend()
plt.savefig('../figs/INTRO/prfpmi_fr.pdf', dpi=300, bbox_inches='tight')
\end{lstlisting}

\begin{lstlisting}[frame=single, language=Python]
plt.semilogx(nu,(180/np.pi)*np.arctan(np.imag(H_FPMI)/np.real(H_FPMI)), '--', 
	     linewidth=line_width)
plt.semilogx(nu,(180/np.pi)*np.arctan(np.imag(H_MI)/np.real(H_MI)), '--', 
	     linewidth=line_width)
plt.semilogx(nu,(180/np.pi)*np.arctan(np.imag(H_PRFPMI)/np.real(H_PRFPMI)),
	     linestyle='--', linewidth=line_width,dashes=(3,10))
plt.xlim([1,100000])
plt.ylabel('phase [deg]')
plt.xlabel('Frequency [Hz]')
\end{lstlisting}

\begin{lstlisting}
Text(0.5, 0, 'Frequency [Hz]')
\end{lstlisting}

\hypertarget{signal-recyclinginitially-not-used-in-early-iterations-of-ligo-intial-ligo-and-enhanced-ligo-signal-recycling-imagines-using-a-partially-reflective-mirror-at-the-anti-symmetric-port.-and-at-first-glance-it-seems-to-not-very-much-make-sense-to-have-a-mirror-at-detector-output-as-you-would-potentially-attenuate-gravitational-wave-signals-by-said-mirror-reflection-coefficient.}{%
\section{SIGNAL RECYCLING}
Initially not used in early iterations of LIGO
(intial LIGO and enhanced LIGO) signal recycling imagines using a
partially reflective mirror at the anti-symmetric port. And at first
glance it seems to not very much make sense to have a mirror at detector
output as you would potentially attenuate gravitational wave signals by
said mirror reflection
coefficient.}\label{signal-recyclinginitially-not-used-in-early-iterations-of-ligo-intial-ligo-and-enhanced-ligo-signal-recycling-imagines-using-a-partially-reflective-mirror-at-the-anti-symmetric-port.-and-at-first-glance-it-seems-to-not-very-much-make-sense-to-have-a-mirror-at-detector-output-as-you-would-potentially-attenuate-gravitational-wave-signals-by-said-mirror-reflection-coefficient.}

While true, it is important to analyze the multi-state configurations
offered by such a mirror with various microscopic length tuning
configurations. What do I mean by this? Well, it helps to start
imagining by analogy of couple cavity relationship as established in the
power recycling discussion. The relationship of the differential signal
output of the PRFPMI with respect to the newly placed mirror at the
anti-symmetric port is represented by the following:

\[ t_\mathrm{SRC} = \frac{t_\mathrm{ITM}t_\mathrm{SRM} e^{i  (k + \Omega/c) \mathscr{l}_\mathrm{SRC}}}{1- r_\mathrm{ITM}r_\mathrm{SRM} e^{2i  (k + \Omega/c) \mathscr{l}_\mathrm{SRC}}}\]

\[ r_\mathrm{SRC} = \frac{r_\mathrm{ITM} - r_\mathrm{SRM} e^{2i  (k + \Omega/c) l_\mathrm{SRC}}}{1- r_\mathrm{ITM}r_\mathrm{SRM} e^{2i  (k + \Omega/c) \mathscr{l}_\mathrm{SRC}}}\]

as \(k \textgreater \textgreater \Omega_\mathrm{gw}/c\) for $ 1 \textless \Omega_\mathrm{gw} \textless 5 \cdot 10^3 $

Therefore with a pre-defined
\(T_\mathrm{ITM} + R_\mathrm{ITM} + L_\mathrm{ITM} = 1\) the coupled
cavity pole AND gain is a function of the SRM reflectivity and
microscopic length tuning:

\[ t_\mathrm{SRC} = \frac{t_\mathrm{ITM}t_\mathrm{SRM} e^{i k \mathscr{l}_\mathrm{SRC}}}{1- r_\mathrm{ITM}r_\mathrm{SRM} e^{2i k \mathscr{l}_\mathrm{SRC}}}\]

\[ r_\mathrm{SRC} = \frac{r_\mathrm{ITM} - r_\mathrm{SRM} e^{2i k l_\mathrm{SRC}}}{1- r_\mathrm{ITM}r_\mathrm{SRM} e^{2i k \mathscr{l}_\mathrm{SRC}}}\]We
now observe the tuning extrema: 
- On resonance
\(2ik \mathscr{l}_\mathrm{SRC} = 2i\phi_\mathrm{SRC} = 0\):
\[ r_\mathrm{SRC \; , \; \phi_{SRC} = 0} = \frac{r_\mathrm{ITM} - r_\mathrm{SRM}}{1- r_\mathrm{ITM}r_\mathrm{SRM}}\]
- On resonance $2ik \mathscr{l}_\mathrm{SRC} = 2i\phi_\mathrm{SRC}= \frac{\pi}{2} $:
\[ r_\mathrm{SRC \; , \; \phi_{SRC} = \pi} = \frac{r_\mathrm{ITM} + r_\mathrm{SRM}}{1+ r_\mathrm{ITM}r_\mathrm{SRM}}\]
\begin{multline*}
	\mathrm{H}_\mathrm{DRFPMI} =  \mathrm{G}_\mathrm{PR} \mathrm{P}_\mathrm{in} L \Omega \bigg[ \frac{ t_\mathrm{ITM}^2 r_\mathrm{ETM}}{(t_\mathrm{ITM}^2 + r_\mathrm{ITM}^2)r_\mathrm{ETM} - r_\mathrm{ITM}} \frac{t_\mathrm{SRM} t_\mathrm{ITM} e^{i\phi_\mathrm{SRC}}}{1-r_\mathrm{ITM} r_\mathrm{SRM} e^{i2\phi_\mathrm{SRC}}} \times \\
				      \frac{e^{-i 2 \pi L f / c} \mathrm{sin}( 2 \pi f / c)}{ 2 \pi L f } \times \\
				      \frac{\mathrm{sin}(\phi_0)}{1- [(r_\mathrm{ITM} - r_\mathrm{SRM} e^{i2\phi_\mathrm{SRC}})/(1-r_\mathrm{ITM} r_\mathrm{SRM} e^{i2\phi_\mathrm{SRC}})] r_\mathrm{ETM} e^{-i 4 \pi L f / c}} \bigg] 
\end{multline*}

\begin{lstlisting}[frame=single, language=Python]
l_SRC = 56  #[m]
T_SRM = .30
R_SRM = 1-T_SRM
t_SRM = T_SRM**.5
r_SRM = R_SRM**.5
phi_SRC = np.pi
\end{lstlisting}

\begin{lstlisting}[frame=single, language=Python]
H_DRFPMI = ifco.drfpmi_freq_resp(nu, G_PRC_opt, r_1, t_1, r_2, r_SRM, t_SRM, 
				 phi_SRC, L, PHI_0, P_IN, OMEG)
\end{lstlisting}

\begin{lstlisting}[frame=single, language=Python]
bode_test=False
if bode_test:
    fig, ax1 = plt.subplots()
    ax1.set_xlabel('frequency [Hz]')
    ax1.set_ylabel('H$_\mathdefault{FPMI}$  [$\mathdefault{W/m}$]  ', color='C0')
    #ax1.plot(w/(FSR), F_w_cc_modsq*100)
    ax1.loglog(nu, ifco.bode_amp(H_FPMI), label='FPMI', linewidth=line_width, 
	       linestyle=':',color='C0')
    ax1.loglog(nu, ifco.bode_amp(H_PRFPMI), label='PRFPMI',  
	       linewidth=line_width, color='C0')
    ax1.loglog(nu, ifco.bode_amp(H_DRFPMI), label='DRFPMI',  
	       linewidth=line_width, color='C1')
    ax1.legend()
    #ax1.loglog(w,H_MI_modsq, label= 'MICH', linewidth= 5)
    #ax1.loglog(w,H_FPMI_LP_modsq*H_FPMI_modsq[0], label='FPMI LP', 
		linewidth = 20.0, alpha=0.25,color='C2')
    #ax1.axvline (x=f_pole,ymin=1e-13, color='red', linestyle='dotted', linewidth=3)
    ax2 = ax1.twinx()
    ax2.semilogx(nu, ifco.bode_ph(H_FPMI),'--', linewidth=7.5, color='C0', alpha=.3)
    ax2.semilogx(nu, ifco.bode_ph(H_DRFPMI),'--', linewidth=7.5, color='C1', alpha=.3)
    ax2.grid(b=False, which='both', axis='y')
    #ax2.semilogx(w,(180/np.pi)*np.arctan(np.imag(H_MI)/np.real(H_MI)), '--')
    #ax2.semilogx(w,(180/np.pi)*np.arctan(np.imag(H_FPMI_LP)/np.real(H_FPMI_LP)),
		  linestyle='--', linewidth=20.0,dashes=(4,10),alpha=.25, color='C2')
    plt.xlim([1,1e5])
    plt.ylabel('phase [deg]', color='C1', alpha=.5)
\end{lstlisting}

\begin{lstlisting}[frame=single, language=Python]
plt.loglog(nu, ifco.bode_amp(H_MI), label= 'MICH', linewidth= line_width, alpha=.4)
plt.loglog(nu, ifco.bode_amp(H_FPMI), label='FPMI', linewidth=line_width, alpha=.4)
plt.loglog(nu, ifco.bode_amp(H_PRFPMI), label='PRFPMI', linewidth = line_width, 
	   alpha=.4)
plt.loglog(nu, ifco.bode_amp(H_DRFPMI), label='DRFPMI', linewidth = line_width)
#plt.axvline (x=f_pole,ymin=1e-11, color='red', linestyle='dotted', linewidth=3)
plt.xlim([1e0, 1e5])
plt.ylim([1e9, 2e15])
plt.xlabel('frequency [Hz]')
plt.ylabel('H(f) [$\mathdefault{W/m}$]')
lgd=plt.legend()
plt.savefig('../figs/INTRO/drfpmi_fr.pdf', dpi=300, bbox_inches='tight')
\end{lstlisting}

\begin{lstlisting}[frame=single, language=Python]
plt.semilogx(nu,ifco.bode_ph(H_MI), '--', linewidth=line_width, 
	     alpha=.4, label='MICH')
plt.semilogx(nu,ifco.bode_ph(H_FPMI),'--', linewidth=line_width, 
	     alpha=.4, label='FPMI')
plt.semilogx(nu,ifco.bode_ph(H_PRFPMI),linestyle='--', 
	     linewidth=line_width,dashes=(3,10), alpha=.4, label='PRFPMI')
plt.semilogx(nu,ifco.bode_ph(H_DRFPMI),'--', linewidth=line_width, label='DRFPMI')
plt.xlim([1,100000])
plt.ylim([-91,91])
plt.ylabel('phase [deg]')
plt.xlabel('Frequency [Hz]')
plt.legend()
\end{lstlisting}

\begin{lstlisting}[frame=single, language=Python]
Sn = ifco.N_shot(OMEG, P_IN)
\end{lstlisting}

\begin{lstlisting}[frame=single, language=Python]
plt.loglog(nu, Sn/ifco.bode_amp(H_MI), label= 'MICH', linewidth=line_width)
plt.loglog(nu, Sn/ifco.bode_amp(H_FPMI), label='FPMI', linewidth=line_width)
plt.loglog(nu, Sn/ifco.bode_amp(H_PRFPMI), label='PRFPMI', linewidth=line_width)
plt.loglog(nu, Sn/ifco.bode_amp(H_DRFPMI), label='DRFPMI', linewidth=line_width)
#plt.axvline (x=f_pole,ymin=1e-11, color='red', linestyle='dotted', linewidth=3)
plt.ylim([1e-24,2e-19])
plt.xlim([1e0, 1e5])
plt.xlabel('frequency [Hz]')
plt.ylabel('$\mathdefault{[ 1 / \sqrt{\mathdefault{Hz}}]}$')
lgd=plt.legend()
plt.savefig('../figs/INTRO/strain_compare.pdf', dpi=300, bbox_inches='tight')
\end{lstlisting}

