\begin{spacing}{1} \begin{lstlisting}[language=Python]
import matplotlib
import matplotlib.pyplot as plt
import numpy as np
from scipy import signal
import h5py
import os

plt_style_dir = '../../../my_python/matplotlib/stylelib/'
if os.path.isdir(plt_style_dir) == True:
    plt.style.use(plt_style_dir + 'ppt2latex')
plt.rcParams["font.family"] = "Times New Roman"
\end{lstlisting} \end{spacing}

\begin{spacing}{1} \begin{lstlisting}[language=Python]
# Establish default color array
prop_cycle = plt.rcParams['axes.prop_cycle']
colors = prop_cycle.by_key()['color']
lin_thickness=4
\end{lstlisting} \end{spacing}

\begin{spacing}{1} \begin{lstlisting}[language=Python]
## Set figure saving directory
thesis_dir = '../doc/figures/python/'
thesis_dir='../../dissertation/figs/TCS/IRHF/'
\end{lstlisting} \end{spacing}

\hypertarget{rh-filter}{%
\section{RH FILTER}\label{rh-filter}}

\hypertarget{generating-plotting-plant-filter}{%
\subsection{Generating / plotting plant
filter}\label{generating-plotting-plant-filter}}

\begin{spacing}{1} \begin{lstlisting}[language=Python]
ITMYRH_data = np.loadtxt('../data/ITMY_trend_10min_int_longer.dat')
t = np.arange(0,len(ITMYRH_data[:,0][2:]))*60.0*10.0
normalize = 3.13
print(len(t))
data_in = ITMYRH_data[:,1][2:]
b, a = signal.butter(2, .2)
#data_new = signal.filtfilt(b,a,data_in)
data_new = data_in
plt.figure()
ir = (data_new[1:] - data_new[:-1])/normalize
ir_new = ir
fig1 = plt.figure(figsize=(13,10))
plt.plot(t, data_new, label='Step response',linewidth=lin_thickness)
#plt.plot(t[:(len(t)-1)], ir, label= 'Impulse response')
plt.xlabel('time [s]')
plt.ylabel('Defocus [m$^{-1}$]')
#plt.legend(fontsize='medium')
plt.show()

Fs = 1/(t[2]-t[1])
#print(Fs)

[F,H]=signal.freqz(ir_new,1, worN=3000,whole=False) 
fig2 = plt.figure(figsize=(13,10))
plt.loglog(F*Fs/(2*np.pi), abs(H), label='Plant filter',linewidth=lin_thickness)
plt.ylabel('Magnitude [m$^{-1}$/W]')
plt.xlabel('Frequency [Hz]')
plt.legend()
plt.show()

print(max(ir_new))
\end{lstlisting} \end{spacing}

\begin{spacing}{1} \begin{lstlisting}
83
\end{lstlisting} \end{spacing}

%\begin{figure}
%\centering
%\includegraphics{IRHF_w_self_heating_files/IRHF_w_self_heating_5_2.png}
%\caption{png}
%\end{figure}
%
%\begin{figure}
%\centering
%\includegraphics{IRHF_w_self_heating_files/IRHF_w_self_heating_5_3.png}
%\caption{png}
%\end{figure}

\begin{spacing}{1} \begin{lstlisting}
4.993706070287539e-06
\end{lstlisting} \end{spacing}

\begin{spacing}{1} \begin{lstlisting}[language=Python]
F
\end{lstlisting} \end{spacing}

\begin{spacing}{1} \begin{lstlisting}
array([0.00000000e+00, 1.04719755e-03, 2.09439510e-03, ...,
       3.13845106e+00, 3.13949826e+00, 3.14054546e+00])
\end{lstlisting} \end{spacing}

\begin{spacing}{1} \begin{lstlisting}[language=Python]
adj_data = data_new + abs(min(data_new))
mod_data = np.concatenate([np.zeros((10,)), adj_data])
mod_t = np.arange(0,len(mod_data))*60.0*10.0/(3600)
mod_rh_inp = np.concatenate([np.ones((10,))*3.13, np.zeros(adj_data.shape)])
\end{lstlisting} \end{spacing}

\begin{spacing}{1} \begin{lstlisting}[language=Python]
fig, ax1 = plt.subplots() 

ax1.set_xlabel('time [hr]') 
ax1.set_ylabel('Primary-axis') 
ax1.plot(mod_t, mod_rh_inp,'--',linewidth=lin_thickness, color = colors[0]) 
ax1.tick_params(axis='y', labelcolor=colors[0])
ax1.set_ylabel('RH power [W]', color=colors[0])
#ax1.grid(b=False,which='minor',linestyle='--')
#ax1.grid(b=False,which='major',linestyle='--')
ax1.minorticks_off()
ax1.set_xlim([0,mod_t[-1]])
ax1.set_ylim([-.01,4])

ax2 = ax1.twinx() 
ax2.plot(mod_t, mod_data,linewidth=lin_thickness, color = colors[1])
ax2.set_ylabel('Defocus [m$^{-1}$]',color= colors[1])
##plt.grid(b=True,which='minor',linestyle='--')
##plt.grid(b=True,which='major',linestyle='--')
##plt.minorticks_on()
ax2.set_xlim([0,mod_t[-1]])
ax2.tick_params(axis='y', labelcolor=colors[1])
ax2.ticklabel_format(style='sci', axis='y',scilimits=(0,-5))

ax2.set_ylim([-.003e-4,1.2e-4])

fig.savefig(thesis_dir + 'Meas_response.pdf', dpi=300, format='pdf', bbox_inches='tight')
\end{lstlisting} \end{spacing}

%\begin{figure}
%\centering
%\includegraphics{IRHF_w_self_heating_files/IRHF_w_self_heating_8_0.png}
%\caption{png}
%\end{figure}

\begin{spacing}{1} \begin{lstlisting}[language=Python]
print('Only plots up to the nyqist frequency: {} Hz'.format(F[-1]*Fs/(2*np.pi)))
\end{lstlisting} \end{spacing}

\begin{spacing}{1} \begin{lstlisting}
Only plots up to the nyqist frequency: 0.0008330555555555556 Hz
\end{lstlisting} \end{spacing}

\begin{spacing}{1} \begin{lstlisting}[language=Python]
zeros = 5.0e-6
fit_zeros = -2.0*np.pi*zeros
poles = np.array([1.3e-5, 5.0e-5 ,9.5e-5])
fit_poles = -2.0*np.pi*poles

k = 1 #This gain is not initally correct

s1 = signal.ZerosPolesGain(fit_zeros, fit_poles, k)
F_2, H_2 = signal.freqresp(s1, F*(Fs/2.0))

#[F_2,H_2] = signal.freqs(b_2, a_2)
k_new = abs(H[0])/abs(H_2[0])

plt.loglog(F_2/(2*np.pi), abs(H_2)*k_new, label='Fitted zpk filter',linewidth=lin_thickness)
plt.loglog(F/(2*np.pi)*Fs, abs(H), label='Measured (step response) filter',linewidth=lin_thickness)
plt.ylabel('Magnitude [W/m$^{-1}$]')
plt.xlabel('Frequency [Hz]')
plt.legend()
#plt.title('RH plant filter (H(s))')
plt.xlim([0,(F[-1]/(2*np.pi)*Fs)])
print(k_new) #Spit out the new gain
##plt.grid(b=True,which='minor')
##plt.grid(b=True,which='major')
##plt.minorticks_on()

model_zpk = signal.ZerosPolesGain(fit_zeros, fit_poles,k_new)

plt.savefig(thesis_dir+'RH_plant_filter_fit.pdf',bbox_inches = 'tight')
\end{lstlisting} \end{spacing}

\begin{spacing}{1} \begin{lstlisting}
/var/folders/7r/5_lt1_453rxbdqfqjyb0kx500000gn/T/ipykernel_20903/2209851138.py:20: UserWarning: Attempted to set non-positive left xlim on a log-scaled axis.
Invalid limit will be ignored.
  plt.xlim([0,(F[-1]/(2*np.pi)*Fs)])


9.729529652779821e-12
\end{lstlisting} \end{spacing}

%\begin{figure}
%\centering
%\includegraphics{IRHF_w_self_heating_files/IRHF_w_self_heating_10_2.png}
%\caption{png}
%\end{figure}

\begin{spacing}{1} \begin{lstlisting}[language=Python]
model_zpk
\end{lstlisting} \end{spacing}

\begin{spacing}{1} \begin{lstlisting}
ZerosPolesGainContinuous(
array([-3.14159265e-05]),
array([-8.16814090e-05, -3.14159265e-04, -5.96902604e-04]),
9.729529652779821e-12,
dt: None
)
\end{lstlisting} \end{spacing}

\hypertarget{now-to-invert-the-plant-filter-just-swapping-the-poles-and-the-zeros-and-inverting-gain-h-1s}{%
\subsection{\texorpdfstring{Now to invert the plant filter (just
swapping the poles and the zeros and inverting gain)
(H\(^{-1}\)(s))}{Now to invert the plant filter (just swapping the poles and the zeros and inverting gain) (H\^{}\{-1\}(s))}}\label{now-to-invert-the-plant-filter-just-swapping-the-poles-and-the-zeros-and-inverting-gain-h-1s}}

\begin{spacing}{1} \begin{lstlisting}[language=Python]
inv_model = signal.ZerosPolesGain(fit_poles, fit_zeros,1/k_new)
F_3, H_3 = signal.freqresp(inv_model, F*(Fs/2.0))
fig4 = plt.figure()
plt.loglog(F_3/(2*np.pi), abs(H_3), label='Fitted zpk Filter',linewidth=lin_thickness)
plt.ylabel('Magnitude [W/m$^{-1}$]')
plt.xlabel('Frequency [Hz]')
#plt.title('RH inverse filter ([H(s)]$^{-1}$)')
plt.xlim([0, F_3[-1]/(2*np.pi)])
###plt.grid(b=True,which='minor',linestyle='--')
##plt.grid(b=True,which='major',linestyle='--')
#plt.minorticks_on()
plt.savefig(thesis_dir+'RH_inv_filt.pdf',bbox_inches = 'tight')
\end{lstlisting} \end{spacing}

\begin{spacing}{1} \begin{lstlisting}
/var/folders/7r/5_lt1_453rxbdqfqjyb0kx500000gn/T/ipykernel_20903/1155100985.py:8: UserWarning: Attempted to set non-positive left xlim on a log-scaled axis.
Invalid limit will be ignored.
  plt.xlim([0, F_3[-1]/(2*np.pi)])
\end{lstlisting} \end{spacing}

%\begin{figure}
%\centering
%\includegraphics{IRHF_w_self_heating_files/IRHF_w_self_heating_13_1.png}
%\caption{png}
%\end{figure}

\hypertarget{stabilize-the-high-frequencies-to-dc-generating-h-1-s-g_ns}{%
\subsection{\texorpdfstring{Stabilize the high frequencies to DC
(Generating H\(^{-1}\) (s) *
G\(_{n}\)(s))}{Stabilize the high frequencies to DC (Generating H\^{}\{-1\} (s) * G\_\{n\}(s))}}\label{stabilize-the-high-frequencies-to-dc-generating-h-1-s-g_ns}}

Will also attempt to reduce the time constant

\begin{spacing}{1} \begin{lstlisting}[language=Python]
#pole_test = .0001113 + 1e-4
Hinv_G_1_filt = signal.ZerosPolesGain(fit_poles, [fit_zeros,-2.0*np.pi*.0001113129672, -2.0*np.pi*.0001113129672],1)
pole_shift = 3
Hinv_G_2_filt = signal.ZerosPolesGain(fit_poles, [fit_zeros,-2.0*np.pi*.0001113129672*pole_shift, -2.0*np.pi*.0001113129672*pole_shift],1)

## Plotting
freq = np.arange(10e-7,10e-2,1e-7)
F_4, H_4 = signal.freqresp(Hinv_G_1_filt,freq)
F_5, H_5 = signal.freqresp(Hinv_G_2_filt,freq)

fig5= plt.figure()
plt.loglog(F_4/(2*np.pi), abs(H_4), label='RH input filter',linewidth=lin_thickness)
#plt.loglog(F_5/(2*np.pi), abs(H_5), label='Livingston filter',linewidth=lin_thickness)
#plt.legend(fontsize='xx-large')
plt.xlim([F_4[0]/(2*np.pi),F_4[-1]/(2*np.pi)])
#plt.grid(b=True,which='minor',linestyle='--')
#plt.grid(b=True,which='major',linestyle='--')
##plt.minorticks_on()
plt.ylabel('Magnitude [arb]')
plt.xlabel('Frequency [Hz]')
#plt.title('Real-time RH filter [H(s)]$^{-1*}$', **title_font)

plt.savefig(thesis_dir+'RH_input_filt.pdf',bbox_inches='tight')
\end{lstlisting} \end{spacing}

%\begin{figure}
%\centering
%\includegraphics{IRHF_w_self_heating_files/IRHF_w_self_heating_15_0.png}
%\caption{png}
%\end{figure}

\begin{spacing}{1} \begin{lstlisting}[language=Python]
Hinv_G_1_filt
\end{lstlisting} \end{spacing}

\begin{spacing}{1} \begin{lstlisting}
ZerosPolesGainContinuous(
array([-8.16814090e-05, -3.14159265e-04, -5.96902604e-04]),
array([-3.14159265e-05, -6.99400000e-04, -6.99400000e-04]),
1,
dt: None
)
\end{lstlisting} \end{spacing}

\begin{spacing}{1} \begin{lstlisting}[language=Python]
Hinv_G_2_filt
\end{lstlisting} \end{spacing}

\begin{spacing}{1} \begin{lstlisting}
ZerosPolesGainContinuous(
array([-8.16814090e-05, -3.14159265e-04, -5.96902604e-04]),
array([-3.14159265e-05, -2.09820000e-03, -2.09820000e-03]),
1,
dt: None
)
\end{lstlisting} \end{spacing}

\begin{spacing}{1} \begin{lstlisting}[language=Python]
fig79= plt.figure()
plt.loglog(F_4/(2*np.pi), abs(H_4), label='G$_1$(s) input filter',linewidth=lin_thickness)
plt.loglog(F_5/(2*np.pi), abs(H_5), label='G$_2$(s) input filter',linewidth=lin_thickness)
plt.legend()
plt.xlim([F_4[0]/(2*np.pi),F_4[-1]/(2*np.pi)])
#plt.grid(b=True,which='minor',linestyle='--')
#plt.grid(b=True,which='major',linestyle='--')
#plt.minorticks_on()
plt.ylabel('Magnitude [arb]')
plt.xlabel('Frequency [Hz]')
#plt.title('Real-time RH filter [H(s)]$^{-1*}$', **title_font)

plt.savefig(thesis_dir+'RH_input_filt_G1_G2.pdf',bbox_inches='tight')
\end{lstlisting} \end{spacing}

%\begin{figure}
%\centering
%\includegraphics{IRHF_w_self_heating_files/IRHF_w_self_heating_18_0.png}
%\caption{png}
%\end{figure}

\hypertarget{comsol-self-heating-filter}{%
\section{COMSOL self heating filter}\label{comsol-self-heating-filter}}

\hypertarget{import-comsol-self-heating-data}{%
\subsection{Import COMSOL self heating
data}\label{import-comsol-self-heating-data}}

\begin{spacing}{1} \begin{lstlisting}[language=Python]
COM_data = np.loadtxt('../data/1W_self_heating_defocus_doublepass.txt')
t_com = COM_data[:,0]*3600
defocus = COM_data[:,1]/max(COM_data[:,1])
\end{lstlisting} \end{spacing}

\begin{spacing}{1} \begin{lstlisting}[language=Python]
fig6 = plt.figure()
plt.plot(t_com/3600,defocus,linewidth=lin_thickness)
plt.title('COMSOL self heating time series')
plt.xlabel('time [hrs]')
plt.ylabel('defocus [arb]')
max(defocus)
\end{lstlisting} \end{spacing}

\begin{spacing}{1} \begin{lstlisting}
1.0
\end{lstlisting} \end{spacing}

%\begin{figure}
%\centering
%\includegraphics{IRHF_w_self_heating_files/IRHF_w_self_heating_22_1.png}
%\caption{png}
%\end{figure}

\begin{spacing}{1} \begin{lstlisting}[language=Python]
ir_com  = (defocus[1:] - defocus[:-1])
t_ir = t_com[:((len(t_com)-1))]
\end{lstlisting} \end{spacing}

\begin{spacing}{1} \begin{lstlisting}[language=Python]
[F_ir,H_ir]=signal.freqz(ir_com, 1, worN=3000,whole=False) 
Fs_com =1/(t_com[1]-t_com[0])
\end{lstlisting} \end{spacing}

\begin{spacing}{1} \begin{lstlisting}[language=Python]
zeros_com = np.array([.9e-3,.3e-3])
fit_zeros_com = -2.0*np.pi*zeros_com
poles_com = np.array([.25e-3,.25e-3,1.6e-3])
fit_poles_com = -2.0*np.pi*poles_com

k_com =1 #This gain is not initally correct

zpk_com = signal.ZerosPolesGain(fit_zeros_com, fit_poles_com, k_com)
F_com, H_com = signal.freqresp(zpk_com, F_ir*(Fs_com/2.0))
k_new_com = abs(H_ir[0])/abs(H_ir[0]*H_com[0])

fig6 = plt.figure()
plt.loglog(F_com/(2*np.pi), abs(H_com)*k_new_com, label='Fitted zpk Filter',linewidth=lin_thickness)
plt.loglog(F_ir*Fs_com/(2*np.pi), abs(H_ir)/abs(H_ir[0]), label='Plant filter',linewidth=lin_thickness)
plt.ylabel('Magnitude [arb]')
plt.xlabel('Frequency [Hz]')
plt.title('Self Heating filter')
\end{lstlisting} \end{spacing}

\begin{spacing}{1} \begin{lstlisting}
Text(0.5, 1.0, 'Self Heating filter (G$_{2}$(s))')
\end{lstlisting} \end{spacing}

%\begin{figure}
%\centering
%\includegraphics{IRHF_w_self_heating_files/IRHF_w_self_heating_25_1.png}
%\caption{png}
%\end{figure}

\begin{spacing}{1} \begin{lstlisting}[language=Python]
G_2 = signal.ZerosPolesGain(fit_zeros_com, fit_poles_com, k_new_com)
unit_step_testing = np.zeros(np.shape(t_com))
unit_step_testing[t_com>0] = 1
[ _ ,y_self_test, _] = signal.lsim(G_2, unit_step_testing, t_com)
\end{lstlisting} \end{spacing}

\begin{spacing}{1} \begin{lstlisting}[language=Python]
fig7= plt.figure()
plt.plot(t_com/3600,defocus,label='measured',linewidth=lin_thickness)
plt.plot(t_com/3600,y_self_test,label='fit',linewidth=lin_thickness)
plt.title('Self heating time series (fit vs measured)')
plt.legend()
\end{lstlisting} \end{spacing}

\begin{spacing}{1} \begin{lstlisting}
<matplotlib.legend.Legend at 0x7fd94977a400>
\end{lstlisting} \end{spacing}

%\begin{figure}
%\centering
%\includegraphics{IRHF_w_self_heating_files/IRHF_w_self_heating_27_1.png}
%\caption{png}
%\end{figure}

\hypertarget{generating-time-series}{%
\section{Generating time series}\label{generating-time-series}}

\hypertarget{step-input-time-series}{%
\paragraph{Step input time series}\label{step-input-time-series}}

\begin{spacing}{1} \begin{lstlisting}[language=Python]
unit_step = np.zeros((t.shape[0]*30))
t_new = np.arange(0,len(unit_step))*60.0*1.0
## Generating simulated response
unit_step[t_new>9000] = 1
[t_mod_new,y_mod_sim,xout] = signal.lsim(model_zpk, unit_step, t_new)
\end{lstlisting} \end{spacing}

\hypertarget{conditioned-input-time-series}{%
\paragraph{Conditioned input time
series}\label{conditioned-input-time-series}}

\begin{spacing}{1} \begin{lstlisting}[language=Python]
unit_step2 = np.zeros((t.shape[0]*30))
unit_step2[t_new>(9000)] = pole_shift**2

[ _ ,y_inp_inv_L, _] = signal.lsim(Hinv_G_2_filt, unit_step2, t_new)
[ _ ,y_inp_inv_H, _] = signal.lsim(Hinv_G_1_filt, unit_step, t_new)
[ _ ,y_mod_sim_inv_L, _] = signal.lsim(model_zpk, y_inp_inv_L, t_new)
[ _ ,y_mod_sim_inv_H, _] = signal.lsim(model_zpk, y_inp_inv_H, t_new)
\end{lstlisting} \end{spacing}

\hypertarget{self-heating-time-series}{%
\paragraph{Self heating time series}\label{self-heating-time-series}}

\begin{spacing}{1} \begin{lstlisting}[language=Python]
unit_step3 = np.zeros((t.shape[0]*30))
t_offset =0
unit_step3[t_new>(9000+t_offset)] = 1
\end{lstlisting} \end{spacing}

\begin{spacing}{1} \begin{lstlisting}[language=Python]
[ _ ,y_sh_resp, _] = signal.lsim(G_2, unit_step3, t_new)
\end{lstlisting} \end{spacing}

\hypertarget{basic-performance}{%
\paragraph{Basic Performance}\label{basic-performance}}

\begin{spacing}{1} \begin{lstlisting}[language=Python]
fig = plt.figure()
plt.subplot(211)
plt.plot(t_new/3600, unit_step,linewidth = lin_thickness,label='RH step input')
plt.plot(t_new/3600, y_inp_inv_H,'--', linewidth = lin_thickness,color = 'red', label='RH filtered input')
plt.ylabel('RH power [W]')
#plt.title('RH step input vs. Filtered input')
plt.legend()
plt.xlim([0, t_new[-1]/3600])
#plt.grid(b=True,which='minor',linestyle='--')
#plt.grid(b=True,which='major',linestyle='--')
#plt.minorticks_on()
plt.subplot(212)
plt.plot(t_new/3600,-y_mod_sim, linewidth = lin_thickness,label = 'RH step input')
plt.plot(t_new/3600,-y_mod_sim_inv_H,'--', linewidth = lin_thickness,color='red',label ='RH filtered input')
plt.ylabel('Defocus [m$^{-1}$]')
plt.xlabel('time [hr]')
#plt.legend()
plt.xlim([0, t_new[-1]/3600])
#plt.grid(b=True,which='minor',linestyle='--')
#plt.grid(b=True,which='major',linestyle='--')
#plt.minorticks_on()
plt.ticklabel_format(style='sci', axis='y',scilimits=(0,-5))
fig.savefig(thesis_dir+'IRHF_step_vs_filt_step.pdf',bbox_inches='tight')
\end{lstlisting} \end{spacing}

%\begin{figure}
%\centering
%\includegraphics{IRHF_w_self_heating_files/IRHF_w_self_heating_37_0.png}
%\caption{png}
%\end{figure}

\hypertarget{all-curves-together}{%
\paragraph{All curves together}\label{all-curves-together}}

\begin{spacing}{1} \begin{lstlisting}[language=Python]
fig = plt.figure()
plt.subplot(211)
plt.plot(t_new/3600, unit_step,linewidth = lin_thickness,label='RH unfiltered step input')
plt.plot(t_new/3600, y_inp_inv_L,'--', linewidth = lin_thickness, color = 'green',label='RH conditioned input (G$_{1}$(s))')
plt.plot(t_new/3600, y_inp_inv_H,'--', linewidth = lin_thickness,color = 'red', label='RH conditioned input (G$_{2}$(s)')
plt.ylabel('RH power [W]')
plt.title('RH filtered response w/ self-heating')
plt.legend(fontsize='medium')
plt.xlim([0,20])
plt.subplot(212)
plt.plot(t_new/3600,-y_mod_sim, linewidth = lin_thickness,label = 'RH unfiltered step input')
plt.plot(t_new/3600,y_sh_resp*20e-6, linewidth = lin_thickness,color='orange',label ='self heating')
plt.plot(t_new/3600,-y_mod_sim_inv_L,'--', linewidth = lin_thickness,color='green',label ='RH conditioned input (G$_{1}$(s))')
plt.plot(t_new/3600,-y_mod_sim_inv_H,'--', linewidth = lin_thickness,color='red',label ='RH conditioned input (G$_{2}$(s)')
plt.plot(t_new/3600,y_sh_resp*20e-6 -y_mod_sim_inv_L,linewidth = lin_thickness,label='self heating + RH conditioned input (G$_{1}$(s)',color='purple')
plt.plot(t_new/3600,y_sh_resp*20e-6 -y_mod_sim_inv_H,linewidth = lin_thickness,label='self heating + RH conditioned input (G$_{2}$(s))',color='magenta')
plt.ylabel('Defocus [m$^{-1}$]')
plt.xlabel('time [hr]')
plt.legend(fontsize='medium')
plt.xlim([0,20])
fig.savefig(thesis_dir+'IRHF_compare_self_w_filter_compare.pdf',bbox_inches='tight')
\end{lstlisting} \end{spacing}

%\begin{figure}
%\centering
%\includegraphics{IRHF_w_self_heating_files/IRHF_w_self_heating_39_0.png}
%\caption{png}
%\end{figure}

\begin{spacing}{1} \begin{lstlisting}[language=Python]
fig8 = plt.figure()
plt.rc('font', size=25)
plt.plot(t_new/3600,y_sh_resp*20e-6, linewidth = lin_thickness,color='orange',label ='self heating with no RH')
plt.plot(t_new/3600,y_sh_resp*20e-6 -y_mod_sim,linewidth = lin_thickness,label='self heating + RH unfiltered input',color='purple')
plt.plot(t_new/3600,y_sh_resp*20e-6 -y_mod_sim_inv_H,'--',linewidth = lin_thickness,label='self heating + RH filtered input (H$^{-1}$(s)*G$_{1}$(s))',color='red')
plt.plot(t_new/3600,y_sh_resp*20e-6 -y_mod_sim_inv_L,'--',linewidth = lin_thickness,label='self heating + RH filtered input (H$^{-1}$(s)*G$_{2}$(s))',color='green')
plt.ylabel('Defocus [m$^{-1}$]')
plt.xlabel('time [hr]')
plt.ticklabel_format(style='sci', axis='y',scilimits=(0,-5))
plt.xlim([0,20])
plt.legend(loc='upper right',bbox_to_anchor=(1.0,.95))
fig8.savefig(thesis_dir+'IRHF_compare_w_self.pdf',bbox_inches='tight')
\end{lstlisting} \end{spacing}

%\begin{figure}
%\centering
%\includegraphics{IRHF_w_self_heating_files/IRHF_w_self_heating_40_0.png}
%\caption{png}
%\end{figure}

\hypertarget{set-rh-upper-limit}{%
\subsection{Set RH upper limit}\label{set-rh-upper-limit}}

\begin{spacing}{1} \begin{lstlisting}[language=Python]
upper_lim = np.ones(np.shape(t_new))*40
\end{lstlisting} \end{spacing}

\begin{spacing}{1} \begin{lstlisting}[language=Python]
fig9= plt.figure(figsize=(25,20))
plt.rc('font', size=30)
plt.subplot(211)
plt.plot(t_new/3600, unit_step,linewidth = lin_thickness,label='Step input', color= 'purple')
plt.plot(t_new/3600, y_inp_inv_L,'--', linewidth = lin_thickness, color = 'green',label='Filtered input')
#plt.plot(t_new/3600, upper_lim,':',linewidth = lin_thickness, color='magenta', label='RH upper limit')
#plt.plot(t_new/3600, y_inp_inv_H,'--', linewidth = lin_thickness,color = 'red', label='Filtered input (H$^{-1}$(s)G$_{1}$(s))')
#plt.minorticks_on()
#plt.grid(b=True,which='minor',linestyle='--')
#plt.grid(b=True,which='major',linestyle='--')
plt.ylabel('RH power [W]')
plt.xlim([0,20])
#plt.title('RH responses')
plt.legend(fontsize='large')
plt.subplot(212)
plt.ylabel('Defocus [m$^{-1}$]')
plt.plot(t_new/3600,y_sh_resp*20e-6, linewidth = lin_thickness,color='orange',label ='central heating with no RH')
plt.plot(t_new/3600,(y_sh_resp*20e-6 -y_mod_sim),linewidth = lin_thickness,label='central heating + RH w/ step input',color='purple')
plt.plot(t_new/3600,(y_sh_resp*20e-6 -y_mod_sim_inv_L),'--',linewidth = lin_thickness,label='central heating + RH w/ filtered input',color='green')
#plt.plot(t_new/3600,-y_mod_sim, linewidth = lin_thickness,label = 'Unfiltered step input',color='purple')
#plt.plot(t_new/3600,y_sh_resp*20e-6-y_mod_sim_inv_L,'--', linewidth = lin_thickness,color='green',label ='Filtered input (H$^{-1}$(s)G$_{2}$(s))')
#plt.plot(t_new/3600,-y_mod_sim_inv_H,'--', linewidth = lin_thickness,color='red',label ='Filtered input (H$^{-1}$(s)G$_{1}$(s))')
#plt.minorticks_on()
#plt.grid(b=True,which='minor',linestyle='--')
#plt.grid(b=True,which='major',linestyle='--')
plt.xlabel('time [hr]')
plt.ticklabel_format(style='sci', axis='y',scilimits=(0,-5))
plt.legend(loc='upper right', bbox_to_anchor=(1.0,.95),fontsize='large')
plt.xlim([0,20])

fig9.savefig(thesis_dir+'IRHF_compare_filts_PI_paper.pdf',bbox_inches='tight')
\end{lstlisting} \end{spacing}

%\begin{figure}
%\centering
%\includegraphics{IRHF_w_self_heating_files/IRHF_w_self_heating_43_0.png}
%\caption{png}
%\end{figure}

\begin{spacing}{1} \begin{lstlisting}[language=Python]
fig9= plt.figure(figsize=(25,20))
plt.rc('font', size=30)
plt.subplot(211)
plt.plot(t_new/3600, unit_step,linewidth = lin_thickness,label='Step input', color= 'purple')
#plt.plot(t_new/3600, upper_lim,':',linewidth = lin_thickness, color='magenta', label='RH upper limit')
plt.plot(t_new/3600, y_inp_inv_L,'--', linewidth = lin_thickness, color = 'green',label='Filtered input(H$^{-1}$(s)G$_{2}$(s))')
plt.plot(t_new/3600, y_inp_inv_H,'--', linewidth = lin_thickness,color = 'red', label='Filtered input (H$^{-1}$(s)G$_{1}$(s))')
#plt.minorticks_on()
#plt.grid(b=True,which='minor',linestyle='--')
#plt.grid(b=True,which='major',linestyle='--')
plt.ylabel('RH power [W]')
plt.xlim([0,20])
#plt.title('RH responses')
plt.legend(fontsize='large')
plt.subplot(212)
plt.ylabel('Defocus [m$^{-1}$]')
plt.plot(t_new/3600,y_sh_resp*20e-6, linewidth = lin_thickness,color='orange',label ='self heating with no RH')
plt.plot(t_new/3600,(y_sh_resp*20e-6 -y_mod_sim),linewidth = lin_thickness,label='self heating + RH w/ step input',color='purple')
#plt.plot(t_new/3600,(y_sh_resp*20e-6 -y_mod_sim_inv_L),'--',linewidth = lin_thickness,label='self heating + RH w/ filtered input',color='green')
#plt.plot(t_new/3600,-y_mod_sim, linewidth = lin_thickness,label = 'Unfiltered step input',color='purple')
plt.plot(t_new/3600,(y_sh_resp*20e-6-y_mod_sim_inv_L),'--', linewidth = lin_thickness,color='green',label ='Filtered input (H$^{-1}$(s)G$_{2}$(s))')
plt.plot(t_new/3600,(y_sh_resp*20e-6-y_mod_sim_inv_H),'--', linewidth = lin_thickness,color='red',label ='Filtered input (H$^{-1}$(s)G$_{1}$(s))')
#plt.minorticks_on()
#plt.grid(b=True,which='minor',linestyle='--')
#plt.grid(b=True,which='major',linestyle='--')
plt.xlabel('time [hr]')
plt.ticklabel_format(style='sci', axis='y',scilimits=(0,-5))
plt.legend(loc='upper right', bbox_to_anchor=(1.0,.97),fontsize='large')
plt.xlim([0,20])

fig9.savefig(thesis_dir+'IRHF_compare_filts.pdf',bbox_inches='tight')
fig9.savefig(thesis_dir+'IRHF_compare_filts.pdf',bbox_inches='tight')
\end{lstlisting} \end{spacing}

%\begin{figure}
%\centering
%\includegraphics{IRHF_w_self_heating_files/IRHF_w_self_heating_44_0.png}
%\caption{png}
%\end{figure}

\begin{spacing}{1} \begin{lstlisting}[language=Python]
fig9= plt.figure(figsize=(17,15))
plt.rc('font', size=25)
plt.subplot(211)
plt.plot(t_new/3600, unit_step,linewidth = lin_thickness,label='Step input', color= 'purple')
plt.plot(t_new/3600, y_inp_inv_L,'--', linewidth = lin_thickness, color = 'green',label='Filtered input (H$^{-1}$(s)G$_{1}$(s))')
#plt.plot(t_new/3600, upper_lim,':',linewidth = lin_thickness, color='magenta', label='RH upper limit')
plt.plot(t_new/3600, y_inp_inv_H,'--', linewidth = lin_thickness,color = 'red', label='Filtered input (H$^{-1}$(s)G$_{2}$(s))')
#plt.minorticks_on()
#plt.grid(b=True,which='minor',linestyle='--')
#plt.grid(b=True,which='major',linestyle='--')
plt.ylabel('RH power [W]')
plt.xlim([0,t_new[-1]/3600])
#plt.title('RH responses')
plt.legend(fontsize='medium')
plt.subplot(212)
plt.ylabel('Defocus [m$^{-1}$]')
plt.plot(t_new/3600,y_sh_resp*20e-6, linewidth = lin_thickness,color='orange',label ='self heating w/ no RH')
plt.plot(t_new/3600,y_sh_resp*20e-6 -y_mod_sim,linewidth = lin_thickness,label='self heating + step input',color='purple')
plt.plot(t_new/3600,y_sh_resp*20e-6 -y_mod_sim_inv_L,'--',linewidth = lin_thickness,label='self heating + filtered input (H$^{-1}$(s)G$_{1}$(s))',color='green')
plt.plot(t_new/3600,y_sh_resp*20e-6 -y_mod_sim_inv_H,'--',linewidth = lin_thickness,label='self heating + filtered input (H$^{-1}$(s)G$_{2}$(s))',color='red')
#plt.plot(t_new/3600,-y_mod_sim, linewidth = lin_thickness,label = 'Unfiltered step input',color='purple')
#plt.plot(t_new/3600,-y_mod_sim_inv_L,'--', linewidth = lin_thickness,color='green',label ='Filtered input (H$^{-1}$(s)G$_{2}$(s))')
#plt.plot(t_new/3600,-y_mod_sim_inv_H,'--', linewidth = lin_thickness,color='red',label ='Filtered input (H$^{-1}$(s)G$_{1}$(s))')
#plt.minorticks_on()
#plt.grid(b=True,which='minor',linestyle='--')
#plt.grid(b=True,which='major',linestyle='--')
plt.xlabel('time [hr]')
plt.xlim([0,t_new[-1]/3600])
plt.ticklabel_format(style='sci', axis='y',scilimits=(0,-5))
plt.legend(loc='upper right', bbox_to_anchor=(1.0,.97),fontsize='medium')

fig9.savefig(thesis_dir+'IRHF_compare_filts.pdf',bbox_inches='tight')
\end{lstlisting} \end{spacing}

%\begin{figure}
%\centering
%\includegraphics{IRHF_w_self_heating_files/IRHF_w_self_heating_45_0.png}
%\caption{png}
%\end{figure}

\begin{spacing}{1} \begin{lstlisting}[language=Python]
fig = plt.figure(figsize=(17,15))
plt.subplot(311)
plt.plot(t_new/3600, unit_step,linewidth = lin_thickness,label='Unfiltered step input')
plt.plot(t_new/3600, y_inp_inv_L,'--', linewidth = lin_thickness, color = 'green',label='Conditioned input (G$_{1}$(s))')
plt.plot(t_new/3600, y_inp_inv_H,'--', linewidth = lin_thickness,color = 'red', label='Conditioned input (G$_{1}$(s))')
plt.ylabel('RH power [W]')
#plt.title('RH responses')
plt.legend(fontsize='small')
plt.subplot(312)
plt.ylabel('RH Defocus [m$^{-1}$]')
plt.plot(t_new/3600,-y_mod_sim, linewidth = lin_thickness,label = 'Unfiltered step input')
plt.plot(t_new/3600,-y_mod_sim_inv_L,'--', linewidth = lin_thickness,color='green',label ='Conditioned input (G$_{2}$(s))')
plt.plot(t_new/3600,-y_mod_sim_inv_H,'--', linewidth = lin_thickness,color='red',label ='Conditioned input (G$_{1}$(s))')
plt.legend(fontsize='x-small',loc='upper right')
plt.subplot(313)
plt.plot(t_new/3600,y_sh_resp*20e-6, linewidth = lin_thickness,color='orange',label ='Self heating')
plt.plot(t_new/3600,y_sh_resp*20e-6 -y_mod_sim,linewidth = lin_thickness,label='Self heating + RH unfiltered input',color='C0')
plt.plot(t_new/3600,y_sh_resp*20e-6 -y_mod_sim_inv_H,'--',linewidth = lin_thickness,label='Self heating + RH conditioned input (G$_{1}$(s))',color='red')
plt.plot(t_new/3600,y_sh_resp*20e-6 -y_mod_sim_inv_L,'--',linewidth = lin_thickness,label='Self heating + RH conditioned input (G$_{2}$(s))',color='green')
plt.ylabel('Total Defocus [m$^{-1}$]')
plt.xlabel('time [hr]')
plt.legend(fontsize='xx-small')
fig.savefig(thesis_dir+'IRHF_compare_all.pdf')
\end{lstlisting} \end{spacing}

%\begin{figure}
%\centering
%\includegraphics{IRHF_w_self_heating_files/IRHF_w_self_heating_46_0.png}
%\caption{png}
%\end{figure}

\hypertarget{g_1s---the-response-functionfor-the-above-scenario-we-have-the-following-g_s-a-double-pole-low-pass-at-1.113e-4}{%
\section{\texorpdfstring{G\(_{1}\)(s) -\textgreater{} The ``response
function''\#For the above scenario we have the following G\_s (a double
pole low pass at
1.113e-4)}{G\_\{1\}(s) -\textgreater{} The ``response function''\#For the above scenario we have the following G\_s (a double pole low pass at 1.113e-4)}}\label{g_1s---the-response-functionfor-the-above-scenario-we-have-the-following-g_s-a-double-pole-low-pass-at-1.113e-4}}

G\_1 = signal.ZerosPolesGain({[}{]}, {[}-2.0\emph{np.pi}.0001113129672,
-2.0\emph{np.pi}.0001113129672{]},1) F\_5, H\_5 =
signal.freqresp(G\_1,F\emph{(Fs/2.0)) k\_upd = 1/abs(H\_5{[}0{]}) G\_1 =
signal.ZerosPolesGain({[}{]}, {[}-2.0\emph{np.pi}.0001113129672,
-2.0\emph{np.pi}.0001113129672{]},k\_upd) plt.loglog(F\_5/(2}np.pi),
abs(H\_5)*k\_upd, label=`Fitted zpk Filter') plt.ylabel(`Magnitude
{[}arb{]}') plt.xlabel(`Frequency {[}Hz{]}')\#\#\#\# Deriving the above
filter is achieved by assuming a double pole solution. We are using
poles in order to reduce the gain at relatively high frequency (to avoid
implementing a unstable and unphysical filter. The reason we need two is
to balance out the number of poles and zeros. The value of the chosen
pole frequency is achieved by setting the ratio of the poles to zeros
equal to 1.\# Alternative response function (G\(_{2}\)(s) w/ self
heating?)\#\# To preface this discussion, the 2nd order low pass filter
appears to be one of two solutions: \#\# At Hanford we use the 2nd order
low pass (G\(_{1}\)(s) but as of this moment is not used to perform
dynamic thermal compensation. \#\# At Livinston they use the self
heating response to inform the final form of their filter

\begin{spacing}{1} \begin{lstlisting}[language=Python]
fig2 = plt.figure(figsize=(15,8))
plt.loglog(F_5/(2*np.pi), abs(H_5)*k_upd, label='G$_{1}(s)$')
plt.loglog(F_com/(2*np.pi), abs(H_com)*k_new_com, label='G$_{2}(s)$')
plt.ylabel('Magnitude [arb]')
plt.xlabel('Frequency [Hz]')
plt.title('G$_{1}$ vs. G$_{2}$')
plt.legend()
\end{lstlisting} \end{spacing}

\begin{spacing}{1} \begin{lstlisting}
<matplotlib.legend.Legend at 0x1601c7470>
\end{lstlisting} \end{spacing}

%\begin{figure}
%\centering
%\includegraphics{IRHF_w_self_heating_files/IRHF_w_self_heating_52_1.png}
%\caption{png}
%\end{figure}

\hypertarget{the-livingston-filter-is-what-we-will-construct-here.-to-do-that-we-will-first-attempt-multiplying-g_2s-the-self-heating-response-to-h-1s}{%
\subsection{\texorpdfstring{The Livingston filter is what we will
construct here. To do that, we will first attempt multiplying
G\(_{2}\)(s) (the self heating response) to
H\(^{-1}\)(s)}{The Livingston filter is what we will construct here. To do that, we will first attempt multiplying G\_\{2\}(s) (the self heating response) to H\^{}\{-1\}(s)}}\label{the-livingston-filter-is-what-we-will-construct-here.-to-do-that-we-will-first-attempt-multiplying-g_2s-the-self-heating-response-to-h-1s}}

\begin{spacing}{1} \begin{lstlisting}[language=Python]
FILT_LIV_zeros= np.append(fit_zeros_com,fit_poles)
FILT_LIV_poles= np.append(fit_poles_com,fit_zeros)
FILT_LIV = signal.ZerosPolesGain(FILT_LIV_zeros, FILT_LIV_poles, 1)
_ , H_G2 = signal.freqresp(FILT_LIV,np.arange(10e-7,10e-3,1e-7))
plt.loglog(np.arange(10e-7,10e-3,1e-7)/(2*np.pi), abs(H_G2)/abs(H_G2[0]))
\end{lstlisting} \end{spacing}

\begin{spacing}{1} \begin{lstlisting}
[<matplotlib.lines.Line2D at 0x16043d3c8>]
\end{lstlisting} \end{spacing}

%\begin{figure}
%\centering
%\includegraphics{IRHF_w_self_heating_files/IRHF_w_self_heating_54_1.png}
%\caption{png}
%\end{figure}

\hypertarget{not-enough-zeros-to-set-high-frequency-to-unity-gain-would-be-an-unphysical-without-one-more-pole}{%
\subsection{Not enough zeros to set high frequency to unity gain (would
be an unphysical without one more
pole)}\label{not-enough-zeros-to-set-high-frequency-to-unity-gain-would-be-an-unphysical-without-one-more-pole}}

\begin{spacing}{1} \begin{lstlisting}[language=Python]
FILT_LIV_poles_2= np.append(FILT_LIV_poles,-0.00020951281288038756)
\end{lstlisting} \end{spacing}

\begin{spacing}{1} \begin{lstlisting}[language=Python]
FILT_LIV = signal.ZerosPolesGain(FILT_LIV_zeros, FILT_LIV_poles_2, 1)
_ , H_G2 = signal.freqresp(FILT_LIV,freq)
plt.loglog(freq/(2*np.pi), abs(H_G2)/abs(H_G2[0]))
\end{lstlisting} \end{spacing}

\begin{spacing}{1} \begin{lstlisting}
[<matplotlib.lines.Line2D at 0x16828c9e8>]
\end{lstlisting} \end{spacing}

%\begin{figure}
%\centering
%\includegraphics{IRHF_w_self_heating_files/IRHF_w_self_heating_57_1.png}
%\caption{png}
%\end{figure}

\begin{spacing}{1} \begin{lstlisting}[language=Python]
[ _ ,y_G2, _] = signal.lsim(FILT_LIV, unit_step, t_new)
\end{lstlisting} \end{spacing}

\begin{spacing}{1} \begin{lstlisting}[language=Python]
[ _ ,y_G2_time, _] = signal.lsim(model_zpk, y_G2, t_new)
\end{lstlisting} \end{spacing}

\begin{spacing}{1} \begin{lstlisting}[language=Python]
fig = plt.figure(figsize=(17,10))
plt.subplot(211)
plt.plot(t_new/3600, unit_step,linewidth = lin_thickness,label='RH step input')
plt.plot(t_new/3600, y_inp_inv,'--', linewidth = lin_thickness,label='G$_{1}$')
plt.plot(t_new/3600,y_G2,'--', linewidth = lin_thickness,color='purple',label ='G$_{2}$')
plt.ylabel('RH power [W]')
plt.title('Comparison between RH inverted response with self heating')
plt.legend(fontsize='xx-large')
plt.subplot(212)
plt.plot(t_new/3600,-y_mod_sim, linewidth = lin_thickness,label = 'RH step input')
plt.plot(t_new/3600,-y_sh_resp*20e-6, linewidth = lin_thickness,color='magenta',label ='self heating (negative)')
plt.plot(t_new/3600,-y_mod_sim_inv,'--', linewidth = lin_thickness,color='orange',label ='G$_{1}$')
#plt.plot(t_new/3600,y_sh_resp*20e-6 -y_mod_sim_inv,linewidth = lin_thickness,label='diff (orange - green)',color='red')
plt.plot(t_new/3600,-y_G2_time,'--', linewidth = lin_thickness,color='purple',label ='G$_{2}$')
plt.ylabel('Defocus [m$^{-1}$]')
plt.xlabel('time [hr]')
plt.legend(fontsize='xx-large')
fig.savefig('G1_and_G2.pdf',bbox_inches='tight')
\end{lstlisting} \end{spacing}

%\begin{figure}
%\centering
%\includegraphics{IRHF_w_self_heating_files/IRHF_w_self_heating_60_0.png}
%\caption{png}
%\end{figure}
