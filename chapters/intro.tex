%% Introduction
\begin{itemize}
\item Brief overview on gravitational wave detection
\item Discuss progression of detector configurations:
\begin{itemize}
\item Resonant bar detectors
\begin{itemize}
\item History of Weber
\end{itemize}
\item Simple Michelson
The design configuration of the Michelson is a critical piece of not THE critial piece behind the story of gravitational wave detection. From a strict gravitational wave perspective the geometry is optimized to sense the induced quadrupole moment induced from gravitational wave perturbations. From an instrument design perspective you might notice much more: The instrument itself is designed to be sensitive to track optical path length differences between it's two arms and with the nature of this design nulls any effects that induce common motion with the arms. 
\begin{itemize}
\item Not enough
\item Arm lengths are too short! (Not enough optical sensitivity to the gravitational wave frequencies that we are interested in)
\item Insert figure of optical gain (sensitivity to change in phase) in Michelson with 4 km arms
\item Also: Cannot surpass the necessary shot noise limit with the amount of possible power that can be injected into the arms
\end{itemize}
\item Arm Folding:
\begin{itemize}
There are two possible ways to fold:
\begin{itemize}
\item Fabry-Pérot cavities
\item Herriot Delay lines
\end{itemize}
\end{itemize}
\item Fabry-Pe\'rot Michelson
\item
\end{itemize}
\end{itemize}
