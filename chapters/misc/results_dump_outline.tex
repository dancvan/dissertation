% Results dump


TCS
% Gathered elog sources:
  % From MARCH LVK 2019 conference presentation on TCS update for O3:
    TCS comissioning for O3
      Analysis of input coupler optics for robust picture of thermal compensation
      Includes actutors as well as probe beams
      Describe the motivation behind pre-loading
        - Using assumed absorption values
        - Updated absorption values when using HWS data (optical power decay over time)
      Point absorbers
        - Extremely disruptive and an inherent limitation to high power operation of DRFPMI inferferometers as gravitational wave detectors
          - Impacts sideband control signals to a point where the interferometer cannot stay locked once ``thermalized"
          - Optical loss to higher order modes and estimated beam contamination
        - Establishing metrics of improvement: arm power buildup, improve RF 9 and 45 buildups, reducing 9MHz RIN coupling.
      Sampling TCS parameter space
          - Sampling a unique thermal compensation, if the change is significantly large enough, can leave the interferometer in a unusable state for 12+ hours
          - Constructed a filter at the RH power signal input to achieve the final thermo-optic state within 2-3 hours
      Some tried solutions
          - First order problem was to see if adjusting for the ITM curvature mismatch based on the spherical power data alone was a solution:
            - This alone cannot solve the problem as higher order spatial actuation across the test mass is required for ITMY
          - CO2 mask
          - Moving beam spots (off center)


AlGaAs electro-optic effect results:

  Motivating the experiment:
    - EO effect with zincblende crystals
    - HR coating calculation
      - Stefan's thermal noise computation
    - d$\phi$/dE estimate

  Single PDH locked cavity with Electrostatic field assembly to measure the EO effect:
    - Concept Schema / Block diagram
    - Servo
      - Electronics
      - Optical plant
        - Cavity params
          - .1665m length (CVI Melles-Griot, ATFilms)
          - .333m ROC mirror
          - pl-pl mirror (AlGaAs)
        - Modelling E-field injection for chosen cavity length
          - Quick math for code
          - Results:
            - Plot
            - Relevant $E_z$ estimate
            - Feasible for EO measurment
            

  Mount assemblies
    Objectives: To maximize the normal electric field without clipping the beam. We are answering the question, ``will our cavity be able to yield a high enough SNR for a coherent measurement?"
      Assembly #1
        - Mount geometry and material
          -
          - Electrodes
            - Aluminum
            - 3inch large disks with 3mm diameter aperture
        - Electrode geometry
        - Likely dithering
          - Compliance intended for pitch / yaw control of mirror backfired and most likely coupled noise
        - Result: Size reduction is
      Assembly #2
        - Mount geometry and material
          - Electrodes
            - Geometry:
              - Pucks (Dims: )
            - Material:
              - Copper (higher inertia, less AC coupling?) and aluminum
          - Mount
            - Geometry:
              - Simple rectangular mount
        - Result: Mount design more mature but still suggesting that there is a flaw in design from drive coupling
      Assembly #3
        -
    PETG

    Plotted results:
: " "
