%% Electro-optic study of AlGaAs coated mirrors

 As mentioned in Section (1?) one of the many LIGO fundamental noise sources is coating thermal noise from the $\mathrm{SiO_2}/\mathrm{TiO_2:Ta_2O_5}$ aLIGO coatings. As aLIGO approaches its designed sensitivity various coating solutions are currently proposed to mitigate thermal noise coupling into the detector output.
With the potential to reduce coating Brownian noise by a factor of 10 \cite{Cole:2013}, $\algaas$ shows much promise with next generation detectors with a corresponding strain reduction by a factor of (5?) in comparison to the aLIGO coatings, though with different material properties of these crystalline coatings introduce new coating noise couplings.

A notable source of noise is the linear electro-optic property of the crystalline material (dn/dE), also known as the Pockels effect \cite{abernathy_poster}. Characterizing currently proposed $\algaas$ coated ``witness" samples thorough extensive analysis and experimental data of the aforementioned property is worthwhile for the use of $\algaas$ coatings in gravitational wave interferometers. The following section is dedicated to the discussion of brownian thermal noise and the fundamental limit it imposes on gravitational wave detection, the improvements of said limitations with $\gaas$/$\algaas$ coated mirrors over $\mathrm{SiO_2}/\mathrm{TiO_2:Ta_2O_5}$ coatings, the $\algaas$ linear electro-optic effect starting with the fundamentals on anisoptropic media and the electro-optic tensor of zincblende crystalline materials, estimates of the differential phase of light reflected from a GaAs/$\algaas$ coating, and an experiment constructed with the intention of measuring the linear electro-optic effect within a 1kHz $\rightarrow$ 1MHz region.

\subsection{Brownian Thermal Noise}
 In 1827 the Scottish botanist Robert Brown noticed a constant motion of pollen particulates on the surface of water; witnessing randomized collisions of the water molecules holding a kinetic energy proportional to the temperature ($k_BT$) \cite{Brown:1828}. It is because of his documented observations we name the phenomena Brownian motion. And although the observations were on motion of particulates in liquids, molecules and atoms within gases and solids also exhibit Brownian motion. For high precision optical experiments operating at room temperature (and higher due to high power resonant beams), understanding how much differential phase noise is imparted on the interferometer light passing through and reflecting from core optics is crucial. This requires knowledge of the mean squared displacement from each degree of freedom of the system which can be realized through the Fluctuation Dissapation theorem. Derived by H.B. Callen and T.A. Welton, the theorem states that for a randomly fluctuating linear force \cite{Callen:1951}:

 %% Further insight into Brownian motion was explored by Einstein where he was able to relate the mean-square displacement of a particle of radius $r_\mathrm{sph}$ on a fluid with viscosity $\eta$.

 %%\begin{equation}
 %%\overline{x^2} = k_B T  \frac{1}{3 \pi \eta  r_\mathrm{sph}}
 %%\end{equation}

 %%This relation has important implications about how the random motion or fluctuations of a particulate (the pollen) is influenced (dissipated) by the viscosity of the surrounding medium (water).

\begin{equation}
F_x^2(f) = 4 k_B T\; \Re[Z]
\end{equation}

 \noindent Where $\Re[Z]$ is the real part of the impedance of the system. This impedance directly relates to equations of motion:

 \begin{equation}
 Z = \frac{F}{\dot{x}}
 \end{equation}

\noindent Another useful form is the power spectrum of the fluctuating motion:
\begin{equation}\label{fdtpsd}
x^2 (f)  = \frac{4k_B T}{(2 \pi f)^2}\; \Re[Y]
\end{equation}

Where $Y$ is the inverse of the impedance or admittance. With this power spectra, modelling and budgeting notable LIGO fundamental noise contributions attributed to the choice of the materials used for mirror substrates, and highly reflective mirror coatings becomes less daunting. Though adequate modelling of internal force couplings for the aforementioned components is required.

\subsubsection{Internal friction in Materials and Loss angle}

Zener provides a model of the internal friction of materials incorporating anelasticity into the equations of motion \cite{zener:1948}:

\begin{equation}
F = k(1+i\phi)x + m\ddot{x}
\end{equation}

Where $m$ is mass attached to a spring with a spring constant $k(1+ i\phi)$ incorporating the degree of anelasticity $\phi$. From equations 3.5 and 3.3 we perform a Laplace transform and acquire the following form of admittance:
\begin{equation}
Y(s) = \frac{\dot{x}(s)}{F(s)} = \frac{-s}{k(1+i\phi) + ms^2}
\end{equation}

\noindent Or more transparently the Fourier representation since we assume a linear time invariant system:

\begin{equation}\label{admitint}
Y(\omega) = \frac{\dot{x}(\omega)}{F(\omega)} = \frac{-i\omega}{k(1+i\phi) - m\omega^2} = \frac{k \omega \phi - i \omega (k - m \omega^2)}{(k-m\omega^2)^2 +k^2 \phi^2}
\end{equation}

\noindent Plugging equation \ref{admitint} back into \ref{fdtpsd}:

\begin{equation}
x^2 (f)  = \frac{2k_B T}{\pi}\frac{k\phi}{(k-4\pi^2 m f^2)^2 + k^2 \phi^2}
\end{equation}
Computing the admittance from a Gaussian beam impinging upon a HR mirror can require expansion of all individual mechanical degrees of freedom of the test mass system across a relevant frequency range, and with that approach convergence is not guaranteed. Saulson and Gonzalez provide an alternative method to computing the admittance coined the ``direct approach" by Levin when computing the noise from a Gaussian beam on a LIGO HR test mass. The admittance can be acquired through:

\begin{equation}\label{admitdirec}
\Re[Y] = \frac{W_\mathrm{diss}}{F_o^2}
\end{equation}

\noindent $W_\mathrm{diss}$ is the dissipated power from the system due to an oscillating force $F_o$. This form of the admittance reveals an important result of the fluctuation dissapation theorem where an undriven system with a dissapative actor, imparts motion to the degrees of freedom via a driving force by virtue of that same actor at finite temperatures. This direct approach also allows the surface pressure applied by the Gaussian beam to interrogate which mechanical modes of the test mass impose a significant energy when \ref{admitdirec} is plugged into \ref{fdtpsd}. In the case of the gaussian beam / uncoated test mass studied by Levin \cite{levin:1998}:

\begin{equation}
S_x(f) = \frac{4 k_B T}{f} \frac{1-\sigma^2}{\pi^3 E_o r_o} I\phi \bigg[1- O\bigg( \frac{r_o}{R} \bigg)\bigg]
\end{equation}

%this requires that the driving force used in a lab mimics that of a force from a centered Gaussian beam.

\textcolor{red}{Refer to Levin appendix for more on how elasticity parameters are introduced?} Where $\phi$ and $E_o$ are the Poisson ratio and Young's modulus respectively, and $O(\frac{r_o}{R})$ contains a correction term contribution as a function of the small beam radius ($r_o$) relative to the mirror radius ($R$).

\subsubsection{Coating Brownian thermal noise}
Further investigations into the beam/optic system utilizing this approach and elasticity theory led to a deeper understanding about Brownian thermal noise contributions from LIGO test masses (substrate, suspensions, HR coating). Levin mentions, with details from Harry, that the noise contributed by a lossy mirror coating is proven to be to be the most significant contributor of brownian thermal noise. Hong provides a power spectral density \cite{Hong:2013}:

\begin{equation}
S_j^X = \frac{4k_B T \lambda \phi_x^j(1- \sigma_j - 2 \sigma_j^2)}{3 \pi^2 f Y_j (1-\sigma_j)^2 \omega_o^2}
\end{equation}

Where X represents bulk and shear with j = odd (material 1) and j = even (material 2) alternating layers representing high and low index materials j = odd (material 1) j = even (material 2) for an HR coating.

\subsubsection{$\mathrm{SiO_2}/\mathrm{TiO_2:Ta_2O_5}$ coating parameters}
Currently the LIGO interferometers deposit $\lambda$/4 stacks of silica and titania doped tantala on fused silica test mass substrates. Effective loss angle measurements \cite{Harry:06}

\textbf{Current $\mathrm{SiO_2}/\mathrm{TiO_2:Ta_2O_5}$ elasticity params, power spectra, and strain spectral density (order of magnitude estimate)}

\subsubsection{$\gaas$/$\algaas$ coating parameters}
\textcolor{red}{Specific coating parameters for most promising $\algaas$ candidates? Chat with Steve. Or just mention parameters that are listed in Cole 2013}
\cite{Cole:2013}

\textcolor{red}{Insert computed curves of the most precise and recent (effective) loss angle measurements (Nick Demos measurements?). More instructive to plot strain spectral density or displacement power spectra}

\noindent Currently thermal noise from the $\mathrm{SiO_2}/\mathrm{TiO_2:Ta_2O_5}$ optical coatings is the largest contributor of Brownian noise in LIGO compared to estimated substrate and suspension thermal noise \cite{Harry:06}. As of the end of O3, Brownian thermal noise is estimated to be ? orders of magnitude below the current sensitivity and it will prove to be the limiting source of noise as that sensitivity is increased with various other upgrades mitigating fundamental and technical noise. (\textcolor{red}{already mentioned in intro prior to this thermal noise section. Need to re-iterate in more detail?})
\\
\\
\subsection{Anisotropic media}
Unlike with isotropic media, we cannot assume that the index of refraction of anisotropic media is the same for all chosen wave vectors. This is a direct consequence of the birefringence of anisotropic media; characterized by the dielectric, permittivity, and polrization tensors.

\subsubsection{The Dielectric tensor}
Further elaborating on the nature of a generalized dielectric tensor for any wavevector is required to proceed:

\begin{equation}\label{eq:3.11}
D_i = \varepsilon_{ij}E_j
\end{equation}
Where D is the displacement vector and E is the electric field vector and $\varepsilon$ is the dielectric tensor. The displacement vector for isotropic media is retrieved when $i = j$ and $\varepsilon_i = \varepsilon$. To further understand the nature of the dielectric tensor we assert Poynting's theorem providing an energy conservation requirement:
\begin{equation}\label{eq:3.12}
\nabla \cdot \vec{S} = \frac{dU}{dt}
\end{equation}
Where $\vec{S} = \vec{E} \times \vec{H}$ is the poynting vector and $U = \frac{1}{8 \pi} \big( \vec{E} \cdot \vec{D} + \vec{B} \cdot \vec{H} \big)$ is the electromagnetic field density. The reader is left to perform the exercise and show that in order for \ref{eq:3.12} to hold true given \ref{eq:3.11}

\begin{equation}
\varepsilon_{ij} = \varepsilon_{ji}
\end{equation}
Stating that the dielectric tensor holds only six unique terms. The energy density is represented as an ellipsoid but with a coordinate transformation we can diagonalize and realize principal axes of the dielectric allowing a simpler form of the displacement, and in turn the energy density.

\subsubsection{Monochromatic plane wave propogation}
Revisiting Maxwell's equations for simple monochromatic plane wave solution gives provides further direction on how crystalline media may effect impinging light. Further elaborating, the following assumptions are made:

\begin{equation}
\vec{E} = E_o e^{(i \omega (\frac{n}{c} \vec{r}\cdot \vec{s}-t))}
\end{equation}
Where $n$ is the index of refraction, $c$ is the speed of light, $\vec{r}$ is the position vector and $\vec{s}$ is the unit wave normal.

\begin{equation}
\nabla \times \vec{H}= \frac{\partial \vec{D}}{\partial t}
\end{equation}
Where $\vec{H}$ is the magnetic field assuming the permeability $\mu$, and the generalized displacement vector $\vec{D}$ and electric field vector $\vec{E}$.

\begin{equation}
\nabla \times \vec{E} = -\mu \vec{H}
\end{equation}
Reducing to only the displacement and electric fields:

\begin{equation}\label{eq:3.17}
\vec{D} = \frac{n^2}{\mu}[\vec{E}-\vec{s}(\vec{s}\cdot \vec{E})]
\end{equation}
Maxwell's equations show that the electric field is not necessarily parallel to the displacement field and in most materials with non-zero polarizability tensors and dielectric tensors, it is not. But as specified above, the displacement vector, Electric field and unit wave normal are co-planar while remaining orthogonal to $\vec{H}$. Assuming we are operating within a coordinate system aligned with the principal dielectric axes, we substitute \ref{eq:3.11} into \ref{eq:3.17}:

\begin{equation}\label{eq:3.18}
E_i = \frac{n^2 s_i (\vec{E}\cdot\vec{s})}{n^2 - \mu \varepsilon_i}
\end{equation}
From here it can be shown that for a general plane wave there exist two unique refractive index solutions within the constructed dielectric. Though using this result to show this requires revisiting geometrical conditions that are best visualized using a method introduced in the next section. \textcolor{red}{For a more rigorous proof, see Appendix H in} \cite{nye}

\subsubsection{Indicatrix}
Acquiring solutions of the two indices along with the corresponding directions of propogation in the crystal for a general plane wave with unit wave vector $\vec{s}$ can be done via a conveniant geometrical construction. The construction begins by considering a constant electric energy density ($U_e$) surface in the $\vec{D}$ space; an ellipsoid is formed:

\begin{equation}\label{eq:lagr1}
\frac{D_x}{\varepsilon_x} + \frac{D_y}{\varepsilon_y} + \frac{D_z}{\varepsilon_z} = 2 U_e \varepsilon_o
\end{equation}
With redefined coordinates $(\vec{D}/\sqrt{2 U_e \varepsilon_o}) \rightarrow \vec{r}$ and setting $\varepsilon_i = n^2_i$:

\begin{equation}
\frac{x^2}{\varepsilon_x} + \frac{y^2}{\varepsilon_y} + \frac{z^2}{\varepsilon_z} = 1
\end{equation}
This equation for the ellipsoid is known as the indicatrix. Given the co-planar solution demonstrated in the last section, we can impose the normal of the plane $\vec{r} \cdot \vec{s} = 0$:

\begin{equation}\label{eq:lagr2}
\vec{r} \cdot \vec{s} = x s_x + y s_y + z s_z = 0
\end{equation}
Equations \ref{eq:lagr1} and \ref{eq:lagr2} both contribute constraints to the method of finding extrema using Lagrange multipliers for the function:

\begin{equation}
r^2 = x^2 + y^2 + z^2
\end{equation}
The Lagrangian ($\mathcal{L}$) with the introduced multiplers ($\lambda_1$, $\lambda_2$) then becomes:

\begin{equation}
\mathcal{L}(\textcolor{red}{\vec{r},\vec{s}},\lambda_1, \lambda_2) =
x^2 + y^2 + z^2 + \lambda_1 (xs_x + ys_y + zs_z) + \lambda_2 \bigg( \frac{x^2}{\varepsilon_x} + \frac{y^2}{\varepsilon_y} + \frac{z^2}{\varepsilon_z} - 1 \bigg)
\end{equation}
With the generated system of equations from the Lagrange multipler method ($\partial F_i/ \partial x_i = 0$, and $\partial F_j/ \partial \lambda_j$) where index $i =x,y,z$ and $j = 1,2$ we obtain a system of 3 equations:

\begin{equation}
i \bigg(1-\frac{r^2}{\varepsilon_{i}} \bigg) + s_{i} \bigg(\frac{x s_x}{\varepsilon_x} + \frac{y s_y}{\varepsilon_y} + \frac{z s_z}{\varepsilon_z} \bigg) = 0
\end{equation}
The result is verified when substituting $r \rightarrow \frac{\vec{D}}{\sqrt{\vec{E} \cdot \vec{D} \varepsilon_o}}$ back which recovers \ref{eq:3.18}.
\\
\textcolor{red}{Figure here would be nice}
\subsection{Crystal systems}
The following section is intended to provide an adequate crystallographic background to allow the further the discussion of light propogation through $\gaas$ and $\algaas$. This includes, crystal classification, the optical crystal classification of $\gaas$ and $\algaas$ under ``normal" operating conditions, and optical crystal classification of $\gaas$  and $\algaas$ with electrical and mechanical perturbations.

\subsubsection{Crystal classification}
%Crystal classification comes with numerous identifiers all having their place in the field of crystallography but all will not be referred to explicity.

The priority of this section is to provide a crystallographic background to some conventions behind the classification and characterization of crystalline materials through the usage of: Miller indices, unit cells, point groups, Bravais lattices and space groups.


% MORE IN DEPTH HISTORY
% The fundamentals of crystallography began to be developed in the 17th century through macroscopic analysis of perfectly planar fractures of crystals. It was then Saobserved by Nicolaus Steno that the angles between two adjacent faces of a particular crystal species is a constant (Steno's law of constant interfacial angles). In the 18th century René-Just Haüy, a French priest and minerologist, and Jean-Baptiste L. Rome de I'Isle sought to understand the nature of crystals leading to further induced fracturing, better angular measurement precision between crystal cleavage plane normals using a goniometer (confirming Steno's Law), and analysis of the macroscopic symmetries based on the planar crystal faces. Haüy would soon after hypothesize that crystals are composed of polyhedrical molecules arranged in a periodic manner in three-dimensional space. This fundamental geometrical structure is a predecesor of the modern unit cell; the smallest geometric atomic/molecular arrangement that represents the symmetry of the crystal. Alongside Haüy's early interpretation of the unit cell, he was able to note two more crystallographic principles known as the law of rational indices and the law of constancy of symmetry. The law of rational indices carries into the modern day as the second of three laws of crystallography and through the use of Miller indices; a set of three integer values representing intercept values for each of a crystal's cleavage planes.
%\textcolor{red}{Miller index figure for a zincblende structure}
%With these geometric tools and crystallographic rules, crystal structures were further categorized into symmetry groups. In the 19th century, Frankenheim, Hessel, and Gadolin independently hypothesized the existence of 32 different crystal classes correlating to different point groups where one of these point groups represents a unique collection of symmetry elements. The ``point" in point group is the analysis of symmetry operations when a single point of a three dimensional object is held fixed, and a group is a set of those symmetry elements that obey multiplication such that it is associative, there exists an idenitiy element in the set, the inverse of the element is an element of the set, and the product of any two elements is an element of the set \cite{sands}.  The symmetry elements that make up these point groups can be decomposed into operations of rotations, mirror planes, inversions, and improper rotations. A more concise description of these operations and the crystallographic point groups can be seen listed in the appendix \textcolor{red}{?}. The point group analysis performed by Frankenheim, Hessel, and Gadolin is an impressive feat given the accuracy while molecular theory was still maturing. These point groups have different label standards but throughout this document we will use Hermann-Maugin notation for AlGaAs. With established crystallographic point groups, established translational symmetries came soon after through the works of Auguste Bravais in the latter part of the 19th century giving rise to 14 different lattice structures corresponding to different crystallographic classes.  The combination of the point group symmetries along with a group of translational symmetries introduced a new symmetry group known as a space group. With maturity of the molecular theory and the introduction of X-ray diffraction more robust models of these unit cells and lattices were developed soon after.
%\\
%\textcolor{red}{Details on symmetry elements?}
%\\
%\textcolor{red}{Image with cleavaged crystal and crystal unit cell}
%\\
%\textcolor{red}{Space groups? (decomposed into point group and bravais lattice categorization)}


The fundamentals of crystal categorization come with the emergence of a formal study of cyrstals (crystallography) in the 17th and 18th centuries through the macroscropic analysis of planar fractures of crystals (cleavage planes). Nicolaus Steno and René-Just Hauy are credited with the discovery of the three crystallographic laws: law of constant interfacial angles (Steno), the law of rational indices (Haüy), and the law of constancy of symmetry (Haüy) \cite{aroyo}.
\\
\\
 The \textbf{law of constant interfacial angles} states that the angles between two adjacent faces of a crystal species is a constant \cite{steno}.
\\
\\
The \textbf{law of rational indices} states that the intersection of the three intercepts of a crystal plane with the crystallographic axes are constant and can be expressed by rational numbers and whole number multiples of them. Alongside this law Haüy would go on to hypothesize that crystals are composed of polyhedrical molecules arranged in a periodic manner in three-dimensional space \cite{hauy}. This early geometric model of crystals, provided fundamental tools of crystallography that we still use today. The modern unit cell is represented by the smallest geometric atomic/molecular arrangement that represents the symmetry of the crystal. When representing a more macroscopic form of the crystal these unit cells are assembled periodically in 3d space along a grid of points known as a lattice\cite{sands}.
\\
\\
The \textbf{law of constancy of symmetry} states that crystals of a particular substance maintain the same symmetry elements \cite{hauy}. This discovery garnered interest in the 19th century to classify crystal structures under the framework of symmetry groups \textcolor{red}{more on groups in appendix}. Point group symmetries were first addressed and sorted crystals into 32 crystallographic point groups. Soon after translational symmetries were addressed by Auguste Bravais with 14 different lattice structures which also implicitly displayed a formal connection between unit cell geometry and lattice structures \cite{aroyo}. The combination of these point group symmetries and translational symmetries addressed by the Bravais, form a new group known as the crystallographic group: a subset of the more generally space group \textcolor{red}{for more details on point groups, Bravais lattices and space groups, see appendix}.
\\
\\

% What are the symmetries that exist in three dimensions?
% What are the geometric representations given by known crystals?
% What symmetries are left after performing all prescribed three dimensional symmetry operations on known crystal geometries?

%When observing a crystal on a molecular level, a periodicity in three dimensional space isfeyfe seen containing a specific a specific collection of atoms with a specific orientation. The smallest unique atomic/molecular arrangement describing the crystal's atomic makeup and structure is known as a motif. And the periodicity can be aligned along a three dimensional geometrical construction called a lattice. Crystallographers categorize crystal systems analyzing the three dimensional point-group symmetries of these motifs and space-group symmetries of the crystal's collective periodic arranged motifs. A periodic structure on which the unit cells are arranged or lattice is a conveniant means to organize and quantize the space when describing and characterizing the structure. Crystallographers characterize the detail of a specific unit cell through observed symmetries after a series of prescribed operations. The symmetries that exist for the unit cell as well as the symmetries after lattice translations define a space group.

\subsubsection{$\gaas$ and $\algaas$ classifications}
The space group of $\gaas$ as well as $\algaas$ are within the $F\bar{4}3m$ space group. Crystals with this particular space group are commonly known as zincblende crystals; a common crystal configuration named after zinc sulfide (ZnS). Cubic crystals by their crystallographic structure display optically isotropic characteristics when no DC and/or slowly varying electric fields are present and are not mechanically perturbed.
\\
\textcolor{red}{Insert figure of GaAs structure (Shapr3D construction)}
\\
\textcolor{red}{Mention the difference in lattice cell constant between $\gaas$ and $\algaas$}
\\
\subsubsection{Perturbed $\gaas$ and $\algaas$}
\textcolor{red}{AlGaAs is a cubic crystal, optically isotropic BUT birefringence can be induced (displays characteristics of an uniaxial crystal)}
\\
\satoshi{Is this true?} \textcolor{blue}{Satoshi, I believe so. There seem to be some geometric differences between $\gaas$ and $\algaas$ in terms of Bravais lattices but Adachi claims it's essentially zincblende. To verify it's important to know how the change in the Bravais lattice might change the unit cell and if it's relevant. Otherwise the birefringence we have seen upon installation was stress induced.}
\\
\textcolor{red}{Nye (Chapter 8.2)}
\\
\\
\textcolor{red}{Born and Wolf (Chapter 15)}
\\
\subsubsection{The pockels effect}
Some anisotropic crystalline media (ACM) exhibit characteristics where the indicatrix changes as a function of a slowly (\satoshi{is this statement correct?}) \textcolor{red}{I can very much word this better. Right now it probably is not correct.} (\textcolor{red}{up to the speed of sound within the media?}) varying electric field ~\cite{yariv,nye}.
\\
\textcolor{red}{Indicatrix with electro-optic coefficients}
\\
Relationship between a slowly varying electric field vector of an arbitrary direction $\vec{E} = [E_1, E_2, E_3]$ to the perturbations of the indicatrix:
\\
\begin{equation}
  \left[ {\begin{array}{c}
   \big( \frac{1}{\Delta n ^2 } \big)_1 \\
   \big( \frac{1}{\Delta n ^2 } \big)_2 \\
   \big( \frac{1}{\Delta n ^2 } \big)_3 \\
   \big( \frac{1}{\Delta n ^2 } \big)_4 \\
   \big( \frac{1}{\Delta n ^2 } \big)_5 \\
   \big( \frac{1}{\Delta n ^2 } \big)_6 \\

  \end{array} } \right]
  =
%
 \left[ {\begin{array}{ccc}
   r_{11} & r_{12} & r_{13}\\
   r_{21} & r_{22} & r_{23}\\
   r_{31} & r_{32} & r_{33}\\
   r_{41} & r_{42} & r_{43}\\
   r_{51} & r_{52} & r_{53}\\
   r_{61} & r_{62} & r_{63}\\
  \end{array}} \right]
 %
 \left[{\begin{array}{c}
   E_1\\
   E_2\\
   E_3\\
 \end{array}} \right]
\end{equation}

The six indices correspond to the six unique elements in the indicatrix.
\\
\\
\textcolor{red}{Nye (Chapter 8.2)}
\\
\\
\textcolor{red}{Yariv (Chapter 14)}
\\
\\
\textcolor{red}{ How the modulation of the phase of the carrier field is dependent on the orientation of its wave vector with respect to the crystal structure, the modulating electric field direction and strength, (other items to discuss in terms of introducing the effect)}
\\
This property of ACM is what allows electro-optic modulators to operate as light phase modulators

\subsubsection{Electro-optic tensor for zincblende structures with $\bar{4}3m$ point group symmetry}
\begin{equation}
 \left[ {\begin{array}{ccc}
   0 & 0 & 0\\
   0 & 0 & 0\\
   0 & 0 & 0\\
   r_{41} & 0 & 0\\
   0 & r_{41} & 0\\
   0 & 0 & r_{41}\\
  \end{array} } \right]
\end{equation}
\\
$r_{41} = r_{52} = r_{63}$
\\
For GaAs @ $10.6\mu$ $r_{41} = 1.6 \times 10^{-12}$ [m/V]
\\
Adachi estimate for $\mathrm{Al_{x}Ga_{1-x}As}$?


\subsubsection{Specific propogation (Which Miller index?) direction of 1064 nm light through GaAs and AlGaAs}



\section{Phase modulation of light propogating through a anisotropic crystalline thin film stack}

\subsection{Coating parameters}

\textcolor{red}{Is the sample that we have a standard AlGaAs sample from Thorlabs/CMS?}


\subsubsection{Growth orientation (Miller indices) with respect to substrate surface}
\begin{itemize}
\item Mirror surface is the [100] plane.
\item Within the [100] plane the AlGaAs coating is grown with a flat indicator that draws a line within the [0-11] plane where the surface normal is pointing towards the sample center.
\end{itemize}


\subsubsection{Layering}
The coating to be studied consists 36 $\lambda$/4  thick layers of GaAs interspersed with 35 layers of $\lambda$/4 thick AlGaAs.   GaAs forms the top and bottom layer to protect the AlGaAs from absorbing oxygen from the air.
High Index:  GaAs, n=3.480, layer thickness is 76.43 nm
Low Index:  $ \mathrm{Al}_{0.92} \mathrm{Ga}_{0.08} \mathrm{As} $, n=2.977, layer thickness is 89.35 nm
\textcolor{red}{Info from Steve. Written source?}


\subsection{Analytical approximation}
Fejer and Bonilla take an analytical approximation approach when finding the impact of the electric field to the change in phase of the light through a crystalline anisotropic thin film ($\lambda/4$) stack \cite{bonilla_fejer}.

\begin{equation}
\hat{\phi}' = \frac{\pi n_1 z}{1-z^2}(z^{2N} -1) \frac{z^{2N} \frac{(n_f)^2}{n_2 n_3}(n_2 \kappa_{\gamma 2} + n_3\kappa_{\gamma 3}) - (n_2 \kappa_{\gamma 3} + n_3\kappa_{\gamma 3})}{(n_1)^2 -(n_f)^2 z^{4N}}
\end{equation}

with $z = \frac{n_2}{n_3}$
and
$
\kappa_{\gamma j} = \frac{d}{d \gamma} \mathrm{log}(n_j h_j)|_{\gamma =\gamma_{O}} \bigg(\frac{\hat{n}_j'}{\hat{n}_j} +\frac{\hat{h}_j'}{\hat{h}_j} \bigg)
$

With $\kappa$ being a scalar parameter.

\satoshi{Adding a schematic would be helpful.}

\textcolor{red}{Figure is in the works}

\subsection{Numerical estimate}

In the appendix of \cite{ballmer2015} Ballmer explains a formalism that allows the coating transfer function given a differential change in phase due to a scalar parameter to be expressed as follows:

\begin{equation}
M = Q_N D_N ...Q_kD_k...Q_1D_1Q_0
\end{equation}

For a single pass through the coating where the coating thickness and indices are left general.

\satoshi{Both approaches are essentially based on the same calculation method i.e., transfer matrices.}

\section{Projected DARM coupling}
To estimate how much DARM coupling can occur, we use use a measured field spectra acquired from installed electric field meters located within LIGO Hanford Observatory EX and EY vacuum chambers. Taking the upper and the lower EFM measurements in $.3\; [\mathrm{V}/\mathrm{m}/\sqrt{\mathrm{Hz}]}$ @ 60 Hz and $4\times10^{-3}\; [\mathrm{V}/\mathrm{m}/\sqrt{\mathrm{Hz}]}$ @ 4kHz ~\cite{efm_log}.
\satoshi{I don't think these values are calibrated. According to Martynov et al. 2016, the fluctuations in the electric filed is $\sim10^{-5}\,\mathrm{[(V/m)/\sqrt{Hz}]}$.}
This along with computed estimate above allows us to create an upper limit for what this noise might be assuming incoherent fields between the end stations and a flat frequency response within LIGO's bandwidth.

\section{Experiment}
To probe for this effect, we took the approach of an in-air Fabry-P\'{e}rot cavity with a GaAs/$\mathrm{Al_{.92}Ga_{.08}As}$ coated sample end mirror mounted in a custom designed MACOR mount with the ability to install electrodes maintaining a fixed distance from the sample surfaces for a controlled E-field injection. Details and specifications are discussed as well as relevant measurements. An improved experimental design is also detailed for a higher sensitivity measurement.

\subsection{Design}
This section details the Pound-Drever-Hall servo which is at the heart of this experiment, the on-table design, a sample / electrode assembly used for a controlled electric field injection, and electrode drive parameters
\subsubsection{PDH servo}
The Pound-Drever-Hall technique, originally imagined for laser frequency stabilization to an ultra-stable length reference, achieves constant resonance with said reference by tracking the phase of the cavity input light with respect to the light reflected from the cavity. Tracking the approximately linear phase response across resonance is at the heart of this technique and recognizes that physical property to be probed around a cavity's resonance cannot be the intensity of the transmitted or reflected light because of the quadratic symmetry. (\textcolor{red}{Derivative of the intensity could work but depends on us operating off resonance})
\\
\textcolor{red}{Insert figure of reflected intensity and phase response around resonance (cavity transfer function). Maybe also have a figure corresponding to what this could represent in a lab. (sweeping a mirror of the laser cavity through resonance)}
\\
Another novelty and insight to the technique is the use of phase modulation onto the carrier field which is the mathematical and physical equivalent to imposing separate optical fields which in most cases do not resonate in the optical cavity.
\\
\begin{equation}
E_\mathrm{inp} = E_o e^{i \omega t + \beta \mathrm{sin}( \Omega t)}
\end{equation}

\textcolor{red}{Approximation with Bessel functions (assumptions about modulation depth)}
\begin{equation}
E_\mathrm{inp} \approx E_o \satoshi{E_0} [J_o\satoshi{J_0}(\beta)e^{i \omega t} + J_1(\beta)e^{i (\omega + \Omega) t} - J_1(\beta)e^{i(\omega -\Omega)t}]
\end{equation}
\\
\textcolor{red}{Discuss how the measurement of the beat power between the sideband and the carrier at the RFPD tracks the phase and how with demodulation and filtering, you can create an error signal.}

\begin{figure}[H]
\includegraphics[width=\textwidth]{figs/ALGAAS/electrooptic_study_algaas_0.pdf}
\caption{Simplified experiment schematic}
\label{fig:poisson_output}
\end{figure}

\subsubsection{On-table schema}

\textcolor{red}{A figure that highlights path that is intended for locking onto the PMC, and a different color highlight for the path to the experiment}

The setup presented in this section, branches off of a previously established optical path used to lock onto a separate optical cavity (PMC).
\\
The laser light source is a Mephisto 2000 NE 1064nm laser.
\\
25 MHz EOM is a New Focus Model 4003 IR resonant phase modulator.
\\
The designed cavity is a 0.1651 m long cavity with a HR IBS coated sample (PL-CC, ROC =0.333 m) input coupler from CVI Melles-Griot and a GaAs/$\mathrm{Al_{.92}Ga_{.08}As}$ (PL-PL) fused silica substrate by the Crystalline Mirror Solutions (CMS) division of Thorlabs with the aforementioned parameters (\textcolor{red}{mentioned in coating parameters section}).

\subsubsection{Sensing S(f)}
\begin{itemize}
\item 25 MHz RFPD
\begin{itemize}
\item Transimpedance measurement (necessary? or should I just use the mixer out PDH to summarize PD/mixer response)
\end{itemize}
\item Frequency Stabilization servo (modified MIT FSS (DCCD980536)) (LTspice model in appendix)
\end{itemize}


\subsubsection{Actuation A(f)}
\begin{itemize}
\item Mephisto 2220 PZT response (capacitance estimated from HVA drive measurement with and without connection to PZT)
\item Channel 3 of SVR 350-3 BIP High Voltage Amplifier from Piezomechanik GmbH with Pomona box (elog 412)
\item \textcolor{red}{Figure of frequency response of A(f)}

\end{itemize}

\subsubsection{Low frequency servo (Thermal loop)}
\begin{itemize}
\item Passed signal from FSS $\rightarrow$ integrators $\rightarrow$ Laser thermal actuator input
\end{itemize}

\subsubsection{OLG(f)}
Isn't quite $\mathrm{A}(f)*\mathrm{S}(f)$ as stated. Doesn't entirely account for the optical plant.
How the measurement is taken (important to take between installations to account for the changes in the optical plant) (elog 831)


\subsubsection{Sample / Electrode assembly}
Maximizing the electric field ($|E_z|$) and within the coating while requiring a through beam to and through the HR coating lead us to imagine disk electrodes with a 3mm central aperture. The aperture size was chosen to be at least 5 times larger than the beam size at the plate locations. This was to avoid any beam clipping while still allowing to maximize the field strength at the region of interest.
\\
Most commercial optical mounts are conductive which proved to be a problem when attempting to find a mounting solution while reducing the non-normal field gradients within the volume of interest around the sample. Because of this, we chose to construct an optical mount made of MACOR a machinable ceramic with high a high Young's modulus (66.9 GPa), and a moderate Poisson ratio (.29) \cite{macor}. An optical mount for the sample made with MACOR, along with glass bearnings .48 $\pm$ .01 cm $\diameter$  and a McMaster-Carr 8-32, 1/2" ceramic screw were used to clamp and suspend the optical sample. A 1.24" $\diameter$ hole was bored into the MACOR with a (\textcolor{red}{depth?}) depth so that there is a ? mm clearance between the front and back surface of the sample to the electrode plates.
\textcolor{red}{Figure with  the sample in-situ}
\\
\begin{figure}[H]
\centering
\includegraphics[width=.75\textwidth]{figs/ALGAAS/macor_assembly.jpeg}
\caption{Placeholder for more updated MACOR assembly}
\label{fig:Ez}
\end{figure}
There are two relevant configurations of this experiment: 1) an all-in-one MACOR assembly where the electrodes are mechanically coupled to the optical mount, and 2) larger mechanically decoupled electrode plates.
\\
\\
\textcolor{red}{A lot of time was dedicated towards preliminary mounts made of PLA and PETG. Do I want to do updated measurements and make statements about noise produced from these mounts?}
\\
\\

\subsection{$|E_z|$ strength estimate}
To convey the problem at hand, it is useful to review the illustration seen \textcolor{red}{here (figure showing the the electrode plates, and sample with AlGaAs coating}
\subsubsection{Math}
To find the Electric field screened by the coating we begin with Gauss' Law:

\begin{equation}
\nabla \cdot D = \rho_\mathrm{free}
\end{equation}

For our problem we assume no free charge, but the fused silica substrate with the AlGaAs coating presents dielectric material between the plates. Our initial boundary conditions are also expressed in terms of plate potentials so it is natural to first solve for the potential ($V$) for every point within our system. We can exploit the cylindrical symmetry with the optic and plate geometry in the $r$ coordinate so we shall express the Laplacian accordingly:
\begin{equation}
(1-\chi)\bigg[\frac{1}{r}\frac{\partial}{\partial r} \bigg( r \frac{\partial}{\partial r}\bigg) + \frac{\partial^2}{\partial z^2}\bigg]V = 0
\end{equation}
Where $\chi$ is a spatially dependent electric susceptibility. (\textcolor{red}{Establish coordinates for $\gaas$/$\algaas$, as well as the fused silica substrate so the computation is transparent})
\\
\satoshi{Definition of $\rho$ must be explained. $\rho$ and $\rho_{\mathrm{free}}$ are confusing.
Define $\chi$ and $V$.}

\noindent\textcolor{red}{I will change $\rho$ to $r$ am going to modify figure text so it matches soon.}

Utilizing this, we can proceed to a construction of a numerical Laplacian.

\subsubsection{Numerical approximation}


\begin{itemize}
\item Potential map computation in cylindrical
\item Computing $E_z$ from potential map
\begin{itemize}
\item inside coating
\item outside coating
\end{itemize}
\end{itemize}


\begin{figure}[H]
\includegraphics[width=\textwidth]{ALGAAS/13-Sep-2021_potential_map}
\caption{Poisson calculator output potential map ($V(z,r)$ in cylindrical coordinates)}
\label{fig:poisson_calc_output}
\end{figure}

\begin{figure}[H]
\includegraphics[width=\textwidth]{ALGAAS/13-Sep-2021_e_field_inside_outside_normal}
\caption{$|E_z|$ screened by the scoating and immediately outside AlGaAs coating. \textcolor{red}{Needs to be updated with more current settings}}

\satoshi{How large applied voltage is assumed?}

\label{fig:Ez}
\end{figure}

\begin{figure}[H]
\centering
\includegraphics[width=.75\textwidth]{figs/ALGAAS/HVA_TREK1010BHS_1260V_out.png}
\caption{TREK 10/10B-HS HVA frequency dependent measurement. Using Poisson calculator to estimate field strength within coating. (\textcolor{red}{Just HVA for now but will update.} \textcolor{red}{Also, assumes a flat response from coating within this studied region (is this a good assumption or could I do better? (dielectric frequency dependence))}}
\label{fig:Ez}
\end{figure}

\subsection{Calibration}
As discussed, we know that the error signal spectra provides us a voltage spectra that with the above information about the Plant/servo electronics, allows us to
$\mathrm{VFSSOUT}_\mathrm{rms}/\sqrt{Hz} \rightarrow m_\mathrm{rms}/\sqrt{\mathrm{Hz}}$

$$\Delta \mathrm{L} = \mathrm{source}*\alpha(f) \mathrm{A}(f)*\frac{1+\mathrm{OLG}(f)}{\mathrm{OLG}(f)}*\frac{\mathrm{L_{cav}}}{f_\mathrm{laser}}$$

\subsection{Noise Floor}
\subsubsection{Various noise contributions that add up to measured noise floor}
\subsubsection{Shot noise}
Look at the derived shot noise estimate in the appendix V in the Black paper \cite{black_pdh}

\subsubsection{Laser frequency noise}

\begin{itemize}
\item Measure with initial LIGO PMC?
\item There is also a spectra in the Mephisto laser spec sheets
\end{itemize}

\subsubsection{Residual gas noise}


\subsubsection{Mount noise (3D printed mount mechanical noise)}

\subsection{Drive coupling}

\begin{figure}[H]
\includegraphics[width=\textwidth]{figs/ALGAAS/cav_polarization_test.png}
\caption{Figure that will include the displacement noise floor, (pockels estimate)*(poisson calculator estimate)*(HVA drive frequency dependence), and the drive coupled measurement \textcolor{red}{figure size needs to be increased}}
\label{fig:measurement_sum}
\end{figure}

\subsubsection{Opto-mechanical coupling}
Sample and mount mechanical mode excitations. Seen with both AlGaAs and a HR coating from an AtFilm (IBS coating)
\begin{itemize}
\item \textbf{Vibration of plates (Leissa)} \cite{leissa} Computing frequencies and order of magnitude
\item \textbf{Steve's COMSOL model results}
\end{itemize}

\subsubsection{Proposed alternative measurement schema}
