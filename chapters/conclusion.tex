% Adaptive optics comissioning for O3
% AlGaAs Electro-optic Effect

\section{Adaptive optics comissioning}
Since the comissioning that had taken place in O3a, there has been heavy consideration into how point absorbers manifest on the core optic surfaces and ways to mitigate their effects. Some of these considerations include but are not exlusive to: non-invasive pre-installation measurements of the ITM surface / coating quality \cite{dcc:paNSFupdate2022}, upgrades to TCS to expand upon the current thermal compensation actuation modes, and scheduled vaccum chamber venting specifically for replacing an offending test mass mirror. 
\\
\\
Increasing interferometer input power is an inevitability to reaching designed detector sensitivity and developing mode matching contingency plans is a natural progression of the current adaptive optics schema. The technique for improving the ring heater transient response represents an iterative step of pushing the existing and future thermal mode matching infrastructure towards a larger scale adaptive optics feedback schema. Some next immediate steps within the author's limited view:

\begin{itemize}
	\item Improved measurement and filter fitting of the RH thermo-optic response
	\item Improving the thermo-optic transient of the CO2 actuators with unique pre-filters using the procedure listed in \autoref{appendix:rhcontrolpf} 
	\item Improving TCS synchronization when raising input power to potentially improve the thermo-optic carrier transient from the high circulating arm power.
	\item Optimization of simultaneous thermo-optic transients between actuators and the carrier 
\end{itemize}

\section{\texorpdfstring{$\gaas / \algaas$}{gaas/algaas} Electro-optic noise}
Though with limited sensitivity, an upper limit with the experiment is established. The discovery of driven mechanical couplings within the longitudinal pockels cell mount are discussed and were shown to be a major limitation with driven electric field injections indicated and have lead to an improved dual-polarization locked experimental design for improved sensitivity. Some additional considerations that can be taken with similar experiments:

\begin{itemize}
	\item Improved mechanical design for improved measurement SNR (esp. between 10 Hz to 1kHz)
	\item Further modelling of opto-mechanical resonances to study a possible separation (if any) between the photo-elastic and electro-optic effects.  
\end{itemize}
