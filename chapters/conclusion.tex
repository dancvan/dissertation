% Adaptive optics comissioning for O3
% AlGaAs Electro-optic Effect

\section{Adaptive optics comissioning for O3}
Since the comissioning that had taken place in O3a, there has been heavy consideration into how point absorbers limit the high power operation of interferometric gravitational wave detectors and hence their sensitivity. Some considerations include but are not exlusive to: pre-installation measurements of ITM surface quality, upgrades to TCS for higher order thermal compensation modes, and halting observing altogether to replace an offending test mass mirror. 
\\
\\
While the increasing interferometer input power is an inevitability to reaching designed detector sensitivity, developing mode matching contingency plans seem to be a natural progression of the current adaptive optics schema. The technique for improved ring heater transient response represents a small iterative step of pushing the existing and future thermal mode matching infrastructure towards a larger scale evolving adaptive optics schema. Some next immediate steps within the author's limited view:

\begin{itemize}
	\item Improved measurement and more methodical filter fitting of the RH thermo-optic response
	\item Improving the thermo-optic transient of CO2 actuation
	\item Coordinating with other interferometer operations when raising input power to potentially improve the thermo-optic transient from the high circulating carrier power.
	\item Potential filter switching to optimize the simultaneous thermo-optic transients between actuators and the carrier 
\end{itemize}

\section{AlGaAs Electro-optic effect}
Though with limited sensitivity an upper limit with the experiment is established. The discovery of driven mechanical couplings within the longitudinal pockels cell mount are discussed and were shown to be a major limitation with driven electric field injections indicated and have lead to an improved dual-polarization locked experimental design for improved sensitivity.

